\documentclass[full]{l3doc}
\usepackage[scheme=plain, fontset=ubuntu]{ctex}
\usepackage{multicol}
\usepackage{adjustbox}
\usepackage{zhlineskip}
\usepackage{enumitem}
\usepackage{indentfirst}
\usepackage{titling}
\usepackage{geometry}
\usepackage{tabularray}
\usepackage{xcolor}
\usepackage{tcolorbox}
\tcbuselibrary{listings, skins, raster, breakable}
\usepackage{tabularray}

\def\zhdate{2022年1月10日}
\def\endate{2022/01/10}
\def\version{v1.4}

\geometry{
  left=4.5cm,
  right=2cm,
  top=2cm,
  bottom=2cm,
}
\hypersetup {
  CJKbookmarks,
  bookmarksopen,
  bookmarksopenlevel=3,
  pdfstartview=FitH,
  pdfinfo = {
   Title = The package 'dbshow' ,
   Subject = A LaTeX package ,
   Author = Li Changkai
 }
}

\IndexPrologue {
  \part*{Index}
  \markboth{Index}{Index}
  \addcontentsline{toc}{part}{Index}
  The~italic~numbers~denote~the~pages~where~the~
  corresponding~entry~is~described,~
  numbers~underlined~point~to~the~definition,~
  all~others~indicate~the~places~where~it~is~used.
}

\GlossaryPrologue {
  \part*{Change~History}
  {\GlossaryParms\ttfamily\hyphenchar\font=`\-}
  \markboth{Change~History}{Change~History}
  \addcontentsline{toc}{part}{Change~History}
}

\let\subsubitem\subitem

\DoNotIndex{\begin, \end}
\setlength{\parskip}{\medskipamount}

\newcommand\tikzmark[1]{\tikz \coordinate[overlay, remember picture] (#1);}
\def\levelchar{?}
\def\orbar{\textup{\textbar}}
\def\defaultval#1{\textbf{\textup{#1}}}
\def\TF{true\orbar false}
\def\TTF{\defaultval{true}\orbar false}
\def\TFF{true\orbar\defaultval{false}}
\def\zhbefore{\textbf{前}}
\def\zhafter{\textbf{后}}
\def\enbefore{\textbf{before}~}
\def\enafter{\textbf{after}~}
\DeclareDocumentEnvironment { note } { +b } {
  \par\textbf{\textsf{NOTE:~}}#1\par
} {}

\AtEndDocument{
  \newgeometry{
    left=2cm,
    right=2cm,
    top=2cm,
    bottom=2cm
  }
  \PrintChanges
  \PrintIndex
}

\ExplSyntaxOn
\makeatletter
\cs_new_protected:Nn \dbshow_changes:nnnn {
  \@bsphack\group_begin:\@sanitize
  \catcode`\\\z@ \catcode`\ 10 \MakePercentIgnore
  \exp_args:Nx \glossary {
    \exp_not:N \textbf{#1}\levelchar
    \exp_not:N \textit{#2}\c_space_tl
    \exp_not:N \textbf{#3}\c_colon_str\c_space_tl
    \exp_not:n { #4 }
  }
  \group_end:\@esphack
}
\cs_generate_variant:Nn \dbshow_changes:nnnn { xnnn }

\cs_new_protected:Nn \dbshow_changes_what:nnnn {
  \regex_extract_once:nnNT { doc|macro|bug|option }
  { #3 } \l_tmpa_seq {
    \dbshow_changes:xnnn {
      \str_case_e:nn { \seq_item:Nn \l_tmpa_seq { 1 } } {
        { doc } { Documentation }
        { macro } { Macro }
        { bug } { Bug }
        { option } {  Option }
        { logic } {  Logic }
      }
    }
      {#2} {#3} {#4}
  }
}

\cs_new_protected:Nn \dbshow_changes_how:nnnn {
  \regex_extract_once:nnNT { Add|Update|Remove } { #3 } \l_tmpa_seq {
    \dbshow_changes:xnnn { \seq_item:Nn \l_tmpa_seq { 1 } }
      {#2} {#3} {#4}
  }
}

% #1 version #2 date #3 type #4 desc
\DeclareDocumentCommand \changes { m m m m } {
  \dbshow_changes:nnnn {#1} {#2} {#3} {#4}
  \dbshow_changes_what:nnnn {#1} {#2} {#3} {#4}
  \dbshow_changes_how:nnnn {#1} {#2} {#3} {#4}
}
\makeatother

\DeclareDocumentEnvironment { Description } { o +b } {
  \hbox_set:Nn \l_tmpa_box { #1 }
  \dim_set:Nn \l_tmpa_dim { \box_wd:N \l_tmpa_box }
  \begin{itemize}[
    itemindent=0pt,
    labelindent=0pt,
    labelwidth=\l_tmpa_dim,
    leftmargin=!,
    align=left]
    #2
  \end{itemize}
} {  }

\cs_new_protected:Nn \dbshowdoc_function_begin:N {
  \cs_set_eq:NN \__codedoc_tmp_cs:nN \__codedoc_typeset_function_block:nN
  \cs_set_protected:Npn \__codedoc_typeset_function_block:nN ##1##2 {
    \__codedoc_function_label:xN {##1} ##2
    \hbox_set:Nn \l_tmpa_box {##1}
    \int_compare:nTF { \str_count:n {##1} <= 22 }
      {##1} { \adjustbox{width=.7\marginparwidth, height=\box_ht:N \l_tmpa_box}{##1} }
    #1{##1}
    \__codedoc_typeset_expandability: \\
  }
}
\cs_new_protected:Nn \dbshowdoc_function_end: {
  \cs_set_eq:NN \__codedoc_typeset_function_block:nN \__codedoc_tmp_cs:nN
}
\DeclareDocumentEnvironment { option } { O{} +v }
  {
    \dbshowdoc_function_begin:N \SpecialOptionIndex
    \__codedoc_function:nnw {#1} {#2}
  }
  {
    \__codedoc_function_end:
    \dbshowdoc_function_end:
  }
\DeclareDocumentEnvironment { environment } { O{} +v }
  {
    \dbshowdoc_function_begin:N \SpecialEnvIndex
    \__codedoc_function:nnw {#1} {#2}
  }
  {
    \__codedoc_function_end:
    \dbshowdoc_function_end:
  }

\DeclareDocumentCommand \opt { O{} m }
  { \__codedoc_cmd:no {#1} { #2 } }

\NewDocumentCommand \linktarget { m m m } {%
  \hyperlink{#1}{#3}%
  \raisebox{1em}{\hypertarget{#2}{}}%
}
\ExplSyntaxOff


\begin{document}
\title{
  \pkg{dbshow} 宏包 \version%
  \protect\footnote{%
    代码仓库:\url{https://github.com/ZhiyuanLck/dbshow},
    QQ群:788706534}
  \rlap{\makebox[4cm][r]{
    \normalsize $\Longrightarrow$ \color{red}
    \linktarget{en}{zh}{English Version}
  }}
}
\author{\textit{李昌锴} \texttt{<lichangkai225@qq.com>}}
\date{\zhdate}
\maketitle

\tableofcontents

\begin{documentation}

\section{引言}
编写本宏包的动机来源于当前没有一个很好的错题本宏包,可以方便的根据各种条件对错
题进行筛选、排序,然后以自定义的样式展示出来。\pkg{dbshow} 宏包实现了四个核心
功能:数据存储和使用、数据筛选、数据排序、数据展示。

数据只需要存储一次,就可以通过预定义的筛选、排序条件和样式展示部分或全部的数据。
如上所述,本宏包其实实现了一个非常简单的数据库,复习错题的功能只是其中一个应用,
和其他数据库宏包比如 \pkg{datatool} 相比,\pkg{dbshow} 更专注于非图表类型的数
据展示。

\changes{1.4}{2022-01-10}{Add check}{version of \pkg{l3kernel}}
\pkg{dbshow} 依赖版本日期至少为 |2022-11-07| 的 \pkg{l3kernel}。

\changes{1.2}{2022-01-09}{Add doc}{descripton for expansion}
\begin{itemize}
  \item 名字后带有 $\star$ 的命令是可以完全展开的(fully-expandable);
  \item 名字后带有 \ding{73} 的命令可以有限制地展开(restricted-expandable);
  \item 名字后不带有特殊字符的命令是不可展开的(non-expandable);
  \item 名字后带有 $\star$ 的选项不影响相关的代码的是否可展开;
  \item 名字后带有 \ding{73} 的选项是否影响相关代码的可展开性取决于选项的设置;
  \item 名字后不带有特殊字符的选项会使与之相关的代码变得不可展开。
\end{itemize}

\subsection{数据类型}

宏包基于 \pkg{expl3} 的基础类型构建了6种类型:
\begin{Description}[\texttt{clist}]
  \item[\texttt{date}]
    日期类型,以 |yyyy/mm/dd| 形式存储,支持大小比较,排序(转
    换成字符串)。默认值为 \cs{dbtoday}。
  \item[\texttt{str}]
    字符串类型,支持正则匹配,英文排序。默认值为空。
  \item[\texttt{tl}]
    \meta{token list}类型,支持正则匹配。默认值为空。
  \item[\texttt{int}]
    整数类型,支持大小比较,排序。默认值为0。
  \item[\texttt{fp}]
    浮点数类型,支持大小比较,排序。默认值为0。
  \item[\texttt{clist}]
    逗号分隔的列表类型。默认值为空列表。
\end{Description}

\changes{1.3}{2022-01-08}{Remove dependency}{\pkg{datatime2}}
除了日期类型,所有类型都是 \pkg{expl3} 的内置类型。\pkg{dbshow} 构建了一个简单
的 |date| 类型,支持转换成整数以及带样式的打印。

\subsection{与 \pkg{datatool} 的区别}

\changes{1.2}{2022-01-07}{Add doc}{add comparison to \pkg{datatool}}

从核心功能上看,\pkg{dbshow} 和 \pkg{datatool} 实现了相同的功能。区别在于
\pkg{dbshow} 基于 \pkg{expl3} 实现,支持字符串的正则匹配,还支持多级排序。使用
方式上更倾向于样式与内容分离,所有的样式都可以通过选项提前定义好并且可以复用。
\pkg{dbshow} 并没有实现从外部文件读取数据以及将数据持久化的功能,我认为这些应
该是更专业的外部程序的工作而不应该在 \LaTeX 中设计这些功能。因此,\pkg{dbshow}
只提供了一个运行时的临时数据库,足够轻便且满足大部分正常需求。如果你想删除或修
改数据库中某一条记录,请去对应的位置删除或修改掉对应的 \env{dbitem} 环境,而不
是让宏包提供一个输出某一行记录的命令。某种意义上记录数据库的 \TeX 源文件本身就
是数据的一种持久化。

\section{接口文档}

\subsection{创建、展示和清空数据库}
\begin{function}[added=2022-01-05]{\dbNewDatabase, \dbNewDatabase*}
  \begin{syntax}
    \cs{dbNewDatabase} \oarg{base database} \marg{database} \{ \\
    ~~\meta{attr1} = \meta{type spec1}, \\
    ~~\meta{attr2} = \meta{type spec2}, \\
    ~~\ldots{} \\
    \} \\
    \cs{dbNewDatabase}* \marg{database} \{ \\
    ~~\meta{attr1} = \meta{type spec1}, \\
    ~~\meta{attr2} = \meta{type spec2}, \\
    ~~\ldots{} \\
    \} \\
  \end{syntax}

\end{function}

  新建一个数据库,不带星号的版本可以指定一个数据库来继承其属性设置,该版本总是
  会舍弃掉之前的定义。

  带星号的版本不会舍弃之前已有的定义,而是将新的选项添加到后面。

  \meta{attr} 为属性名称,\meta{type spec} 负责声明属性类型和属性默认值:

  \noindent\begin{tblr}{
    colspec = {ll},
    column{1} = {font = \ttfamily}
  }
    \meta{attr} = \meta{type} &
    将 \meta{attr} 声明为 \meta{type} 类型 \\
    \meta{attr} = \meta{type}\orbar\meta{default} &
    将 \meta{attr} 声明为 \meta{type} 类型,并且将默认值设置为 \meta{default}。
    \\
  \end{tblr}

  \begin{note}
    每个数据库都有一个默认的属性 |id| 用来存储数据的索引。
  \end{note}

  下面是定义一个错题数据库的示例,|question| 和 |answer| 属性用来存储问题和答
  案,|date| 属性存储日期,|info| 属性存储额外信息,|labels| 存储题目标签。
\begin{verbatim}
  \dbNewDatabase{ques}{
    question = tl,
    answer = tl,
    date = date,
    info = tl,
    labels = clist
  }
\end{verbatim}

\begin{function}[added=2022-01-05]{\dbshow}
  \begin{syntax}
    \cs{dbshow} \marg{style} \marg{database}
  \end{syntax}

  使用 \meta{style} 样式来展示 \meta{database}。
\end{function}

\changes{1.2}{2022-01-07}{Add macro}{\cs{dbclear}}
\begin{function}[added=2022-01-07]{\dbclear}
  \begin{syntax}
    \cs{dbclear} \marg{database}
  \end{syntax}
  清空 \meta{database} 里的所有内容。
\end{function}

\subsection{\cs{dbNewStyle} 和样式选项}

\begin{function}[added=2022-01-05]{\dbNewStyle}
  \begin{syntax}
    \cs{dbNewStyle} \oarg{base styles} \marg{style} \marg{database} \marg{opts}
  \end{syntax}

  为 \meta{database} 定义一个新的样式 \meta{style},该样式可以基于已有的样式
  \meta{base styles},比如 |\dbNewStyle[base1, base2]{new-style}{ques}{}|。
\end{function}

\subsubsection{通用选项}

\begin{option}[added=2022-01-05]{filter}
  \begin{syntax}
    filter = <filter>
  \end{syntax}

  为当前样式设置由 \cs{dbCombineFilters} 所定义的过滤器
\end{option}

\begin{option}[added=2022-01-06]{raw-filter}
  \begin{syntax}
    raw-filter = <conditional expression>
  \end{syntax}

  使用条件表达式设置匿名过滤器,这里的条件指通过 \cs{dbNewConditional} 定义的
  条件。下面代码中两个示例的过滤器具有相同的功能。
\end{option}

\changes{1.1}{2022-01-06}{Add option}{\opt{raw-filter}}

\begin{verbatim}
  % method 1
  \begin{dbFilters}{db}
    \dbNewConditional{cond1}{int-attr}{\rval > 1}
    \dbNewConditional*{cond2}{str-attr}{\d+}
  \end{dbFilters}
  \dbNewStyle{style}{db}{raw-filter={cond1 && cond2}}
  % method 2
  \begin{dbFilters}{db}
    \dbNewConditional{cond1}{int-attr}{\rval > 1}
    \dbNewConditional*{cond2}{str-attr}{\d+}
    \dbCombineFilters{filter}{cond1 && cond2}
  \end{dbFilters}
  \dbNewStyle{style}{db}{filter=filter}
\end{verbatim}

\changes{1.2}{2022-01-08}{Fix bug}{string sorting bug}
\begin{option}[added=2022-01-05]{sort}
  \begin{syntax}
    sort = \{ <attr spec1>, <attr spec2>, \ldots{} \}
  \end{syntax}

  为当前样式设置排序规则。支持根据 |str|,|date|,|int|,|fp| 类型的数据进行排
  序,支持多级排序。\meta{attr} 表示增序,\meta{attr}* 表示降序。下面例子中,
  使用 |sort-style| 展示数据时的顺序为先按 |level| 降序,|level| 相同的再按出
  生日期 |birth| 增序,以此类推。
\end{option}

\begin{verbatim}
  \dbNewDatabase{sort-example}{
    name = str,
    birth = date,
    level = int,
    weight = fp,
  }
  \dbNewStyle{sort-style}{sort-example}{
    sort = { level*, birth, name, weight }
  }
\end{verbatim}

\begin{option}[added=2022-01-05, rEXP]{item-code}
  \begin{syntax}
    item-code = <code>
  \end{syntax}

  该选项用来设置展示数据库中每条记录的代码。你可以使用 \cs{dbuse} 来展示属性的
  值。
\end{option}

\changes{1.3}{2022-01-09}{Update option}{\opt{<attr>/sep}}
\begin{option}[added=2022-01-05, updated=2022-01-08, rEXP]{<attr>/sep}
  \begin{syntax}
    <attr>/sep = <separator> \\
    <attr>/sep = \{ \\
    ~~\meta{separator between two}, \\
    ~~\meta{separator between more than two}, \\
    ~~\meta{separator between final two} \\
    \} \\
    <attr>/sep = \{ \\
    ~~\meta{separator before year}, \\
    ~~\meta{separator between year and month}, \\
    ~~\meta{separator between month and day}, \\
    ~~\meta{separator after day} \\
    \} \\
  \end{syntax}

  该选项只适用于类型为 |clist| 或 |date| 的属性,用来设置列表间元素的间隔。参
  数为一个 \meta{separator} 时,所有元素间的分隔符被设置为 \meta{separator}。
  \meta{separator before year} 和 \meta{separator after day} 被设置为空。
\end{option}

  参数为3个元素的逗号分隔的列表时,此选项用来设置列表元素的分隔符,分别用来设
  置只有两个元素时的分隔符 \meta{separator between two},超过两个元素时的分隔
  符 \meta{separator between more than two},和最后两个元素之间的分隔符
  \meta{separator between final two}。对于类型为 |clist| 的属性,设置此选项时
  如果参数列表数量不是1或者3会触发报错。

\begin{verbatim}
  % clist-attr is an attribute of database db
  % suppose the val of clist-attr is { 1, 2, 3 }
  \dbNewStyle{clist-sep}{db}{
    clist-attr/sep = { ,~ },                % print 1, 2, 3
    clist-attr/sep = { {,~}, {,~}, {and~} } % print 1, 2 and 3
  }
\end{verbatim}

  参数为4个元素的逗号分隔的列表时,此选项用来设置日期的分隔符,分别用来设
  置 \meta{year} 之前的分隔符 \meta{separator before year} ,\meta{year} 和
  \meta{month} 之间的分隔符 \meta{separator between year and month} ,
  \meta{month} 和 \meta{day} 之间的分隔符,以及 \meta{day} 之后的分隔符。对于
  类型为 |date| 的属性,设置此选项时如果参数列表数量不是1或者4会触发报错。

\begin{verbatim}
  % date-attr is an attribute of database db
  % suppose the val of date-attr is 2022/01/01
  \dbNewStyle{date-sep}{db}{
    date-attr/sep = -,             % print 2022-01-01
    date-attr/sep = { |, -, -, | } % print |2022-01-01|
  }
\end{verbatim}

\changes{1.3}{2022-01-08}{Add option}{\opt{<attr>/zfill}}
\begin{option}[added=2022-01-08, EXP]{<attr>/zfill}
  \begin{syntax}
    <attr>/zfill = <\TTF>
  \end{syntax}

  该选项只适用于类型为 |date| 的属性。控制输出月份和天时是否补零。
\end{option}

\subsubsection{装饰器}

下面这些选项在不同层次上装饰原有的展示代码,有些其实不必通过选项的形式来装饰,
但这样做的好处是可以进一步使样式与内容分离。下面的例子中,\meta{style1} 和
\meta{style2} 是相同的样式,都用 * 将 \meta{attr1} 包裹住了,但是如果你还想定
义一个样式用 = 将 \meta{attr1} 包裹住,如果用 \meta{style1} 的方式,那就可能
需要重复大片代码,用 \meta{style2} 的方式则可以很轻松的继承 \meta{style1} 中的
代码。

\begin{verbatim}
  \dbNewStyle{style1}{db}{
    item-code = {%
      *\rvuse{attr1}*\rvuse{attr2}
      % more code
    }
  }
  \dbNewStyle{base-style}{db}{
    item-code = {%
      \rvuse{attr1}\rvuse{attr2}
      % more code
    }
  }
  \dbNewStyle[base-style]{style2}{db}{
    attr1/before-code = { * },
    attr1/after-code = { * },
  }
  \dbNewStyle[base-style]{style3}{db}{
    attr1/before-code = { = },
    attr1/after-code = { = },
  }
\end{verbatim}

\changes{1.3}{2022-01-08}{Add option}{\opt{<attr>/wrapper}}
\begin{option}[added=2022-01-08, rEXP]{<attr>/wrapper}
  \begin{syntax}
    <attr>/wrapper = <control sequence>
  \end{syntax}

  该选项只适用于类型为 |date| 的属性。\meta{control sequence} 只接收一个参数即
  日期,如果设置了此选项,则最后输出的日期为
  \meta{control sequence}\marg{date}。
\end{option}

\begin{option}[added=2022-01-05, rEXP]{before-code}
  \begin{syntax}
    before-code = <code>
  \end{syntax}

  该选项用来设置在展示整个数据库之\zhbefore 需要执行的代码。
\end{option}

\begin{option}[added=2022-01-05, rEXP]{after-code}
  \begin{syntax}
    after-code = <code>
  \end{syntax}

  该选项用来设置在展示整个数据库之\zhafter 需要执行的代码。
\end{option}

\changes{1.2}{2022-01-08}{Add options}{\opt{record-before-code},
\opt{record-after-code}}
\begin{option}[added=2022-01-05, rEXP]{record-before-code}
  \begin{syntax}
    record-before-code = <code>
  \end{syntax}

  该选项用来设置在展示当前记录之\zhbefore 需要执行的代码。
\end{option}

\begin{option}[added=2022-01-05, rEXP]{record-after-code}
  \begin{syntax}
    record-after-code = <code>
  \end{syntax}

  该选项用来设置在展示当前记录之\zhafter 需要执行的代码。
\end{option}

\begin{option}[added=2022-01-05, rEXP]{<attr>/before-code}
  \begin{syntax}
    <attr>/before-code = <code>
  \end{syntax}

  该选项用来设置展示数据库中属性 \meta{attr} 对应数据之\zhbefore 需要执行的代
  码。\cs{dbuse} 会在展示属性数据\zhbefore 执行此代码。
\end{option}

\begin{option}[added=2022-01-05, rEXP]{<attr>/after-code}
  \begin{syntax}
    <attr>/after-code = <code>
  \end{syntax}

  该选项用来设置展示数据库中属性 \meta{attr} 对应数据之\zhafter 需要执行的代码。
  \cs{dbuse} 会在展示属性数据\zhafter 执行此代码。
\end{option}

\begin{option}[added=2022-01-05, rEXP]{<attr>/item-before-code}
  \begin{syntax}
    <attr>/item-before-code = <code>
  \end{syntax}

  该选项只适用于类型为 |clist| 的属性,用来设置展示列表每个元素\zhbefore 需要
  执行的代码。
\end{option}

\begin{option}[added=2022-01-05, rEXP]{<attr>/item-after-code}
  \begin{syntax}
    <attr>/item-after-code = <code>
  \end{syntax}

  该选项只适用于类型为 |clist| 的属性,用来设置展示列表每个元素\zhafter 需要执
  行的代码。
\end{option}

\subsection{使用 \cs{dbNewReviewPoints} 定义复习点}

\begin{function}[added=2022-01-05]{\dbNewReviewPoints}
  \begin{syntax}
    \cs{dbNewReviewPoints} \marg{name} \marg{points}
  \end{syntax}

  定义名为 \meta{name} 的复习点。这是专门为错题本或复习所定制的功能,
  \meta{points}是一系列整数,现在假设每道错题你都将写错时的日期记录在了 |date|
  属性中,并且你希望每隔2,5,15天复习一次。下面的代码给出了一个实现示例。
\end{function}

\begin{verbatim}
  \dbNewReviewPoints{review-point}{2, 5, 15}            % 定义复习点
  \begin{dbFilters}
    \dbNewConditional{cond1}{date}{review-point|\Today} % 定义复习条件
    \dbCombineConditionals{filter1}{cond1}              % 定义过滤器
  \end{dbFilters}
  \dbNewStyle{review-style}{ques}{filter=filter1}       % 定义展示样式
\end{verbatim}

\subsection{在 \env{dbFilters} 环境中定义过滤器}

\begin{environment}[added=2022-01-05]{dbFilters}
  \begin{syntax}
    |\begin{dbFilters}|\marg{database} \\
    ~~\meta{code}
    |\end{dbFilters}| \\
  \end{syntax}

  \env{dbFilters}用来定义过滤器,此环境中定义了 \cs{dbNewConditional} 命令用来
  定义条件和 \cs{dbCombineConditionals} 命令用来组合条件定义过滤器。过滤器独立
  于每个 \meta{database},这意味着你可以在不同数据库中定义名称相同的过滤条件和
  过滤器。
\end{environment}

\begin{function}[added=2022-01-05, updated=2022-01-08]{\dbNewConditional, \dbNewConditional*}
  \begin{syntax}
    \cs{dbNewConditional}  \marg{name}        \marg{attr} \marg{cond spec} \\
    \cs{dbNewConditional}* \marg{name}        \marg{attr} \marg{cond spec} \\[2pt]
    \cs{dbNewConditional}  \marg{name} \marg{int/fp attr} \marg{expr} \\
    \cs{dbNewConditional}* \marg{name} \marg{int/fp attr} \marg{expr} \\
    \cs{dbNewConditional}  \marg{name} \marg{str/tl attr} \marg{regex expr} \\
    \cs{dbNewConditional}* \marg{name} \marg{str/tl attr} \marg{regex expr} \\
    \cs{dbNewConditional}  \marg{name}  \marg{clist attr} \marg{val list} \\
    \cs{dbNewConditional}* \marg{name}  \marg{clist attr} \marg{val list} \\
    \cs{dbNewConditional}  \marg{name}   \marg{date attr} \marg{expr}
    \cs{dbNewConditional}* \marg{name}   \marg{date attr} \{\meta{review points}\orbar\meta{date}\} \\
  \end{syntax}

  \cs{dbNewConditional} 用来定义名为 \meta{name} 的条件,\meta{attr} 指定条件
  所绑定的属性,在 \meta{cond spec} 中可以用 \cs{dbval} 指代属性的值。
\end{function}

  \changes{1.3}{2022-01-10}{Update doc}{truncated division}
  对于类型为 |int| 和 |fp| 的属性,\meta{expr} 传递给 \cs{int_compare:nTF} 或
  \cs{fp_compare:nTF} 处理。
  \begin{note}
    |/| 为四舍五入除法,截断除法请用 \cs{dbIntDivTruncate}。
  \end{note}

  对于类型为 |str| 和 |tl| 的属性,\meta{regex} 为正则表达式,
  \cs{dbNewConditional} 表示部分匹配,\cs{dbNewConditional*} 表示整体匹配。

\begin{verbatim}
  \dbNewConditional {cond1}{str-attr}{abc} % 匹配 abc, abcd, 1abc, =abc= 等
  \dbNewConditional*{cond2}{str-attr}{abc} % 只匹配 abc
\end{verbatim}

  对于类型为 |clist| 的属性,使用 \cs{dbNewConditional} 定义的条件只要
  \meta{val list} 中的任意一个元素在属性值(列表)中则条件成立;使用
  \cs{dbNewConditional*} 定义的条件只有 \meta{val list} 中每一个值都在属性值
  (列表)中条件才成立。

\begin{verbatim}
  \dbNewConditional {cond1}{clist-attr}{a, b, c} % a, b, d 满足条件
  \dbNewConditional*{cond2}{clist-attr}{a, b, c} % a, b, d 不满足条件
\end{verbatim}

  \changes{1.3}{2022-01-08}{Update logic}{swap definition of starred and
  unstarred conditionals of date}
  对于类型为 |date| 的属性,\cs{dbNewConditional} 定义的条件后续处理中会将
  \meta{expr} 中的所有日期转换成相对\textit{1971年1月1日}的一个整数值,然后将
  处理后的表达式传递给 \cs{int_compare:nTF} 做进一步处理;
  \cs{dbNewConditional*} 使用复习点来定义过滤条件,\meta{review points} 是
  \cs{dbNewReviewPoints} 定义的复习点,\meta{date} 是用来比较的日期。

\begin{function}[added=2022-01-05]{\dbCombineConditionals}
  \begin{syntax}
    \cs{dbCombineConditionals} \marg{name} \marg{cond combination} \oarg{info}
  \end{syntax}

  \cs{dbCombineConditionals} 定义名为 \marg{name} 的过滤器,并将
  \cs{dbNewConditional} 定义的条件组合起来,比如
  \verb=\dbCombineConditionals{filter}{(cond1 && cond2) || !cond3}=。
  \meta{cond combination} 中可以使用的关系操作符为 \verb=&&, ||, !=。
  可以将 \opt{filter} 选项设置为 \meta{name} 来应用过滤器。\meta{info} 为过滤
  器的相关信息,在展示数据库的时候可以用 \cs{dbFilterInfo} 指代。
\end{function}

\subsection{使用 \env{dbitem} 环境存储数据}

\begin{environment}[added=2022-01-05]{dbitem}
  \begin{syntax}
    |\begin{dbitem}|\marg{database}[ \\
    ~~\meta{attr1} = \meta{val1}, \\
    ~~\meta{attr2} = \meta{val2}, \\
    ~~\ldots{} \\
    ] \\
    ~~\meta{code} \\
    |\end{dbitem}|
  \end{syntax}

  \env{dbitem} 环境用来存储数据。有两种存储数据的方法,较短的数据可以在选项列
  表中通过键值对设置值,较长的数据可以在 \meta{code} 中使用 \cs{dbsave} 存储。
  \cs{dbsave}会覆盖选项中设置的值。没有设置的值将会被设置为全局默认值,下面给
  出一个存储示例。
\end{environment}

\begin{verbatim}
  \begin{dbitem}[date = 2022-01-01, info = 测试]
    \dbsave{question}{这是一个测试问题}
    \dbsave{answer}  {这是一个测试答案}
  \end{dbitem}
\end{verbatim}

\subsection{\cs{dbsave} 和 \cs{dbuse}}

\changes{1.3}{2022-01-08}{Add macro}{\cs{dbsave*}}
\begin{function}[added=2022-01-05, updated=2022-01-08]{\dbsave, \dbsave*}
  \begin{syntax}
    \cs{dbsave}  \marg{attr} \marg{data} \\
    \cs{dbsave}* \marg{attr} \marg{data}
  \end{syntax}

  \cs{dbsave} 用来存储数据,只能在 \env{item} 环境中使用。使用 \cs{dbsave*} 存
  储的数据会被 \cs{exp_not:n} 包裹。
\end{function}

\changes{1.2}{2022-01-08}{Update macro}{make \cs{dbuse} fully-expandable}
\begin{function}[added=2022-01-05, updated=2022-01-08, EXP]{\dbuse}
  \begin{syntax}
    \cs{dbuse} \marg{attr}
  \end{syntax}

  \cs{dbuse} 用来展示数据,只能在 \opt{item-code} 选项中使用。\cs{dbuse} 是可
  展开的。
\end{function}

\subsection{条件判别式}

\begin{function}[added=2022-01-05, EXP]{\dbIfEmptyT, \dbIfEmptyF, \dbIfEmptyTF}
  \begin{syntax}
    \cs{dbIfEmptyTF} \marg{true code} \marg{false code} \\
    \cs{dbIfEmptyT} \marg{true code} \\
    \cs{dbIfEmptyF} \marg{false code}
  \end{syntax}

  该判别式用来判断当前数据库是否为空。下面的示例展示了如何预防空的列表环境。
\end{function}

\begin{verbatim}
  \dbNewStyle{style-cond1}{database-test}{
    before-code = {\dbIfEmptyF{\begin{enumerate}}},
    after-code = {\dbIfEmptyF{\end{enumerate}}},
    item-code = {\item \dbuse{attr-test}}
  }
\end{verbatim}

\changes{1.2}{2022-01-08}{Remove macros}{\cs{dbItemIfEmpty(TF)}, \cs{dbClistItemIfEmpty(TF)}}

\subsection{表达式函数}

\changes{1.3}{2022-01-10}{Add macros}{expression function aliases}
\begin{function}[added=2022-01-10, EXP]{
  \dbIntAbs, \dbIntSign, \dbIntDivRound, \dbIntDivTruncate, \dbIntMax,
  \dbIntMin, \dbIntMod, \dbFpSign,
}
  \begin{syntax}
    \cs{dbIntAbs} \Arg{intexpr}
    \cs{dbIntSign} \Arg{intexpr}
    \cs{dbIntDivRound} \Arg{intexpr_1} \marg{intexpr_2}
    \cs{dbIntDivTruncate} \Arg{intexpr_1} \marg{intexpr_2}
    \cs{dbIntMax} \Arg{intexpr_1} \marg{intexpr_2}
    \cs{dbIntMin} \Arg{intexpr_1} \marg{intexpr_2}
    \cs{dbIntMod} \Arg{intexpr_1} \marg{intexpr_2}
    \cs{dbFpSign} \Arg{fpexpr}
  \end{syntax}
  \cs{dbIntAbs} 等同于 \cs{int_abs:n} \\
  \cs{dbIntSign} 等同于 \cs{int_sign:n} \\
  \cs{dbIntDivRound} 等同于 \cs{int_div_round:nn} \\
  \cs{dbIntDivTruncate} 等同于 \cs{int_div_truncate:nn} \\
  \cs{dbIntMax} 等同于 \cs{int_max:nn} \\
  \cs{dbIntMin} 等同于 \cs{int_min:nn} \\
  \cs{dbIntMod} 等同于 \cs{int_mod:nn} \\
  \cs{dbFpSign} 等同于 \cs{fp_sign:n} \\
   详细的文档见 \pkg{interface3} \\
\end{function}

\subsection{特殊命令}

\pkg{dbshow} 定义了一些特殊的命令,会根据语境展开为不同的内容。

\changes{1.1}{2022-01-05}{Add macro}{
  \cs{dbarabic}, \cs{dbalph}, \cs{dbAlph}, \cs{dbroman},
  \cs{dbRoman}
}
\changes{1.1}{2022-01-06}{Fix bug}{\cs{dbIndex} not defined}
\begin{function}[added=2022-01-05, EXP]{
  \dbval, \dbDatabase, \dbFilterName, \dbFilterInfo,
  \dbIndex, \dbarabic, \dbalph, \dbAlph, \dbroman, \dbRoman
}
\begin{tblr}{ll}
  \cs{dbval} & 当前属性的值 \\
  \cs{dbDatabase} & 数据库名称 \\
  \cs{dbFilterName} & 当前样式过滤器的名称 \\
  \cs{dbFilterInfo} & 当前样式过滤器的相关信息 \\
  \cs{dbIndex} & 数据索引,等同于 \cs{dbuse}|{id}| \\
  \cs{dbarabic} & 用数字表示的查询集数据计数 \\
  \cs{dbalph} & 用小写字母表示的查询集数据计数 \\
  \cs{dbAlph} & 用大写字母表示的查询集数据计数 \\
  \cs{dbroman} & 用小写罗马字母表示的查询集数据计数 \\
  \cs{dbroman} & 用大写罗马字母表示的查询集数据计数 \\
  \end{tblr}
\end{function}

\section{错题本示例}
见第 \ref{sec:example} 节。

\changes{1.1}{2022-01-07}{Update doc}{improve example}

\title{
  Package \pkg{dbshow} \version%
  \protect\footnote{%
    Repository: \url{https://github.com/ZhiyuanLck/dbshow},
    Telegram Group: \url{https://t.me/latex_dbshow}}
  \rlap{\makebox[4cm][r]{
    \normalsize $\Longrightarrow$ \color{red}
    \linktarget{zh}{en}{中文版本}
  }}
}
\author{Li Changkai \texttt{<lichangkai225@qq.com>}}
\date{\endate}
\maketitle

\section{Introduction}

The initial motivation to write this package is that I want to write a
template, which can collect questions you gave the wrong answer and can
display those questions you would like to review by some conditionals, such as
questions with certain label, questions you have answered uncorrectly for
certain times or questions having not been reviewed for certain days. So this
package provides a database to do such thing.

The package provides four core functions: data storage and display, data
filtering, data sorting and data display. All data is saved once and then you
can display these data with custom filters, orders and styles.

\changes{1.4}{2022-01-10}{Add check}{version of \pkg{l3kernel}}
\pkg{dbshow} depends on \pkg{l3kernel} with version date after |2022-11-07|.

\changes{1.2}{2022-01-09}{Add doc}{descripton for expansion}
\begin{itemize}
  \item Macros with a $\star$ are fully-expandable;
  \item Macros with a \ding{73} are restricted-expandable;
  \item Macros without appending a special symbol are nonexpandable;
  \item Options with a $\star$ \textit{do not} affect the expandability of
    related macros;
  \item Options with a \ding{73} affect the expandability of related macros
    according to its value;
  \item Options without appending a special symbol make the related macros
    nonexpandable.
\end{itemize}

\subsection{Data Types}

The package constructs 6 types based on the internal typed of \pkg{expl3}:
\begin{Description}[\texttt{clist}]
  \item[\texttt{date}]
    date saved in |yyyy/mm/dd| format, supports comparison, sorting
    (converting to string), default \cs{dbtoday}.
  \item[\texttt{str}]
    string,supports regex match and sorting, default empty.
  \item[\texttt{tl}]
    token list, supports regex match, default empty.
  \item[\texttt{int}]
    integer, supports comparison and sorting, default 0.
  \item[\texttt{fp}]
    floating point, supports comparison and sorting, default 0.
  \item[\texttt{clist}]
    comma list, default empty.
\end{Description}

\changes{1.3}{2022-01-08}{Remove dependency}{\pkg{datatime2}}
All types are internal types of \pkg{expl3} except |date| type, which provides
by \pkg{dbshow} itself and supports converting to integer and printing with
style.

\subsection{Comparison to \pkg{datatool}}

\changes{1.2}{2022-01-07}{Add doc}{add comparison to \pkg{datatool}}

\pkg{dbshow} and \pkg{datatool} implement the same core functions. But
\pkg{dbshow} is based on \pkg{expl3} and it supports string regex and
multi-level sorting. \pkg{dbshow} tries to divide style from the contents
(data in database): all styles are predefined and can be reused conveniently
so that there can be only codes to save data and one-line code to show the
database inside the \env{document} environment. You can hide the details in
the preamble and focus on the data you want to display. \pkg{dbshow} provides
a simple temporary runtime database, which means it can not input and output
data from/to extern files (they should be responsible by some professional
programming languages). When you need to delete or revise a record, just go to
where it is recorded in the source code rather than use a macro to manipulate
data after they are saved. In a sense, \TeX~file is also a kind of data
persistence.

\section{Interfaces}

\subsection{Create, Display and Clear Database}

\begin{function}[added=2022-01-05]{\dbNewDatabase, \dbNewDatabase*}
  \begin{syntax}
    \cs{dbNewDatabase} \oarg{base database} \marg{database} \{ \\
    ~~\meta{attr1} = \meta{type spec1}, \\
    ~~\meta{attr2} = \meta{type spec2}, \\
    ~~\ldots{} \\
    \} \\
    \cs{dbNewDatabase}* \marg{database} \{ \\
    ~~\meta{attr1} = \meta{type spec1}, \\
    ~~\meta{attr2} = \meta{type spec2}, \\
    ~~\ldots{} \\
    \} \\
  \end{syntax}

\end{function}

Create a new database named \meta{database}, unstarred form provides the optional
\meta{base database} from which current database inherit the attributes settings.
The unstarred form always replace the old definition, while starred form
appends the new options.

\begin{syntax}
  \meta{attr} = \meta{type} \\
  \meta{attr} = \meta{type}\orbar\meta{default}
\end{syntax}

The first form defines the \meta{attr} as \meta{type}, and the second also
sets the default value.

\begin{note}
  Every database has a default attribute |id| to store the index of the item.
\end{note}

The example below define a database named |ques|.
\begin{verbatim}
  \dbNewDatabase{ques}{
    question = tl, % store question
    answer = tl,   % store corresponding answer
    date = date,   % store the date when you were wrong
    info = tl,     % store extra info
    labels = clist % store question labels
  }
\end{verbatim}

\begin{function}[added=2022-01-05]{\dbshow}
  \begin{syntax}
    \cs{dbshow} \marg{style} \marg{database}
  \end{syntax}

  Show the \meta{database} with \meta{style}.
\end{function}

\changes{1.2}{2022-01-07}{Add macro}{\cs{dbclear}}
\begin{function}[added=2022-01-07]{\dbclear}
  \begin{syntax}
    \cs{dbclear} \marg{database}
  \end{syntax}

  Clear the content of \meta{database}.
\end{function}

\subsection{\cs{dbNewStyle} and Style Options}

\begin{function}[added=2022-01-05]{\dbNewStyle}
  \begin{syntax}
    \cs{dbNewStyle} \oarg{base styles} \marg{style} \marg{database} \marg{opts}
  \end{syntax}

  Define a new \meta{style} that binds to \meta{database}. The style can
  inherit from a list of \meta{base styles} such as
  |\dbNewStyle[base1, base2]{new-style}{ques}{}|.
\end{function}

\subsubsection{General Options}

\begin{option}[added=2022-01-05]{filter}
  \begin{syntax}
    filter = <filter>
  \end{syntax}

  Set the \meta{filter} defined by \cs{dbCombineFilters}.
\end{option}

\changes{1.1}{2022-01-06}{Add option}{\opt{raw-filter}}
\begin{option}[added=2022-01-06]{raw-filter}
  \begin{syntax}
    raw-filter = <conditional expression>
  \end{syntax}

  Set anonymous with conditionals defined by \cs{dbNewConditional}. Two
  filters shows in the code below have the same meaning.
\end{option}

\begin{verbatim}
  % method 1
  \begin{dbFilters}{db}
    \dbNewConditional{cond1}{int-attr}{\rval > 1}
    \dbNewConditional*{cond2}{str-attr}{\d+}
  \end{dbFilters}
  \dbNewStyle{style}{db}{raw-filter={cond1 && cond2}}
  % method 2
  \begin{dbFilters}{db}
    \dbNewConditional{cond1}{int-attr}{\rval > 1}
    \dbNewConditional*{cond2}{str-attr}{\d+}
    \dbCombineFilters{filter}{cond1 && cond2}
  \end{dbFilters}
  \dbNewStyle{style}{db}{filter=filter}
\end{verbatim}

\changes{1.2}{2022-01-08}{Fix bug}{string sorting bug}
\begin{option}[added=2022-01-05]{sort}
  \begin{syntax}
    sort = \{ <attr spec1>, <attr spec2>, \ldots{} \}
  \end{syntax}

  Set sorting rules. Attributes of type |str, date, int, fp| is supported to
  sort.  Multi-level sort is allowed. \meta{attr} represents for ascending
  order, and \meta{attr}* represents for descending order. The example below
  use four fields to determine the order of the records. It sorts on |level|
  in descending order first and if two |levels| are same then sorts on |birth|
  in ascending order and so on.
\end{option}

\begin{verbatim}
  \dbNewDatabase{sort-example}{
    name = str,
    birth = date,
    level = int,
    weight = fp,
  }
  \dbNewStyle{sort-style}{sort-example}{
    sort = { level*, birth, name, weight }
  }
\end{verbatim}

\begin{option}[added=2022-01-05, rEXP]{item-code}
  \begin{syntax}
    item-code = <code>
  \end{syntax}

  Set the code that show a single record. You can use \cs{dbuse} to display
  certian attribute.
\end{option}

\changes{1.3}{2022-01-09}{Update option}{\opt{<attr>/sep}}
\begin{option}[added=2022-01-05, updated=2022-01-08, rEXP]{<attr>/sep}
  \begin{syntax}
    <attr>/sep = <separator> \\
    <attr>/sep = \{ \\
    ~~\meta{separator between two}, \\
    ~~\meta{separator between more than two}, \\
    ~~\meta{separator between final two} \\
    \} \\
    <attr>/sep = \{ \\
    ~~\meta{separator before year}, \\
    ~~\meta{separator between year and month}, \\
    ~~\meta{separator between month and day}, \\
    ~~\meta{separator after day} \\
    \} \\
  \end{syntax}

  Only for attributes of type |clist| or |date|. Set the separator between
  items. If the argument is an one-item comma list, all separators are set to
  \meta{separator} but \meta{separator before year} and \meta{separator after
  day} is set empty.
\end{option}

  If the argument is a comma list of 3 items, it is used to set the separator
  between items of the comma list. Following documentation is quoted from
  \pkg{interface3}:
  \begin{quote}
    If the comma list has more than two items, the \meta{separator between
    more than two} is placed between each pair of items except the last, for
    which the \meta{separator between final two} is used. If the comma list
    has exactly two items, then they are placed in the input stream separated
    by the \meta{separator between two}. If the comma list has a single item,
    it is placed in the input stream, and a comma list with no items produces
    no output.
  \end{quote}
  For attributes of type |clist|, incorrect number (numbers exclude 1 and 3)
  of items of the argument will raise an error.

\begin{verbatim}
  % clist-attr is an attribute of database db
  % suppose the val of clist-attr is { 1, 2, 3 }
  \dbNewStyle{clist-sep}{db}{
    clist-attr/sep = { ,~ },                % print 1, 2, 3
    clist-attr/sep = { {,~}, {,~}, {and~} } % print 1, 2 and 3
  }
\end{verbatim}

  If the argument is a comma list of 4 items, it is used to set the separators
  of the date. For attributes of type |date|, incorrect number (numbers
  exclude 1 and 4) will raise an error.

\begin{verbatim}
  % date-attr is an attribute of database db
  % suppose the val of date-attr is 2022/01/01
  \dbNewStyle{date-sep}{db}{
    date-attr/sep = -,             % print 2022-01-01
    date-attr/sep = { |, -, -, | } % print |2022-01-01|
  }
\end{verbatim}

\changes{1.3}{2022-01-08}{Add option}{\opt{<attr>/zfill}}
\begin{option}[added=2022-01-08, EXP]{<attr>/zfill}
  \begin{syntax}
    <attr>/zfill = <\TTF>
  \end{syntax}

  Only for attributes of type |date|. Control whether to fill zero on the left
  of the month or day.
\end{option}

\subsubsection{Decorators}

The options below serves as decorators. In some cases, decorator can also be
encoded directly into |item-code| or some other places, which is convenient
sometimes. The benefit of defining decorators with options is that styles step
further to be divided with contents. In the examples below, \meta{style1} and
\meta{style2} is the same style, which wrap \meta{attr1} with *. When you want
another style which wrap \meta{attr1} with =, if you choose the way of
\meta{style1}, \meta{item code} are repeated, otherwise if you choose the way
of \meta{style2}, \meta{item code} is inherited and you only need define the
decorators.

\begin{verbatim}
  \dbNewStyle{style1}{db}{
    item-code = {%
      *\rvuse{attr1}*\rvuse{attr2}
      % more code
    }
  }
  \dbNewStyle{base-style}{db}{
    item-code = {%
      \rvuse{attr1}\rvuse{attr2}
      % more code
    }
  }
  \dbNewStyle[base-style]{style2}{db}{
    attr1/before-code = { * },
    attr1/after-code = { * },
  }
  \dbNewStyle[base-style]{style3}{db}{
    attr1/before-code = { = },
    attr1/after-code = { = },
  }
\end{verbatim}

\changes{1.3}{2022-01-08}{Add option}{\opt{<attr>/wrapper}}
\begin{option}[added=2022-01-08, rEXP]{<attr>/wrapper}
  \begin{syntax}
    <attr>/wrapper = <control sequence>
  \end{syntax}

  Only for attributes of type |date|. Output of \cs{dbuse}\marg{date attr}
  will be \meta{control sequence}\marg{date}.
\end{option}

\begin{option}[added=2022-01-05, rEXP]{before-code}
  \begin{syntax}
    before-code = <code>
  \end{syntax}

  Set the \meta{code} that is executed \enbefore displaying the database.
\end{option}

\begin{option}[added=2022-01-05, rEXP]{after-code}
  \begin{syntax}
    after-code = <code>
  \end{syntax}

  Set the \meta{code} that is executed \enafter displaying the database.
\end{option}

\changes{1.2}{2022-01-08}{Add options}{\opt{record-before-code},
\opt{record-after-code}}
\begin{option}[added=2022-01-05, rEXP]{record-before-code}
  \begin{syntax}
    record-before-code = <code>
  \end{syntax}

  Set the \meta{code} that is executed \enbefore displaying a record.
\end{option}

\begin{option}[added=2022-01-05, rEXP]{record-after-code}
  \begin{syntax}
    record-after-code = <code>
  \end{syntax}

  Set the \meta{code} that is executed \enafter displaying the record.
\end{option}

\begin{option}[added=2022-01-05, rEXP]{<attr>/before-code}
  \begin{syntax}
    <attr>/before-code = <code>
  \end{syntax}

  Set the \meta{code} that is executed by \cs{dbuse} \enbefore displaying
  certain attribute.
\end{option}

\begin{option}[added=2022-01-05, rEXP]{<attr>/after-code}
  \begin{syntax}
    <attr>/after-code = <code>
  \end{syntax}

  Set the \meta{code} that is executed by \cs{dbuse} \enafter displaying
  certain attribute.
\end{option}

\begin{option}[added=2022-01-05, rEXP]{<attr>/item-before-code}
  \begin{syntax}
    <attr>/item-before-code = <code>
  \end{syntax}

  Only for attributes of type |clist|. Set the \meta{code} that is excuted
  \enbefore displaying the item of the comma list.
\end{option}

\begin{option}[added=2022-01-05, rEXP]{<attr>/item-after-code}
  \begin{syntax}
    <attr>/item-after-code = <code>
  \end{syntax}

  Only for attributes of type |clist|. Set the \meta{code} that is excuted
  \enafter displaying the item of the comma list.
\end{option}

\subsection{Use \cs{dbNewReviewPoints} to Define Review Points}

\begin{function}[added=2022-01-05]{\dbNewReviewPoints}
  \begin{syntax}
    \cs{dbNewReviewPoints} \marg{name} \marg{points}
  \end{syntax}

  Define the new \meta{points} that is specially designed for reviewing
  something. \meta{points} is a list of integers. Suppose you record the date
  when you did not answer correctly and you plan to review every 2, 5 and 15
  days. The following code give what you want.
\end{function}

\begin{verbatim}
  \dbNewReviewPoints{review-point}{2, 5, 15}            % define points
  \begin{dbFilters}
    \dbNewConditional{cond1}{date}{review-point|\Today} % define conditional
    \dbCombineConditionals{filter1}{cond1}              % define filter
  \end{dbFilters}
  \dbNewStyle{review-style}{ques}{filter=filter1}       % define style
\end{verbatim}

\subsection{Define Filters inside \env{dbFilters} Environment}

\begin{environment}[added=2022-01-05]{dbFilters}
  \begin{syntax}
    |\begin{dbFilters}|\marg{database} \\
    ~~\meta{code}
    |\end{dbFilters}| \\
  \end{syntax}

  Filters are defined inside \env{dbFilters} environment, inside which,
  \cs{dbNewConditional} is defined to declare conditionals and
  \cs{dbCombineConditionals} is defined to combine conditionals. Filters are
  independent in different databases, which means the same name of filters is
  allowed in different databases.
\end{environment}

\begin{function}[added=2022-01-05, updated=2022-01-08]{\dbNewConditional, \dbNewConditional*}
  \begin{syntax}
    \cs{dbNewConditional}  \marg{name}        \marg{attr} \marg{cond spec} \\
    \cs{dbNewConditional}* \marg{name}        \marg{attr} \marg{cond spec} \\[2pt]
    \cs{dbNewConditional}  \marg{name} \marg{int/fp attr} \marg{expr} \\
    \cs{dbNewConditional}* \marg{name} \marg{int/fp attr} \marg{expr} \\
    \cs{dbNewConditional}  \marg{name} \marg{str/tl attr} \marg{regex expr} \\
    \cs{dbNewConditional}* \marg{name} \marg{str/tl attr} \marg{regex expr} \\
    \cs{dbNewConditional}  \marg{name}  \marg{clist attr} \marg{val list} \\
    \cs{dbNewConditional}* \marg{name}  \marg{clist attr} \marg{val list} \\
    \cs{dbNewConditional}  \marg{name}   \marg{date attr} \marg{expr}
    \cs{dbNewConditional}* \marg{name}   \marg{date attr} \{\meta{review points}\orbar\meta{date}\} \\
  \end{syntax}

  Define the conditional named \meta{name} that binds to \meta{attr}. \cs{dbval}
  is replaced with the real value of the attribute inside the \meta{cond spec}.
\end{function}

  \changes{1.3}{2022-01-10}{Update doc}{truncated division}
  For attributes of type |int| and |fp|, \meta{expr} is passed to
  \cs{int_compare:nTF} or \cs{fp_compare:nTF}.
  \begin{note}
    Division using |/| rounds to the closest integer. Use \cs{dbIntDivTruncate} to rounds
    the result toward 0.
  \end{note}

  For attribute of type |str| and |tl|, unstarred form matches any part while
  starred form matches the whole part with the \meta{regex expr}.

\begin{verbatim}
  \dbNewConditional {cond1}{str-attr}{abc}  % match abc, abcd, 1abc, =abc=, etc
  \dbNewConditional*{cond2}{str-attr}{abc} % only match abc
\end{verbatim}

  For attributes of type |clist|, the conditional defined by unstarred form is
  true if any item of \meta{val list} is in the comma list. While the
  conditional defined by starred form is true only if every item of \meta{val
  list} is in the comma list. As is showed below, for |cond1|, |a| is in
  |{a, b, d}| so |cond1| is true. While |c| is not in |{a, b, d}| so |cond2|
  is false.

\begin{verbatim}
  \dbNewConditional {cond1}{clist-attr}{a, b, c}  % {a, b, d} -> true
  \dbNewConditional*{cond2}{clist-attr}{a, b, c} % {a, b, d} -> false
\end{verbatim}

  \changes{1.3}{2022-01-08}{Update logic}{swap definition of starred and
  unstarred conditionals of date}
  For attributes of type |date|, unstarred form replace each date with a
  integer representing for the days between \meta{date} and
  \textit{1971/01/01}, and the result is passed to \cs{int_compare:nTF}.
  Starred form defines the conditional with review points defined by
  \cs{dbNewRdbNewReviewPoints} and \meta{date} is the date to be compared.

\begin{function}[added=2022-01-05]{\dbCombineConditionals}
  \begin{syntax}
    \cs{dbCombineConditionals} \marg{name} \marg{cond combination} \oarg{info}
  \end{syntax}

  Define the filter \meta{name}, which combine the conditionals and store the
  extra \meta{info} into \cs{dbFilterInfo}. So you can write something as\\
  \verb=\dbCombineConditionals{filter}{(cond1 && cond2) || !cond3}=.\\
  Supported operators are \verb=&&, ||, !=. You can set the option \opt{filter}
  to \meta{name} to apply the filter when you display the database.
\end{function}

\subsection{Store Data with \env{dbitem} Environment}

\begin{environment}[added=2022-01-05]{dbitem}
  \begin{syntax}
    |\begin{dbitem}|\marg{database}[ \\
    ~~\meta{attr1} = \meta{val1}, \\
    ~~\meta{attr2} = \meta{val2}, \\
    ~~\ldots{} \\
    ] \\
    ~~\meta{code} \\
    |\end{dbitem}|
  \end{syntax}

  The data are stored with \env{dbitem} environment in two ways. Smaller data
  can be stored in the option list and the bigger data can be stored by
  \cs{dbsave}, which will suppress the value set by the option list. An
  example code is showned below.
\end{environment}

\begin{verbatim}
  \begin{dbitem}[date=2022-01-01, info=test]
    \dbsave{question}{This is a test question.}
    \dbsave{answer}  {This is a test answer.}
  \end{dbitem}
\end{verbatim}

\subsection{\cs{dbsave} and \cs{dbuse}}

\changes{1.3}{2022-01-08}{Add macro}{\cs{dbsave*}}
\begin{function}[added=2022-01-05, updated=2022-01-08]{\dbsave, \dbsave*}
  \begin{syntax}
    \cs{dbsave}  \marg{attr} \marg{data} \\
    \cs{dbsave}* \marg{attr} \marg{data}
  \end{syntax}

  \cs{dbsave} save the \meta{data} to \meta{attr} of current record.
  \cs{dbsave} can be used only inside the \env{dbitem} environment.
  \meta{data} stored by \cs{dbsave*} is wrapped with \cs{exp_not:n} while
  \meta{data} stored by \cs{dbsave} keeps the same.
\end{function}

\changes{1.2}{2022-01-08}{Update macro}{make \cs{dbuse} fully-expandable}
\begin{function}[added=2022-01-05, updated=2022-01-08, EXP]{\dbuse}
  \begin{syntax}
    \cs{dbuse} \marg{attr}
  \end{syntax}

  Display the value of \meta{attr} of current record. \cs{dbuse} is
  \textbf{expandable} and can be only used inside the option \opt{item-code}.
\end{function}

\subsection{Conditionals}

\begin{function}[added=2022-01-05, EXP]{\dbIfEmptyT, \dbIfEmptyF, \dbIfEmptyTF}
  \begin{syntax}
    \cs{dbIfEmptyTF} \marg{true code} \marg{false code} \\
    \cs{dbIfEmptyT} \marg{true code} \\
    \cs{dbIfEmptyF} \marg{false code}
  \end{syntax}

  Test if the database is empty. The example below shows how to avoid an empty
  list environment.
\end{function}

\begin{verbatim}
  \dbNewStyle{style-cond1}{database-test}{
    before-code = {\dbIfEmptyF{\begin{enumerate}}},
    after-code = {\dbIfEmptyF{\end{enumerate}}},
    item-code = {\item \dbuse{attr-test}}
  }
\end{verbatim}

\changes{1.2}{2022-01-08}{Remove macros}{\cs{dbItemIfEmpty(TF)}, \cs{dbClistItemIfEmpty(TF)}}

\subsection{Expression Functions}

\changes{1.3}{2022-01-10}{Add macros}{expression function aliases}
\begin{function}[added=2022-01-10, EXP]{
  \dbIntAbs, \dbIntSign, \dbIntDivRound, \dbIntDivTruncate, \dbIntMax,
  \dbIntMin, \dbIntMod, \dbFpSign,
}
  \begin{syntax}
    \cs{dbIntAbs} \Arg{intexpr}
    \cs{dbIntSign} \Arg{intexpr}
    \cs{dbIntDivRound} \Arg{intexpr_1} \marg{intexpr_2}
    \cs{dbIntDivTruncate} \Arg{intexpr_1} \marg{intexpr_2}
    \cs{dbIntMax} \Arg{intexpr_1} \marg{intexpr_2}
    \cs{dbIntMin} \Arg{intexpr_1} \marg{intexpr_2}
    \cs{dbIntMod} \Arg{intexpr_1} \marg{intexpr_2}
    \cs{dbFpSign} \Arg{fpexpr}
  \end{syntax}
  \cs{dbIntAbs} is identical to \cs{int_abs:n} \\
  \cs{dbIntSign} is identical to \cs{int_sign:n} \\
  \cs{dbIntDivRound} is identical to \cs{int_div_round:nn} \\
  \cs{dbIntDivTruncate} is identical to \cs{int_div_truncate:nn} \\
  \cs{dbIntMax} is identical to \cs{int_max:nn} \\
  \cs{dbIntMin} is identical to \cs{int_min:nn} \\
  \cs{dbIntMod} is identical to \cs{int_mod:nn} \\
  \cs{dbFpSign} is identical to \cs{fp_sign:n} \\
   Detailed documentation see \pkg{interface3} \\
\end{function}

\subsection{Special Macros}

Some special macros are defined to expand to different contents according to context.

\changes{1.1}{2022-01-05}{Add macro}{
  \cs{dbarabic}, \cs{dbalph}, \cs{dbAlph}, \cs{dbroman},
  \cs{dbRoman}
}
\changes{1.1}{2022-01-06}{Fix bug}{\cs{dbIndex} not defined}
\begin{function}[added=2022-01-05, EXP]{
  \dbval, \dbDatabase, \dbFilterName, \dbFilterInfo,
  \dbIndex, \dbarabic, \dbalph, \dbAlph, \dbroman, \dbRoman
}
\begin{tblr}{ll}
  \cs{dbval} & Attribute value, only according in \cs{dbNewConditional}. \\
  \cs{dbDatabase} & Database name. \\
  \cs{dbFilterName} & Filter name. \\
  \cs{dbFilterInfo} & Filter information. \\
  \cs{dbIndex} & Record index, identical to \cs{dbuse}|{id}| \\
  \cs{dbarabic} & Show the counter of query set as digits. \\
  \cs{dbalph} & Show the counter of query set as lowercase letters. \\
  \cs{dbAlph} & Show the counter of query set as uppercase letters. \\
  \cs{dbroman} & Show the counter of query set as lowercase roman numerals. \\
  \cs{dbroman} & Show the counter of query set as uppercase roman numerals. \\
  \end{tblr}
\end{function}

\section{Example of Flaw Sweeper Template}
\label{sec:example}

\changes{1.1}{2022-01-07}{Update doc}{improve example}

\begin{tcblisting}{enhanced jigsaw,lower separated=false,
  leftlower=0pt,rightlower=0pt,
  colframe=red!50!black,colback=yellow!10!white,
  listing options={style=tcblatex,texcsstyle=*\color{red!70!black}},
  listing and comment,
  pdf comment,freeze pdf,
  compilable listing,
  breakable,
  run pdflatex
}
\documentclass{article}
\usepackage{amsmath, physics}
\usepackage{geometry}
\usepackage{dbshow}
\usepackage{tikz}
\usepackage{tcolorbox}
\tcbuselibrary{skins}
\usetikzlibrary{shadings}
\usepackage[hidelinks]{hyperref}

\geometry{
  margin=2cm
}

% #1 link node #2 target node #3 text to show
\NewDocumentCommand \linktarget { m m m } {%
  \hyperlink{#1}{#3}%
  \raisebox{1em}{\hypertarget{#2}{}}%
}

% question box
\tcbset{
  base/.style={
    empty,
    frame engine=path,
    colframe=yellow!10,
    coltitle=red!70,
    fonttitle=\bfseries\sffamily,
    sharp corners,
    left=4pt,
    right=4pt,
    drop fuzzy shadow,
    drop fuzzy shadow,
    borderline west={3pt}{-3pt}{red!80},
  }
}

\newtcolorbox{mybox}[1]{%
  base, title = {#1}
}

\dbNewReviewPoints{review}{1, 3, 7, 15, 30, 60}

\dbNewDatabase{ques-book}{
  ques = tl,
  answer = tl,
  count = int|1,
  labels = clist,
  date = date,
}

\begin{dbFilters}{ques-book}
  \dbNewConditional{hard}{labels}{hard}
  \dbNewConditional{bad}{count}{\dbval > 1}
  \dbNewConditional*{review}{date}{review|2022/01/07}
  \dbNewConditional{after}{date}{\dbval > 2022/01/02}
\end{dbFilters}

% show all questions with hyperlink to answers
\dbNewStyle{ques}{ques-book}{
  before-code = {\section{Questions}},
  item-code = {
    \begin{mybox}{%
      \linktarget{answer_\dbIndex}{ques_\dbIndex}{%
        Question \dbarabic%
        \hspace{2em}\dbuse{date}%
        \hspace{2em}\dbuse{labels}%
        \hfill\dbuse{count}%
      }%
    }
      \dbuse{ques}%
    \end{mybox}
  },
  labels/sep = /,
}

% show all questions and answers with hyperlink to questions
\dbNewStyle{answer}{ques-book}{
  before-code = {\section{Questions and Answers}},
  item-code = {
    \begin{mybox}{%
      \linktarget{ques_\dbIndex}{answer_\dbIndex}{%
        Question \dbarabic%
        \hspace{2em}\dbuse{date}%
        \hspace{2em}\dbuse{labels}%
        \hfill\dbuse{count}%
      }%
    }
      \dbuse{ques}\tcbsubtitle{Answer}\dbuse{answer}%
    \end{mybox}
  },
  labels/sep = /,
}

% show all hard questions with hyperlink to answers
\dbNewStyle{hard}{ques-book}{
  before-code = {\section{Hard Questions}},
  item-code = {
    \begin{mybox}{%
      \hyperlink{answer_\dbIndex}{%
        Question \dbarabic%
        \hspace{2em}\dbuse{date}%
        \hspace{2em}\dbuse{labels}%
        \hfill\dbuse{count}%
      }%
    }
      \dbuse{ques}%
    \end{mybox}
  },
  raw-filter = hard,
  labels/sep = /,
}

% show all hard questions that have been answered incorrectly for more than
% one time with hyperlink to answers
\dbNewStyle[hard]{bad}{ques-book}{
  before-code = {\section{Bad Questions}},
  raw-filter = {bad && hard},
}
% show all questions that plan to be reviewed on 2022/01/07 with hyperlink to
% answers
\dbNewStyle[hard]{review}{ques-book}{
  before-code = {\section{Questions to be Reviewed}},
  raw-filter = {review},
}
% show all questions that is record after 2022/01/02 with hyperlink to answers
\dbNewStyle[hard]{after}{ques-book}{
  before-code = {\section{Questions after 2022/01/02}},
  raw-filter = {after},
}

\AtEndDocument{
  \dbshow{review}{ques-book}
  \dbshow{hard}{ques-book}
  \dbshow{bad}{ques-book}
  \dbshow{after}{ques-book}
  \dbshow{ques}{ques-book}
  \dbshow{answer}{ques-book}
}

\begin{document}

\begin{dbitem}{ques-book}[
  date=2022/01/01,
  labels={math, equation, easy},
  count=2
  ]
  \dbsave{ques}{%
    Solve the linear equation: $x + 16 = 31$.
  }
  \dbsave{answer}{%
    $x = 31 - 16 = 15$
  }
\end{dbitem}

\begin{dbitem}{ques-book}[
  date=2022/01/01,
  labels={math, equation, hard},
  count=3
  ]
  \dbsave{ques}{%
    Solve the linear equation: $2y = 16$.
  }
  \dbsave{answer}{%
    $y = 16 / 2 = 8$
  }
\end{dbitem}

\begin{dbitem}{ques-book}[
  date=2022/01/04,
  labels={math, integral, hard},
  count=1
  ]
  \dbsave{ques}{%
    Find the indefinite integral: $\int 2x \dd x$.
  }
  \dbsave{answer}{%
    $\int 2x \dd x = x^2$
  }
\end{dbitem}

\end{document}
\end{tcblisting}

\end{documentation}
\end{document}
