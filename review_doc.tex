\documentclass[full]{l3doc}
\usepackage[scheme=plain]{ctex}
\usepackage{zhlineskip}
\usepackage{enumitem}
\usepackage{indentfirst}
\usepackage{titling}
\usepackage{geometry}
\usepackage{tabularray}
\usepackage{xcolor}

\geometry{
  left=4.5cm,
  right=2cm,
  top=2cm,
  bottom=2cm,
}
\hypersetup {
  CJKbookmarks,
  bookmarksopen,
  bookmarksopenlevel=3,
  pdfstartview=FitH,
  pdfinfo = {
   Title = The package 'review' ,
   Subject = A LaTeX package ,
   Author = Li Changkai
 }
}

\DoNotIndex{\begin, \end}
\setlength{\parskip}{\medskipamount}
\DeclareDocumentEnvironment { note } { +b } {
  \par\textbf{\textsf{NOTE:~}}#1\par
} {}

\AtEndDocument{
  \newgeometry{
    left=2cm,
    right=2cm,
    top=2cm,
    bottom=2cm
  }
  \PrintIndex
}

\ExplSyntaxOn
\dim_new:N \l__my_syntax_dim
\box_new:N \g__my_syntax_box
\NewDocumentEnvironment { Syntax } { s }
  {
    \dim_set:Nn \l__my_syntax_dim
      { \textwidth }
    \hbox_gset:Nw \g__my_syntax_box
      \small \ttfamily
      \begin{minipage}[t]{\l__my_syntax_dim}
        \raggedright\obeyspaces\obeylines
  }
  {
      \end{minipage}
    \hbox_gset_end:
    \IfValueF { #1 } { \smallskip }
    \box_use_drop:N \g__my_syntax_box
    \smallskip
  }

\DeclareDocumentEnvironment { Description } { o +b } {
  \hbox_set:Nn \l_tmpa_box { #1 }
  \dim_set:Nn \l_tmpa_dim { \box_wd:N \l_tmpa_box }
  \begin{itemize}[labelwidth=\l_tmpa_dim, align=left]
    #2
  \end{itemize}
} {  }

\keys_define:nn { review/doc } {
  opt .tl_set:N = \l_opt_tl,
  desc .tl_set:N = \l_desc_tl,
  init .tl_set:N = \l_init_tl,
  init .initial:n = init-none,
}

\box_new:N \l__option_box
\NewDocumentEnvironment { option } { m +b } {
  \keys_set:nn { review/doc } { #1 }
  \hbox_set:Nw \l__option_box
    \small \ttfamily
    \begin{minipage}[t]{\textwidth}
      \obeyspaces\obeylines
      \textcolor{red}{
        \l_opt_tl
        \exp_args:Nx\SpecialOptionIndex{\l_opt_tl}
      }
      {~}\l_desc_tl
      \hfill(
      \tl_if_eq:NnTF \l_init_tl { init-none } { no~value }
        { initially~\texttt{\l_init_tl} }
      )
    \end{minipage}
  \hbox_gset_end:
  \box_use_drop:N \l__option_box
  #2
  \medskip
} {  }

\DeclareDocumentCommand \opt { O{} m }
  { \__codedoc_cmd:no {#1} { #2 } }
\ExplSyntaxOff


\begin{document}
\title{\pkg{review}宏包\protect\footnote{\url{https://github.com/ZhiyuanLck/review}}}
\author{\textit{李昌锴} \texttt{<lichangkai225\@qq.com>}}
\date{2022年1月3日}
\maketitle
\tableofcontents
\newpage

\begin{documentation}

\section{引言}
编写本宏包的动机来源于当前没有一个很好的错题本宏包,可以方便的根据各种条件对错
题进行筛选、排序,然后以自定义的样式展示出来。\pkg{review} 宏包实现了四个核心
功能:数据存储和使用、数据筛选、数据排序、数据展示。

数据只需要存储一次,就可以通过预定义的筛选、排序条件和样式展示部分或全部的数据。
如上所述,本宏包其实实现了一个非常简单的数据库,复习错题的功能只是其中一个应用,
和其他数据库宏包比如 \pkg{datatool} 相比,\pkg{review} 更专注于非图表类型的数
据展示。

宏包基于 \pkg{expl3} 的基础类型构建了6种类型:
\begin{Description}[\texttt{clist}]
  \item[\texttt{date}]
    基于宏包 \pkg{datetime2} 的日期类型,以iso标准存储,支持大小比较,排序(转
    换成字符串)。默认值为 \cs{Today}。
  \item[\texttt{str}]
    字符串类型,支持正则匹配,英文排序。默认值为空。
  \item[\texttt{tl}]
    \meta{token list}类型,支持正则匹配。默认值为空。
  \item[\texttt{int}]
    整数类型,支持大小比较,排序。默认值为0。
  \item[\texttt{fp}]
    浮点数类型,支持大小比较,排序。默认值为0。
  \item[\texttt{clist}]
    逗号分隔的列表类型。默认值为空列表。
\end{Description}

\section{文档接口}
\subsection{\cs{rvNewGroup}}
\begin{function}{\rvNewGroup, \rvNewGroup*}
  \begin{syntax}
    \cs{rvNewGroup} \oarg{base group} \{ \\
    ~~\meta{attr1} = \meta{type spec1}, \\
    ~~\meta{attr2} = \meta{type spec2}, \\
    ~~\ldots{} \\
    \} \\
    \cs{rvNewGroup}* \marg{group} \{ \\
    ~~\meta{attr1} = \meta{type spec1}, \\
    ~~\meta{attr2} = \meta{type spec2}, \\
    ~~\ldots{} \\
    \} \\
  \end{syntax}

\end{function}

  新建一个组来存储数据,不带星号的版本可以指定一个组来继承属性设置,该版本总是
  会舍弃掉之前的定义。

  带星号的版本不会舍弃之前已有的定义,而是将新的选项添加到后面。

  \meta{attr} 为属性名称,\meta{type spec} 负责声明属性类型和属性默认值:

  \noindent\begin{tblr}{
    colspec = {ll},
    column{1} = {font = \ttfamily}
  }
    \meta{attr} = \meta{type} &
    将 \meta{attr} 声明为 \meta{type} 类型 \\
    \meta{attr} = \meta{type}\textbar\meta{default} &
    将 \meta{attr} 声明为 \meta{type} 类型,并且将默认值设置为 \meta{default}。
    \\
  \end{tblr}

  \begin{note}
    每个组都有一个默认的属性 |id| 用来存储数据的索引。
  \end{note}

  下面是定义一个错题组的示例,|question| 和 |answer| 属性用来存储问题和答案,
  |date| 属性存储日期,|info| 属性存储额外信息,|label| 存储题目标签。
\begin{verbatim}
  \rvNewGroup{ques}{
    question = tl,
    answer = tl,
    date = date,
    info = tl,
    label = clist
  }
\end{verbatim}

\subsection{\cs{rvNewGroupStyle} 和样式选项}

\begin{function}{\rvNewGroupStyle}
  \begin{syntax}
    \cs{rvNewGroupStyle} \oarg{base styles} \marg{style} \marg{group} \marg{opts}
  \end{syntax}

  为 \meta{group} 定义一个新的样式 \meta{style},该样式可以基于已有的样式
  \meta{base styles},比如 |\rvNewGroupStyle[base1, base2]{new-style}{ques}{}|。
\end{function}

\bigskip

\begin{option}{opt=filter, desc={= \meta{filter}}, init=-none-}
  为当前样式设置由 \cs{rvCombineFilters} 所定义的过滤器
\end{option}

\begin{option}{
  opt = sort,
  desc = {= \{ \meta{attr spec1}, \meta{attr spec2}, \ldots{} \}},
}
  为当前样式设置排序规则。支持根据 |str|,|date|,|int|,|fp| 类型的数据进行排
  序,支持多级排序。\meta{attr} 表示增序,\meta{attr}* 表示降序。下面例子中,
  使用 |sort-style| 展示数据时的顺序为先按 |level| 降序,|level| 相同的再按出
  生日期 |birth| 增序,以此类推。
\end{option}

\begin{verbatim}
  \rvNewGroup{sort-example}{
    name = str,
    birth = date,
    level = int,
    weight = fp,
  }
  \rvNewGroupStyle{sort-style}{sort-example}{
    sort = { level*, birth, name, weight }
  }
\end{verbatim}

\begin{option}{
  opt = before-code,
  desc = {= \meta{before code}}
}
  该选项用来设置在展示整个组之前需要执行的代码。
\end{option}

\begin{option}{
  opt = after-code,
  desc = {= \meta{after code}}
}
  该选项用来设置在展示整个组之后需要执行的代码。
\end{option}

\begin{option}{
  opt = item-code,
  desc = {= \meta{item code}}
}
  该选项用来设置展示组中每个元素的代码。
\end{option}

\begin{option}{
  opt = {\meta{attr}/before-code},
  desc = {= \meta{before code}}
}
  该选项用来设置展示组中属性 \meta{attr} 对应数据之前需要执行的代码。\cs{rvuse}
  会在展示属性数据前执行此代码。
\end{option}

\begin{option}{
  opt = {\meta{attr}/after-code},
  desc = {= \meta{after code}}
}
  该选项用来设置展示组中属性 \meta{attr} 对应数据之后需要执行的代码。\cs{rvuse}
  会在展示属性数据后执行此代码。
\end{option}

\begin{option}{
  opt = {\meta{attr}/sep},
  desc = {= \meta{sep spec}},
  init = {,~\~}
}
\begin{Syntax}
  \meta{attr}/sep = \meta{sep} \\
  \meta{attr}/sep = \{\meta{sep between two}, \meta{sep between more than two}, \meta{sep between last two}\}
\end{Syntax}

  该选项只适用于类型为 |clist| 的属性,用来设置列表间元素的间隔。第一个版本接
  受一个参数,将所有的元素间隔设置为 \meta{sep}。第二个版本接受逗号分隔的三个
  参数,分别用来设置只有两个元素时的分隔符 \meta{sep between two},超过两个元
  素时的分隔符 \meta{sep between more than two},和最后两个元素之间的分隔符
  \meta{sep between last two}。

\end{option}

\begin{option}{
  opt = item-before-code,
  desc = {= \meta{before code}}
}
  该选项只适用于类型为 |clist| 的属性,用来设置展示列表每个元素前需要执行的代
  码。
\end{option}

\begin{option}{
  opt = item-after-code,
  desc = {= \meta{after code}}
}
  该选项只适用于类型为 |clist| 的属性,用来设置展示列表每个元素后需要执行的代
  码。
\end{option}

\subsection{使用 \cs{rvNewReviewPoints} 定义复习点}

\begin{function}{\rvNewReviewPoints}
  \begin{syntax}
    \cs{rvNewReviewPoints} \marg{name} \marg{points}
  \end{syntax}

  定义名为 \meta{name} 的复习点。这是专门为错题本或复习所定制的功能,
  \meta{points}是一系列整数,现在假设每道错题你都将写错时的日期记录在了 |date|
  属性中,并且你希望每隔2,5,15天复习一次。下面的代码给出了一个实现示例。
\begin{verbatim}
  \rvNewReviewPoints{review-point}{2, 5, 15}            % 定义复习点
  \begin{rvGroupFilters}
    \rvNewConditional{cond1}{date}{review-point|\Today} % 定义复习条件
    \rvCombineConditionals{filter1}{cond1}              % 定义过滤器
  \end{rvGroupFilters}
  \rvNewGroupStyle{review-style}{ques}{filter=filter1}  % 定义展示样式
\end{verbatim}
\end{function}

\subsection{在 \env{rvGroupFilters} 环境中定义过滤器}

\noindent\DescribeEnv{rvGroupFilters}
\begin{Syntax}*
  |\begin{rvGroupFilters}|\marg{group} \\
  ~~\meta{code}
  |\end{rvGroupFilters}| \\
\end{Syntax}

\env{rvGroupFilters}用来定义过滤器,此环境中定义了 \cs{rvNewConditional} 命令
用来定义条件和 \cs{rvCombineConditionals} 命令用来组合条件定义过滤器。过滤器独
立于每个 \meta{group},这意味着你可以在不同组中定义名称相同的过滤条件和过滤器。

\begin{function}{\rvNewConditional, \rvNewConditional*}
  \begin{syntax}
    \cs{rvNewConditional}  \marg{name}        \marg{attr} \marg{cond spec} \\
    \cs{rvNewConditional}* \marg{name}        \marg{attr} \marg{cond spec} \\[2pt]
    \cs{rvNewConditional}  \marg{name} \marg{int/fp attr} \marg{rela} \\
    \cs{rvNewConditional}* \marg{name} \marg{int/fp attr} \marg{rela} \\
    \cs{rvNewConditional}  \marg{name} \marg{str/tl attr} \marg{regex} \\
    \cs{rvNewConditional}* \marg{name} \marg{str/tl attr} \marg{regex} \\
    \cs{rvNewConditional}  \marg{name}  \marg{clist attr} \marg{val list} \\
    \cs{rvNewConditional}* \marg{name}  \marg{clist attr} \marg{val list} \\
    \cs{rvNewConditional}  \marg{name}   \marg{date attr} \{ \meta{review points}\textbar\meta{date} \} \\
    \cs{rvNewConditional}* \marg{name}   \marg{date attr} \marg{rela}
  \end{syntax}

  \cs{rvNewConditional} 用来定义名为 \meta{name} 的条件,\meta{attr} 指定条件
  所绑定的属性,在 \meta{cond spec} 中可以用 \cs{rval} 指代属性的值。
\end{function}

  对于类型为 |int| 和 |fp| 的属性,两个版本的定义是一致的,\meta{rela} 可
  以是单个关系式,比如 |\rval > 3|,也可以是组合关系式,比如
  |\rval > 3 && \rval < 10.2|。支持的操作符有 |<, >, =, ==, !=, >=, <=, !|。

  对于类型为 |str| 和 |tl| 的属性,\meta{regex} 为正则表达式,
  \cs{rvNewConditional} 表示部分匹配,\cs{rvNewConditional*} 表示整体匹配。
\begin{verbatim}
  \rvNewConditional{cond1}{str-attr}{abc}  % 匹配 abc, abcd, 1abc, =abc= 等
  \rvNewConditional*{cond2}{str-attr}{abc} % 只匹配 abc
\end{verbatim}

  对于类型为 |clist| 的属性,使用 \cs{rvNewConditional} 定义的条件只要
  \meta{val list} 中的任意一个元素在属性值(列表)中则条件成立;使用
  \cs{rvNewConditional*} 定义的条件只有 \meta{val list} 中每一个值都在属性值
  (列表)中条件才成立。

\begin{verbatim}
  \rvNewConditional{cond1}{clist-attr}{a, b, c}  % a, b, d 满足条件
  \rvNewConditional*{cond2}{clist-attr}{a, b, c} % a, b, d 不满足条件
\end{verbatim}

  对于类型为 |date| 的属性,\cs{rvNewConditional} 使用复习点来定义过滤条件,
  \meta{review points} 是 \cs{rvNewReviewPoints} 定义的复习点,\meta{date} 是
  用来比较的日期;\cs{rvNewConditional*} 使用\textbf{单个}关系式来定义过滤条件。
  支持的操作符有 |<, >, =, ==, !=, >=, <=|。

\begin{function}{\rvCombineConditionals}
  \begin{syntax}
    \cs{rvCombineConditionals} \marg{name} \marg{cond combination} \oarg{info}
  \end{syntax}

  \cs{rvCombineConditionals} 定义名为 \marg{name} 的过滤器,并将
  \cs{rvNewConditional} 定义的条件组合起来,比如
  \verb=\rvCombineConditionals{filter}{(cond1 && cond2) || !cond3}=。
  \meta{cond combination} 中可以使用的关系操作符为 \verb=&&, ||, !=。
  可以将 \opt{filter} 选项设置为 \meta{name} 来应用过滤器。\meta{info} 为过滤
  器的相关信息,在展示组的时候可以用 \cs{rvFilterInfo} 指代。
\end{function}

\subsection{使用 \env{rvitem} 环境存储数据}

\noindent\DescribeEnv{rvitem}
\begin{Syntax}*
  |\begin{rvitem}|\marg{group}[ \\
  ~~\meta{attr1} = \meta{val1}, \\
  ~~\meta{attr2} = \meta{val2}, \\
  ~~\ldots{} \\
  ] \\
  ~~\meta{code} \\
  |\end{rvitem}|
\end{Syntax}

  \env{rvitem} 环境用来存储数据。有两种存储数据的方法,较短的数据可以在选项列
  表中通过键值对设置值,较长的数据可以在 \meta{code} 中使用 \cs{rvsave} 存储。
  \cs{rvsave}会覆盖选项中设置的值。没有设置的值将会被设置为全局默认值,下面给
  出一个存储示例。

\begin{verbatim}
  \begin{rvitem}[date=2022-01-01, info=测试]
    \rvsave{question}{这是一个测试问题}
    \rvsave{answer}{这是一个测试答案}
  \end{rvitem}
\end{verbatim}

\subsection{\cs{rvsave} 和 \cs{rvuse}}

\begin{function}{\rvsave, \rvuse}
  \begin{syntax}
    \cs{rvsave} \marg{attr} \marg{data} \\
    \cs{rvuse}  \marg{attr} \\
  \end{syntax}

  \cs{rvsave} 用来存储数据,只能在 \env{item} 环境中使用。\cs{rvuse} 用来使用
  数据,只能在 \opt{item-code} 选项中使用。
\end{function}

\subsection{使用 \cs{rvshow} 展示组}

\begin{function}{\rvshow}
  \begin{syntax}
    \cs{rvshow} \marg{style} \marg{group}
  \end{syntax}

  使用 \meta{style} 样式来展示 \meta{group}。
\end{function}

\subsection{条件判别式}

\begin{function}{\rvIfEmptyT, \rvIfEmptyF, \rvIfEmptyTF}
  \begin{syntax}
    \cs{rvIfEmptyTF} \marg{true code} \marg{false code} \\
    \cs{rvIfEmptyT} \marg{true code} \\
    \cs{rvIfEmptyF} \marg{false code}
  \end{syntax}

  该判别式用来判断当前组是否为空。下面的示例展示了如何预防空的列表环境。
\end{function}

\begin{verbatim}
  \rvNewGroupStyle{style-cond1}{group-test}{
    before-code = {\rvIfEmptyF{\begin{enumerate}}},
    after-code = {\rvIfEmptyF{\end{enumerate}}},
    item-code = {\item \rvuse{attr-test}}
  }
\end{verbatim}

\begin{function}{\rvItemIfEmptyT, \rvItemIfEmptyF, \rvItemIfEmptyTF}
  \begin{syntax}
    \cs{rvItemIfEmptyTF} \marg{true code} \marg{false code} \\
    \cs{rvItemIfEmptyT} \marg{true code} \\
    \cs{rvItemIfEmptyF} \marg{false code}
  \end{syntax}

  该判别式用来判断当前元素是否为空。下面的示例展示了如何在展示 |group-test| 组
  中 |text| 属性的元素时在元素非空的时候前后都加上 |*|。
\end{function}

\begin{verbatim}
  \rvNewGroupStyle{style-cond2}{group-test}{
    text/before-code = {\rvItemIfEmptyF{*}},
    text/after-code = {\rvItemIfEmptyF{*}},
  }
\end{verbatim}

\begin{function}{\rvClistItemIfEmptyT, \rvClistItemIfEmptyF, \rvClistItemIfEmptyTF}
  \begin{syntax}
    \cs{rvClistItemIfEmptyTF} \marg{true code} \marg{false code} \\
    \cs{rvClistItemIfEmptyT} \marg{true code} \\
    \cs{rvClistItemIfEmptyF} \marg{false code}
  \end{syntax}

  该判别式用来判断列表属性中的元素是否为空。下面的示例展示了如何在展示 |group-test| 组
  中 |labels| 属性(标签列表)的元素时在标签非空的时候前后都加上 |*|。
\end{function}

\begin{verbatim}
  \rvNewGroupStyle{style-cond3}{group-test}{
    labels/item-before-code = {\rvClistItemIfEmptyF{*}},
    labels/item-after-code = {\rvClistItemIfEmptyF{*}},
  }
\end{verbatim}

\subsection{常量}

\begin{function}{\rvGroup, \rvFilterName, \rvFilterInfo}
  \cs{rvGroup} 指代组名,\cs{rvFilterName} 指代过滤器名称,\cs{rvFilterInfo}
  指代过滤器额外信息。
\end{function}

\end{documentation}

% \title{Package \texttt{review}}
% \author{Li Changkai \texttt{<lichangkai225\@qq.com>}}
% \date{2022-01-03}
% \maketitle
% \section{测试}
\end{document}
