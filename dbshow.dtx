% \iffalse meta-comment
%<*internal>
\iffalse
%</internal>
%<*readme>
Package dbshow
==============

Introduction
------------

The package provides four core functions:

1. data storage
2. data filtering
3. data sorting
4. data display

All data is saved once and then you can display these data with custom filters,
orders and styles. The package can be used, for example, to record and
display something you'd like to review, maybe the question you always answered
incorrectly or some forgettable knowledge. But obviously, the package is much
more powerful and extensible for more interesting tasks depending on the
individual.

Install
-------

If you are using TeX Live, install `dbshow` by `tlmgr` package manager

    tlmgr update --self --all
    tlmgr install dbshow

You can also download the zip file from
[CTAN](https://ctan.org/pkg/dbshow) or
[GitHub](https://github.com/ZhiyuanLck/dbshow/releases/latest).

The zip file should contain the file `dbshow.dtx`, and you can extract
`dbshow.sty` from it by

    tex dbshow.dtx

Compile Document
----------------

Use following commands to compile the document:

    git clone https://github.com/ZhiyuanLck/dbshow.git
    cd dbshow
    latexmk dbshow.dtx

Copyright and License
---------------------

    Copyright (C) 2022- by Changkai Li <lichangkai225@qq.com>
    --------------------------------------------------------------------------

    This work may be distributed and/or modified under the
    conditions of the LaTeX Project Public License, either
    version 1.3c of this license or (at your option) any later
    version. This version of this license is in
       http://www.latex-project.org/lppl/lppl-1-3c.txt
    and the latest version of this license is in
       http://www.latex-project.org/lppl.txt
    and version 1.3 or later is part of all distributions of
    LaTeX version 2005/12/01 or later.

    This work has the LPPL maintenance status "maintained".

    The Current Maintainer of this work is Changkai Li.

    This package consists of the file  dbshow.dtx,
                 and the derived files dbshow.pdf,
                                       dbshow.sty,
                                       dbshow.ins and
                                       README.md (this file).
%</readme>
%<*internal>
\fi
\begingroup
  \def\temp{LaTeX2e}
\expandafter\endgroup\ifx\temp\fmtname\else
\csname fi\endcsname
%</internal>
%<*install>

\input l3docstrip.tex
\keepsilent
\askforoverwritefalse
\preamble

    Copyright (C) 2022- by Changkai Li <lichangkai225@qq.com>
    --------------------------------------------------------------------------

    This work may be distributed and/or modified under the
    conditions of the LaTeX Project Public License, either
    version 1.3c of this license or (at your option) any later
    version. This version of this license is in
       http://www.latex-project.org/lppl/lppl-1-3c.txt
    and the latest version of this license is in
       http://www.latex-project.org/lppl.txt
    and version 1.3 or later is part of all distributions of
    LaTeX version 2005/12/01 or later.

    This work has the LPPL maintenance status "maintained".

    The Current Maintainer of this work is Changkai Li.

    This package consists of the file  dbshow.dtx,
                 and the derived files dbshow.pdf,
                                       dbshow.sty,
                                       dbshow.ins and
                                       README.md.
    --------------------------------------------------------------------------

\endpreamble
\postamble

    --------------------------------------------------------------------------
    This package consists of the file  dbshow.dtx,
                 and the derived files dbshow.pdf,
                                       dbshow.sty,
                                       dbshow.ins and
                                       README.md.
\endpostamble

\generate{
%</install>
%<*internal>
  \usedir{source/latex/dbshow}
  \file{dbshow.ins}{\from{\jobname.dtx}{install}}
%</internal>
%<*install>
  \usedir{tex/latex/dbshow}
  \file{dbshow.sty}{\from{\jobname.dtx}{package}}
  \nopreamble\nopostamble
  \usedir{doc/latex/dbshow}
  \file{README.md} {\from{\jobname.dtx}{readme}}
}
\endbatchfile
%</install>
%<*internal>
\fi
%</internal>
%<package>\NeedsTeXFormat{LaTeX2e}
%<+package|config>\GetIdInfo$Id: dbshow.dtx 461d9bb 2022-01-14 02:50:24 +0800 Changkai Li <lichangkai225@qq.com> $
%<package>  {Database to store and display data}
%<package>\ProvidesExplPackage{\ExplFileName}
%<package|config>  {\ExplFileDate}{1.4}{\ExplFileDescription}
%<*driver>
\documentclass[full]{l3doc}
\usepackage[scheme=plain, fontset=ubuntu]{ctex}
\usepackage{dbshow}
\usepackage{adjustbox}
\usepackage{zhlineskip}
\usepackage{enumitem}
\usepackage{indentfirst}
\usepackage{titling}
\usepackage{geometry}
\usepackage{tabularray}
\usepackage{xcolor}
\usepackage{tabularray}
\usepackage{zhnumber}
\usepackage{tcolorbox}
\tcbuselibrary{skins, minted, breakable, xparse}
\usepackage{accsupp}

\geometry{
  left=4.5cm,
  right=2cm,
  top=2cm,
  bottom=2cm,
}
\hypersetup {
  CJKbookmarks,
  bookmarksopen,
  bookmarksopenlevel=3,
  pdfstartview=FitH,
  pdfinfo = {
   Title = The package 'dbshow' ,
   Subject = A LaTeX package ,
   Author = Li Changkai
 }
}

\IndexPrologue {
  \part*{Index}
  \markboth{Index}{Index}
  \addcontentsline{toc}{part}{Index}
  The~italic~numbers~denote~the~pages~where~the~
  corresponding~entry~is~described,~
  numbers~underlined~point~to~the~definition,~
  all~others~indicate~the~places~where~it~is~used.
}

\GlossaryPrologue {
  \part*{Change~History}
  {\GlossaryParms\ttfamily\hyphenchar\font=`\-}
  \markboth{Change~History}{Change~History}
  \addcontentsline{toc}{part}{Change~History}
}

\DoNotIndex{\begin, \end}
\setlength{\parskip}{\medskipamount}

\newcommand\tikzmark[1]{\tikz \coordinate[overlay, remember picture] (#1);}
\def\levelchar{?}
\def\orbar{\textup{\textbar}}
\def\defaultval#1{\textbf{\textup{#1}}}
\def\TF{true\orbar false}
\def\TTF{\defaultval{true}\orbar false}
\def\TFF{true\orbar\defaultval{false}}
\def\zhbefore{\textbf{前}}
\def\zhafter{\textbf{后}}
\def\enbefore{\textbf{before}~}
\def\enafter{\textbf{after}~}
\DeclareDocumentEnvironment { note } { +b } {
  \par\textbf{\textsf{NOTE:~}}#1\par
} {}

\definecolor {exambg}    {RGB}  {248, 241, 224} % 示例背景
\definecolor {examno}    {RGB}  {176, 101, 90}  % 示例行号
\definecolor {examnobg}  {RGB}  {241, 225, 208} % 示例行号背景
\definecolor {examframe} {RGB}  {156, 129, 110} % 示例边框(标题背景)
\definecolor {option}    {HTML} {009933}        % 选项
\definecolor {cs}        {HTML} {FF6600}        % 命令
\definecolor {env}       {HTML} {C81531}        % 环境
\definecolor {link}      {HTML} {33539E}        % 链接

% \definecolor {exambg}    {RGB}  {241, 225, 208} % 示例背景
% \definecolor {examno}    {HTML} {F3AA20} % 示例行号
% \definecolor {examnobg}  {HTML} {8B4C70} % 示例行号背景
% \definecolor {examframe} {HTML} {58094F} % 示例边框(标题背景)
% \definecolor {option}    {HTML} {8B4C70} % 选项
% \definecolor {cs}        {HTML} {F3AA20} % 命令
% \definecolor {cs}        {HTML} {F9A911} % 命令
% \definecolor {env}       {HTML} {669933} % 环境
\hypersetup{linkcolor=link}
\tcbset{
  exam-base/.style={
    listing engine=minted, listing and text,
    minted style=emacs, breakable,
    minted options={fontsize=\small,breaklines,linenos,numbersep=3mm},
    colback=exambg, colframe=examframe,
    left=5mm, enhanced,
    fonttitle=\small\sffamily\bfseries, fontlower=\small,
    overlay={
      \begin{tcbclipinterior}
        \fill[examnobg]
        (frame.south west) rectangle ([xshift=5mm]frame.north west);
      \end{tcbclipinterior}
    }
  }
}
\DeclareTCBListing[auto counter]{example}{ O{} D(){} m }{%
  exam-base, title={示例 \thetcbcounter:#3}, label={cn-#2}, #1
}
\DeclareTCBListing[auto counter]{example*}{ O{} D(){} m }{%
  exam-base, title={Example \thetcbcounter: #3}, label={en-#2}, #1
}
\renewcommand{\theFancyVerbLine}{%
  \ttfamily\textcolor{examno}{%
    \scriptsize\oldstylenums{%
      \protect\BeginAccSupp{ActualText={}}%
      \arabic{FancyVerbLine}%
      \protect\EndAccSupp{}%
    }%
  }%
}

\ExplSyntaxOn
\makeatletter

\DeclareDocumentCommand { \csnot } { m } {
  \group_begin:\ttfamily
  \tl_set:Nn \l_csnot_tl { \c_backslash_str #1 }
  \tl_replace_all:Nnn \l_csnot_tl { ~ } { \@xobeysp }
  \l_csnot_tl
  \group_end:
}

\DeclareDocumentEnvironment{ arguments } { O{1} } {
  \tl_set:Nx \l_tmpa_tl {
    \int_case:nn {#1} {
      { 1 } { \# }
      { 2 } { \#\# }
      { 3 } { \#\#\#\# }
    }
  }
  \enumerate [
    nolistsep ,
    label = \texttt{\l_tmpa_tl\arabic*} ~ : ,
    labelsep = * ,
    labelwidth = !,
    align=left,
  ]
} { \endenumerate }

\DoNotIndex{\begin, \end}

\renewenvironment{theglossary}{
\glossary@prologue\GlossaryParms \let\item\@idxitem \ignorespaces}{}

\cs_gset_eq:NN \dbshowdocnull \use_none:n

% #1 sort #2 group #3 date #4 type #5 desc
\cs_new_protected:Nn \dbshowdoc_changes:nnnnn {
  \@bsphack\group_begin:\@sanitize
  \catcode`\\\z@ \catcode`\ 10 \MakePercentIgnore
  \exp_args:Nx \glossary {
    \dbshowdocnull{#1}~
    \exp_not:N \textbf{#2}
    \exp_not:n { \raisebox{1em}{\pdfbookmark[1]{#2}{#2}} }
    \levelchar
    \exp_not:N \textup{#3}\c_space_tl
    \exp_not:N \textbf{#4}\c_colon_str\c_space_tl
    \exp_not:n {#5}
  }
  \group_end:\@esphack
}
\cs_generate_variant:Nn \dbshowdoc_changes:nnnnn { xxnnn, VVnnn }

\cs_new_protected:Nn \dbshowdoc_print_version:n {
  v#1
  \int_compare:nNnT { \str_count:n {#1} } = { 3 } { \phantom{0} }
  \c_space_tl
}
\cs_gset_eq:NN \dbshowdocver \dbshowdoc_print_version:n

% #1 ver #2 date #3 type #4 desc
\cs_new_protected:Nn \dbshowdoc_changes_what:nnnn {
  \regex_extract_once:nnNT { doc|macro|bug|option }
  {#3} \l_tmpa_seq {
    \tl_set:Nx \l_tmpa_tl {
      \str_case_e:nn { \seq_item:Nn \l_tmpa_seq { 1 } } {
        { doc } { Documentation }
        { macro } { Macro }
        { bug } { Bug }
        { option } {  Option }
        { logic } {  Logic }
      }
    }
    \dbshowdoc_changes:VVnnn \l_tmpa_tl \l_tmpa_tl
      { \dbshowdocver{#1}#2 } {#3} {#4}
  }
}

% #1 ver #2 date #3 type #4 desc
\cs_new_protected:Nn \dbshowdoc_changes_how:nnnn {
  \regex_extract_once:nnNT { Add|Update|Remove|Fix }
  {#3} \l_tmpa_seq {
    \dbshowdoc_changes:xxnnn
      { \seq_item:Nn \l_tmpa_seq { 1 } }
      { \seq_item:Nn \l_tmpa_seq { 1 } }
      { \dbshowdocver{#1}#2 } {#3} {#4}
  }
}

\cs_new:Nn \dbshowdoc_fill_two:n {
  \int_compare:nNnTF {#1} > { 9 }
    {#1} { 0#1 }
}

% #1 major number #2 minor number
\cs_new_protected:Npn \dbshowdoc_changes_ver:w #1.#2\dbshowdoc_stop {
  \dbshowdoc_changes:xxnnn { #1.\dbshowdoc_fill_two:n {#2} }
}

% #1 ver #2 date #3 type #4 desc
\DeclareDocumentCommand \changes { m m m m } {
  \dbshowdoc_changes_ver:w #1\dbshowdoc_stop {#1} {#2} {#3} {#4}
  \dbshowdoc_changes_what:nnnn {#1} {#2} {#3} {#4}
  \dbshowdoc_changes_how:nnnn {#1} {#2} {#3} {#4}
}
\makeatother

\DeclareDocumentEnvironment { Description } { o +b } {
  \hbox_set:Nn \l_tmpa_box {#1}
  \dim_set:Nn \l_tmpa_dim { \box_wd:N \l_tmpa_box }
  \begin{itemize}[
    itemindent=0pt,
    labelindent=0pt,
    labelwidth=\l_tmpa_dim,
    leftmargin=!,
    align=left]
    #2
  \end{itemize}
} {}

\cs_gset_protected:Npn \__codedoc_typeset_function_block:nN #1#2 {
    \__codedoc_function_index:x
      { #1 \bool_if:NT #2 { \tl_to_str:n {TF} } }
    \__codedoc_function_label:xN {#1} #2
    \color{cs}\bfseries #1
    \bool_if:NT #2 { \__codedoc_typeset_TF: }
    \__codedoc_typeset_expandability:
    \seq_if_empty:NF \g__codedoc_variants_seq
      { \__codedoc_typeset_variant_list:nN {#1} #2 }
    \\
}
\cs_new_protected:Nn \dbshowdoc_function_begin:Nn {
  \cs_set_eq:NN \__codedoc_tmp_cs:nN \__codedoc_typeset_function_block:nN
  \cs_set_protected:Npn \__codedoc_typeset_function_block:nN ##1##2 {
    \__codedoc_function_label:xN {##1} ##2
    \hbox_set:Nn \l_tmpa_box {##1}
    \group_begin: \color{#2}\bfseries
    \int_compare:nTF { \str_count:n {##1} <= 20 }
      {##1} { \adjustbox{width=.7\marginparwidth, height=\box_ht:N \l_tmpa_box}{##1} }
    \group_end:
    #1{##1}
    \__codedoc_typeset_expandability: \\
  }
}
\cs_new_protected:Nn \dbshowdoc_function_end: {
  \cs_set_eq:NN \__codedoc_typeset_function_block:nN \__codedoc_tmp_cs:nN
}
\DeclareDocumentEnvironment { option } { O{} +v }
  {
    \dbshowdoc_function_begin:Nn \SpecialOptionIndex { option }
    \__codedoc_function:nnw {#1} {#2}
  }
  {
    \__codedoc_function_end:
    \dbshowdoc_function_end:
  }
\DeclareDocumentEnvironment { environment } { O{} +v }
  {
    \dbshowdoc_function_begin:Nn \SpecialEnvIndex { env }
    \__codedoc_function:nnw {#1} {#2}
  }
  {
    \__codedoc_function_end:
    \dbshowdoc_function_end:
  }

\makeatletter
\cs_gset_protected:Npn \__codedoc_cmd:nn #1#2
  {
    \bool_set_false:N \l__codedoc_cmd_noindex_bool
    \bool_set_true:N \l__codedoc_cmd_replace_bool
    \tl_set:Nn \l__codedoc_cmd_index_tl { \q_no_value }
    \tl_set:Nn \l__codedoc_cmd_module_tl { \q_no_value }
    \keys_set:nn { l3doc/cmd } {#1}
    \tl_set:Nn \l__codedoc_cmd_tl {#2}
    \bool_if:NT \l__codedoc_cmd_replace_bool
      {
        \tl_set_rescan:Nnn \l__codedoc_tmpb_tl { } { _ }
        \tl_replace_all:Non \l__codedoc_cmd_tl \l__codedoc_tmpb_tl { _ }
        \__codedoc_replace_at_at:N \l__codedoc_cmd_tl
        \tl_replace_all:Nno \l__codedoc_cmd_tl { _ } \l__codedoc_tmpb_tl
      }
    \mode_if_math:T { \mbox }
      {
        \bool_if:NT \l__codedoc_allow_indexing_bool { \__codedoc_target: }
        \verbatim@font
        \__codedoc_if_almost_str:VT \l__codedoc_cmd_tl
          {
            \__kernel_tl_set:Nx \l__codedoc_cmd_tl { \tl_to_str:N \l__codedoc_cmd_tl }
            \bool_if:NT \g__codedoc_cs_break_bool
              {
                \regex_replace_all:nnN
                  { ([^\\\_]\_*) \_ ([^\_]) }
                  { \1 \c{BreakableUnderscore} \2 }
                  \l__codedoc_cmd_tl
              }
          }
        \tl_replace_all:Nnn \l__codedoc_cmd_tl { ~ } { \@xobeysp }
        \textbf{\l__codedoc_cmd_tl}
        \@
      }
    \bool_if:NT \l__codedoc_allow_indexing_bool
     {
      \bool_if:NF \l__codedoc_cmd_noindex_bool
       {
        \quark_if_no_value:NF \l__codedoc_cmd_index_tl
          {
            \__kernel_tl_set:Nx \l__codedoc_cmd_tl
              { \c_backslash_str \exp_not:o { \l__codedoc_cmd_index_tl } }
          }
        \exp_args:No \__codedoc_key_get:n { \l__codedoc_cmd_tl }
        \quark_if_no_value:NF \l__codedoc_cmd_module_tl
          {
            \__kernel_tl_set:Nx \l__codedoc_index_module_tl
              { \tl_to_str:N \l__codedoc_cmd_module_tl }
          }
        \__codedoc_special_index_module:ooonN
          { \l__codedoc_index_key_tl }
          { \l__codedoc_index_macro_tl }
          { \l__codedoc_index_module_tl }
          { usage }
          \l__codedoc_index_internal_bool
       }
     }
  }
\cs_generate_variant:Nn \__codedoc_cmd:nn { no }
\makeatother

% #1 color #2 opt #3 content
\cs_new_protected:Nn \dbshowdoc_cmd:nnn {
  \__codedoc_get_hyper_target:xN {#3} \l_tmpa_tl
  \hyperref[\l_tmpa_tl]{\textcolor{#1}{\__codedoc_cmd:no {#2} {#3}}}
}
\DeclareDocumentCommand \opt { O{} m }
  { \dbshowdoc_cmd:nnn { option } {#1} {#2} }
\DeclareDocumentCommand \nopt { O{} m }
  { \textcolor{option}{\textbf{#2}} }
\DeclareDocumentCommand \env { O{} m }
  { \dbshowdoc_cmd:nnn { env } {#1} {#2} }
\DeclareDocumentCommand \cs  { O{} m }
  { \dbshowdoc_cmd:nnn { cs } {#1} { \c_backslash_str #2 } }

\NewDocumentCommand \linktarget { m m m } {%
  \hyperlink{#1}{#3}%
  \raisebox{1em}{\hypertarget{#2}{}}%
}

\DeclareDocumentCommand \GetFileId { m } {
  \GetFileInfo {#1}
  \file_get:nnNTF { \c_sys_jobname_str .id }
    { \int_set:Nn \tex_endlinechar:D { -1 } } \l__ctxdoc_tmp_tl
    { \exp_after:wN \GetIdInfo \l__ctxdoc_tmp_tl }
    { \GetIdInfo $Id$ }
    { \fileinfo }
}

\DeclareDocumentCommand \inidef { s d() s o } {
  \group_begin:
  \hfill\normalfont(
  initially~
  \IfBooleanTF {#1} {empty}
    { \IfValueTF {#2} { \texttt{#2} } { unset } }
  ,~
  \IfBooleanTF {#3} {default empty}
    { \IfValueTF {#4} { default~\texttt{#4} } { no~default } }
  )
  \group_end:
}
\ExplSyntaxOff

\begin{document}
\DocInput{dbshow.dtx}
\newgeometry{
  left=2cm,
  right=2cm,
  top=2cm,
  bottom=2cm
}
\changes{1.4}{2022-01-10}{Fix doc code}{wrong changes history sorting}
\PrintChanges
\PrintIndex
\end{document}
%</driver>
% \fi
%
% \CheckSum{0}
% \GetFileId{dbshow.sty}
%
% \ExplSyntaxOn
% \tl_set:Nx \cnfiledate{ \exp_args:NV \zhdate\filedate }
% \ExplSyntaxOff
%
% \title{
%   \pkg{dbshow} 宏包 \fileversion%
%   \protect\footnote{%
%     代码仓库:\url{https://github.com/ZhiyuanLck/dbshow},
%     QQ群:788706534}
%   \rlap{\makebox[4cm][r]{
%     \normalsize $\Longrightarrow$ \color{red}
%     \linktarget{en}{zh}{English Version}
%   }}
% }
% \author{\textit{李昌锴} \protect\path{<lichangkai225@qq.com>}}
% \date{\cnfiledate}
% \maketitle
%
% \tableofcontents
%
% \begin{documentation}
%
% \section{引言}
% 编写本宏包的动机来源于当前没有一个很好的错题本宏包,可以方便的根据各种条件对错
% 题进行筛选、排序,然后以自定义的样式展示出来。\pkg{dbshow} 宏包实现了四个核心
% 功能:数据存储和使用、数据筛选、数据排序、数据展示。
%
% 数据只需要存储一次,就可以通过预定义的筛选、排序条件和样式展示部分或全部的数据。
% 如上所述,本宏包其实实现了一个非常简单的数据库,复习错题的功能只是其中一个应用,
% 和其他数据库宏包比如 \pkg{datatool} 相比,\pkg{dbshow} 更专注于非图表类型的数
% 据展示。
%
% \changes{1.4}{2022-01-10}{Add check}{version of \pkg{l3kernel}}
% \pkg{dbshow} 依赖版本日期至少为 |2022-11-07| 的 \pkg{l3kernel}。
%
% \changes{1.3}{2022-01-09}{Add doc}{descripton for expansion}
% \begin{itemize}
%   \item 名字后带有 $\star$ 的命令是可以完全展开的(fully-expandable);
%   \item 名字后带有 \ding{73} 的命令可以有限制地展开(restricted-expandable);
%   \item 名字后不带有特殊字符的命令是不可展开的(non-expandable);
%   \item 名字后带有 $\star$ 的选项不影响相关的代码的是否可展开;
%   \item 名字后带有 \ding{73} 的选项是否影响相关代码的可展开性取决于选项的设置;
%   \item 名字后不带有特殊字符的选项会使与之相关的代码变得不可展开。
% \end{itemize}
%
% \subsection{数据类型}
%
% 宏包基于 \pkg{expl3} 的基础类型构建了6种类型:
% \begin{Description}[\texttt{clist}]
%   \item[\texttt{date}]
%     日期类型,以 |yyyy/mm/dd| 形式存储,支持大小比较,排序(转
%     换成字符串)。默认值为 \cs{dbtoday}。
%   \item[\texttt{str}]
%     字符串类型,支持正则匹配,英文排序。默认值为空。
%   \item[\texttt{tl}]
%     \meta{token list}类型,支持正则匹配。默认值为空。
%   \item[\texttt{int}]
%     整数类型,支持大小比较,排序。默认值为0。
%   \item[\texttt{fp}]
%     浮点数类型,支持大小比较,排序。默认值为0。
%   \item[\texttt{clist}]
%     逗号分隔的列表类型。默认值为空列表。
% \end{Description}
%
% \changes{1.3}{2022-01-08}{Remove dependency}{\pkg{datatime2}}
% 除了日期类型,所有类型都是 \pkg{expl3} 的内置类型。\pkg{dbshow} 构建了一个简单
% 的 |date| 类型,支持转换成整数以及带样式的打印。
%
% \subsection{与 \pkg{datatool} 的区别}
%
% \changes{1.2}{2022-01-07}{Add doc}{add comparison to \pkg{datatool}}
%
% 从核心功能上看,\pkg{dbshow} 和 \pkg{datatool} 实现了相同的功能。区别在于
% \pkg{dbshow} 基于 \pkg{expl3} 实现,支持字符串的正则匹配,还支持多级排序。使用
% 方式上更倾向于样式与内容分离,所有的样式都可以通过选项提前定义好并且可以复用。
% \pkg{dbshow} 并没有实现从外部文件读取数据以及将数据持久化的功能,我认为这些应
% 该是更专业的外部程序的工作而不应该在 \LaTeX 中设计这些功能。因此,\pkg{dbshow}
% 只提供了一个运行时的临时数据库,足够轻便且满足大部分正常需求。如果你想删除或修
% 改数据库中某一条记录,请去对应的位置删除或修改掉对应的 \env{dbitem} 环境,而不
% 是让宏包提供一个输出某一行记录的命令。某种意义上记录数据库的 \TeX 源文件本身就
% 是数据的一种持久化。
%
% \section{接口文档}
%
% \subsection{创建、展示和清空数据库}
%
% \changes{1.4}{2022-01-10}{Update macro}{change \orbar\ to \cs{dbshow_sep} in
% weird argument of \cs{dbshow_process_default_value:w}}
% \begin{function}[added=2022-01-05, updated=2022-01-10]{\dbNewDatabase, \dbNewDatabase*}
%   \begin{syntax}
%     \cs{dbNewDatabase} \oarg{base database} \marg{database} \{ \\
%     ~~\meta{attr1} = \meta{type spec1}, \\
%     ~~\meta{attr2} = \meta{type spec2}, \\
%     ~~\ldots{} \\
%     \} \\
%     \cs{dbNewDatabase}* \marg{database} \{ \\
%     ~~\meta{attr1} = \meta{type spec1}, \\
%     ~~\meta{attr2} = \meta{type spec2}, \\
%     ~~\ldots{} \\
%     \} \\
%   \end{syntax}
%
% \end{function}
%
%   新建一个数据库,不带星号的版本可以指定一个数据库来继承其属性设置,该版本总是
%   会舍弃掉之前的定义。
%
%   带星号的版本不会舍弃之前已有的定义,而是将新的选项添加到后面。
%
%   \meta{attr} 为属性名称,\meta{type spec} 负责声明属性类型和属性默认值:
%
%   \noindent\begin{tblr}{
%     colspec = {ll},
%     column{1} = {font = \ttfamily}
%   }
%     \meta{attr} = \meta{type} &
%     将 \meta{attr} 声明为 \meta{type} 类型 \\
%     \meta{attr} = \meta{type}\orbar\meta{default} &
%     将 \meta{attr} 声明为 \meta{type} 类型,并且将默认值设置为 \meta{default}。
%     \\
%   \end{tblr}
%
%   \begin{note}
%     每个数据库都有一个默认的属性 |id| 用来存储数据的索引。
%   \end{note}
%
%   下面是定义一个错题数据库的示例,|question| 和 |answer| 属性用来存储问题和答
%   案,|date| 属性存储日期,|info| 属性存储额外信息,|labels| 存储题目标签。
% \begin{verbatim}
%   \dbNewDatabase{ques}{
%     question = tl,
%     answer = tl,
%     date = date,
%     info = tl,
%     labels = clist
%   }
% \end{verbatim}
%
% \begin{function}[added=2022-01-05]{\dbshow}
%   \begin{syntax}
%     \cs{dbshow} \marg{style} \marg{database}
%   \end{syntax}
%
%   使用 \meta{style} 样式来展示 \meta{database}。
% \end{function}
%
% \changes{1.2}{2022-01-07}{Add macro}{\cs{dbclear}}
% \begin{function}[added=2022-01-07]{\dbclear}
%   \begin{syntax}
%     \cs{dbclear} \marg{database}
%   \end{syntax}
%   清空 \meta{database} 里的所有内容。
% \end{function}
%
% \subsection{\cs{dbNewStyle} 和样式选项}
%
% \begin{function}[added=2022-01-05, updated=2022-01-15]{\dbNewStyle}
%   \begin{syntax}
%     \cs{dbNewStyle} \oarg{base styles} \marg{style} \marg{database} \marg{opts}
%   \end{syntax}
%
%   为 \meta{database} 定义一个新的样式 \meta{style},该样式可以基于已有的样式
%   \meta{base styles},比如 |\dbNewStyle[base1, base2]{new-style}{ques}{}|。
% \end{function}
%
% \subsubsection{通用选项}
%
% \begin{option}[added=2022-01-05]{filter}
%   \begin{syntax}
%     \opt{filter} = <filter> \inidef(-none-)
%   \end{syntax}
%
%   为当前样式设置由 \cs{dbCombineFilters} 所定义的过滤器。示例 \ref{cn-filter}
%   演示了如何定义条件,将条件组合成过滤器以及使用过滤器。
% \end{option}
%
% \iffalse
%<*verb>
% \fi
\DeleteShortVerb{\|}
\begin{example}(filter){使用过滤器筛选数据}
  \dbNewDatabase{filter-db}{name=str, count=int}
  \begin{dbFilters}{filter-db}
    \dbNewCond {cond1}{count}{\dbval > 3}
    \dbNewCond*{cond2}{name} {\d+}
    \dbCombCond{filter-and}{cond1 && cond2}
    \dbCombCond{filter-or} {cond1 || cond2}
  \end{dbFilters}
  \dbitemkv{filter-db}{name=123, count=4}
  \dbitemkv{filter-db}{name=ab3, count=2}
  \dbitemkv{filter-db}{name=bag, count=5}
  \dbNewStyle{filter-and-style}{filter-db}{
    filter      = filter-and,
    before-code = \par Filter And\par,
    item-code   = {\dbuse{name}: \dbuse{count}\quad},
  }
  \dbNewStyle{filter-or-style}{filter-db}{
    filter      = filter-or,
    before-code = \par Filter Or\par,
    item-code   = {\dbuse{name}: \dbuse{count}\quad},
  }
  \dbshow{filter-and-style}{filter-db}
  \dbshow{filter-or-style} {filter-db}
\end{example}
\MakeShortVerb{\|}
% \iffalse
%</verb>
% \fi
%
% \changes{1.1}{2022-01-06}{Add option}{\opt{raw-filter}}
% \begin{option}[added=2022-01-06]{raw-filter}
%   \begin{syntax}
%     \opt{raw-filter} = <conditional expression> \inidef
%   \end{syntax}
%
%   使用条件表达式设置匿名过滤器,这里的条件指通过 \cs{dbNewConditional} 定义的
%   条件。示例 \ref{cn-raw-filter} 使用 \opt{raw-filter} 选项,直接组合筛选条
%   件,达到与示例 \ref{cn-filter} 相同的效果。
% \end{option}
%
% \DeleteShortVerb{\|}
% \iffalse
%<*verb>
% \fi
\begin{example}(raw-filter){使用匿名过滤器筛选数据}
  \dbNewDatabase{filter-db}{name=str, count=int}
  \begin{dbFilters}{filter-db}
    \dbNewCond {cond1}{count}{\dbval > 3}
    \dbNewCond*{cond2}{name} {\d+}
  \end{dbFilters}
  \dbitemkv{filter-db}{name=123, count=4}
  \dbitemkv{filter-db}{name=ab3, count=2}
  \dbitemkv{filter-db}{name=bag, count=5}
  \dbNewStyle{filter-and-style}{filter-db}{
    raw-filter  = {cond1 && cond2},
    before-code = \par Filter And\par,
    item-code   = {\dbuse{name}: \dbuse{count}\quad},
  }
  \dbNewStyle{filter-or-style}{filter-db}{
    raw-filter  = {cond1 || cond2},
    before-code = \par Filter Or\par,
    item-code   = {\dbuse{name}: \dbuse{count}\quad},
  }
  \dbshow{filter-and-style}{filter-db}
  \dbshow{filter-or-style} {filter-db}
\end{example}
% \iffalse
%</verb>
% \fi
% \MakeShortVerb{\|}
%
% \changes{1.2}{2022-01-08}{Fix bug}{string sorting bug}
% \begin{option}[added=2022-01-05]{sort}
%   \begin{syntax}
%     \opt{sort} = \{ <attr spec1>, <attr spec2>, \ldots{} \} \inidef
%   \end{syntax}
%
%   为当前样式设置排序规则。支持对 |str|,|date|,|int|,|fp| 类型的数据进行排
%   序,支持多级排序。\meta{attr} 表示增序,\meta{attr}\texttt{*} 表示降序。示
%   例 \ref{cn-sort} 的排序规则为先按 |count| 降序排序,|count| 相同的再按
%   |name| 增序排序。
% \end{option}
%
% \iffalse
%<*verb>
% \fi
\begin{example}(sort){多级排序}
  \dbNewDatabase{sort-db}{name=str, count=int}
  \dbNewStyle{sort-style}{sort-db}{
    sort = {count*, name},
    item-code = {\dbuse{name}: \dbuse{count}\quad}
  }
  \dbitemkv{sort-db}{name=bag, count=1}
  \dbitemkv{sort-db}{name=box, count=1}
  \dbitemkv{sort-db}{name=tag, count=2}
  \dbitemkv{sort-db}{name=pen, count=3}
  \dbshow{sort-style}{sort-db}
\end{example}
% \iffalse
%</verb>
% \fi
%
% \begin{option}[added=2022-01-05, rEXP]{before-code}
%   \begin{syntax}
%     \opt{before-code} = <before code> \inidef*
%   \end{syntax}
%
%   该选项用来设置在展示整个数据库之\zhbefore 需要执行的代码(见示例
%   \ref{cn-db-wrap})。
% \end{option}
%
% \begin{option}[added=2022-01-05, rEXP]{after-code}
%   \begin{syntax}
%     \opt{after-code} = <after code> \inidef*
%   \end{syntax}
%
%   该选项用来设置在展示整个数据库之\zhafter 需要执行的代码(见示例
%   \ref{cn-db-wrap})。
% \end{option}
%
% \iffalse
%<*verb>
% \fi
\begin{example}(db-wrap){设置展示数据库前后的代码}
  \dbNewDatabase{wrap-db}{text=tl}
  \dbNewStyle{wrap-style}{wrap-db}{
    before-code = \textit{before code}\quad,
    after-code  = \textit{after code},
    item-code   = \dbarabic.~\dbuse{text}\quad
  }
  \dbitemkv{wrap-db}{text=text1}
  \dbitemkv{wrap-db}{text=text2}
  \dbitemkv{wrap-db}{text=text3}
  \dbshow{wrap-style}{wrap-db}
\end{example}
% \iffalse
%</verb>
% \fi
%
% \begin{option}[added=2022-01-05, rEXP]{item-code}
%   \begin{syntax}
%     \opt{item-code} = <item code> \inidef(\#1)
%   \end{syntax}
%
%   该选项用来设置展示数据库中每条记录的代码。你可以使用 \cs{dbuse}\marg{attr}
%   来指代属性 \meta{attr} 的值。示例 \ref{cn-item-code} 演示了如何展示一个首
%   字母缩写词表。
% \end{option}
%
% \iffalse
%<*verb>
% \fi
\begin{example}(item-code){展示数据库条目}
  \dbNewDatabase{item-db}{acronym=str, desc=tl}
  \dbNewStyle{item-style}{item-db}{
    before-code = {\dbIfEmptyF{\begin{description}}},
    after-code  = {\dbIfEmptyF{\end{description}}},
    item-code   = {\item[\dbuse{acronym}] \dbuse{desc}},
    sort        = acronym,
  }
  \dbitemkv{item-db}{acronym=PM,  desc={Prime Minister}}
  \dbitemkv{item-db}{acronym=CBD, desc={Central Business District}}
  \dbitemkv{item-db}{acronym=DL,  desc={Deep Learning}}
  \dbshow{item-style}{item-db}
\end{example}
% \iffalse
%</verb>
% \fi
%
% \changes{1.2}{2022-01-08}{Add options}{\opt{record-before-code},
% \opt{record-after-code}}
% \changes{1.5}{2022-01-14}{Update options}{Rename \opt{record-before-code}
% and \opt{record-after-code} to \opt{item-before-code} and
% \opt{item-after-code}}
% \begin{option}[added=2022-01-08, updated=2022-01-14, rEXP]{item-before-code}
%   \begin{syntax}
%     \opt{item-before-code} = <before code> \inidef*
%   \end{syntax}
%
%   在 \meta{item code} 之\zhbefore 执行的代码(见示例 \ref{cn-item-wrapper})。
% \end{option}
%
% \begin{option}[added=2022-01-08, updated=2022-01-14, rEXP]{item-after-code}
%   \begin{syntax}
%     \opt{item-after-code} = <after code> \inidef*
%   \end{syntax}
%
%   在 \meta{item code} 之\zhafter 执行的代码(见示例 \ref{cn-item-wrapper})。
% \end{option}
%
% \iffalse
%<*verb>
% \fi
\begin{example}(item-wrapper){设置展示条目之前和之后的代码}
  \dbNewDatabase{item-wrap-db}{text=tl, hint=tl}
  \dbNewStyle{item-wrap-style}{item-wrap-db}{
    item-before-code = \begingroup\ttfamily<,
    item-after-code  = >\endgroup,
    item-code        = \dbuse{text}~(\dbuse{hint}),
  }
  \dbitemkv{item-wrap-db}{text=example, hint={this is an example}}
  \dbshow{item-wrap-style}{item-wrap-db}
\end{example}
% \iffalse
%</verb>
% \fi
%
% \subsubsection{属性选项}
%
% \changes{1.5}{2022-01-14}{Remove option}{\opt{<attr>/wrapper}}
%
% \changes{1.5}{2022-01-14}{Add options}{\opt{<attr>/code},
% \opt{<attr>/code*}}
% \begin{option}[added=2022-01-14, rEXP]{<attr>/code}
%   \begin{syntax}
%     \nopt{<attr>/code} = <code> \inidef(\#1)
%   \end{syntax}
%
%     设置 \meta{attr} 的样式代码。在 \meta{code} 中用 |#1| 指代属性的值。示例
%     \ref{cn-attr-code} 将数量超过10个的物品打印为红色,少于10个的则打印为青色。
% \end{option}
%
% \iffalse
%<*verb>
% \fi
\begin{example}(attr-code){设置单个属性样式}
  \dbNewDatabase{attr-code-db}{name=str, count=int}
  \begin{dbFilters}{attr-code-db}
    \dbNewCond{large}{count}{\dbval >= 10}
  \end{dbFilters}
  \dbNewStyle{base-style}{attr-code-db}{
    item-code  = \dbuse{name}:~\dbuse{count}\quad,
  }
  \dbNewStyle[base-style]{large-style}{attr-code-db}{
    raw-filter = large,
    name/code  = \textcolor{red}{#1},
  }
  \dbNewStyle[base-style]{small-style}{attr-code-db}{
    raw-filter = !large,
    name/code  = \textcolor{teal}{#1},
  }
  \dbitemkv{attr-code-db}{name=bag, count=1}
  \dbitemkv{attr-code-db}{name=pen, count=12}
  \dbitemkv{attr-code-db}{name=pig, count=5}
  \dbitemkv{attr-code-db}{name=egg, count=50}
  \dbshow{large-style}{attr-code-db}
  \dbshow{small-style}{attr-code-db}
\end{example}
% \iffalse
%</verb>
% \fi
%
% \begin{option}[added=2022-01-14, rEXP]{<attr>/code*}
%   \begin{syntax}
%     \nopt{<attr>/code*} = <code> \inidef
%   \end{syntax}
%
%     设置 \meta{attr} 的样式代码。在 \meta{code} 中用 |#1| 指代\textbf{展开}的
%     属性的值。这对某些需要以特定格式解析参数的命令比较有用。
% \end{option}
%
%     示例 \ref{cn-exp-code} 中 \pkg{zhnumber} 宏包的 \cs{zhdate} 命令接收
%     |yyyy/mm/dd| 格式的时期并输出中文日期,本宏包默认的日期输出格式是
%     |yyyy/mm/dd|,因此可以通过 \opt{date/code*} 选项,将日期完全展开后然后传递
%     给 \cs{zhdate} 命令,如果不展开,\cs{zhdate} 接收到的是若干个用来展示日期
%     的控制序列,而不是 |yyyy/mm/dd| 格式的时期,进而触发报错。
%
% \iffalse
%<*verb>
% \fi
\MakePercentComment
\begin{example}(exp-code){中文日期}
  % \usepackage{zhnumber}
  \dbNewDatabase{exp-db}{date=date, event=tl}
  \dbNewStyle{exp-style}{exp-db}{
    item-code  = \par\makebox[4cm][l]{\dbuse{date}}\dbuse{event},
    date/code* = \zhdate{#1},
  }
  \dbitemkv{exp-db}{date=2020/12/31, event=eat}
  \dbitemkv{exp-db}{date=2021/01/01, event=sleep}
  \dbshow{exp-style}{exp-db}
\end{example}
\MakePercentIgnore
% \iffalse
%</verb>
% \fi
%
% \begin{option}[added=2022-01-05, rEXP]{<attr>/before-code}
%   \begin{syntax}
%     \nopt{<attr>/before-code} = <before code> \inidef*
%   \end{syntax}
%
%   该选项用来设置展示数据库中属性 \meta{attr} 对应数据之\zhbefore 需要执行的代
%   码。\cs{dbuse} 会在展示属性数据\zhbefore 执行此代码。
% \end{option}
%
% \begin{option}[added=2022-01-05, rEXP]{<attr>/after-code}
%   \begin{syntax}
%     \nopt{<attr>/after-code} = <after code> \inidef*
%   \end{syntax}
%
%   该选项用来设置展示数据库中属性 \meta{attr} 对应数据之\zhafter 需要执行的代码。
%   \cs{dbuse} 会在展示属性数据\zhafter 执行此代码。
% \end{option}
%
% 属性样式代码的执行顺序为:
% \begin{enumerate}[nolistsep]
%   \item \opt{<attr>/before-code}
%   \item \opt{<attr>/code} or \opt{<attr>/code*}
%   \item \opt{<attr>/after-code}
% \end{enumerate}
%
% \changes{1.5}{2022-01-16}{Add options}{\opt{<attr>/item-code},
% \opt{<attr>/item-code*}}
% \begin{option}[added=2022-01-16, rEXP]{<attr>/item-code}
%   \begin{syntax}
%     \nopt{<attr>/item-code} = <item code> \inidef(\#1)
%   \end{syntax}
%
%   设置列表元素的样式代码。在 \meta{item code} 中用 |#1| 指代列表元素的值。示例
%   \ref{cn-clist-code} 演示了如何为列表元素设置额外的样式。
% \end{option}
%
% \iffalse
%<*verb>
% \fi
\begin{example}(clist-code){设置列表元素样式代码}
  \dbNewDatabase{clist-db}{name=str, label=clist}
  \dbNewStyle{clist-style}{clist-db}{
    item-code       = \par\dbuse{name}:~\dbuse{label},
    label/item-code = (\textcolor{red}{\textit{#1}}),
  }
  \dbitemkv{clist-db}{name=pig,  label={animal, meat}}
  \dbitemkv{clist-db}{name=Alex, label={person, male}}
  \dbshow{clist-style}{clist-db}
\end{example}
% \iffalse
%</verb>
% \fi
%
% \begin{option}[added=2022-01-16, rEXP]{<attr>/item-code*}
%   \begin{syntax}
%     \nopt{<attr>/item-code*} = <item code> \inidef
%   \end{syntax}
%
%   设置列表元素的样式代码。在 \meta{item code} 中用 |#1| 指代\textbf{展开}的列
%   表元素的值。
% \end{option}
%
% \begin{option}[added=2022-01-05, rEXP]{<attr>/item-before-code}
%   \begin{syntax}
%     \nopt{<attr>/item-before-code} = <code> \inidef*
%   \end{syntax}
%
%   该选项只适用于类型为 |clist| 的属性,用来设置展示列表每个元素\zhbefore 需要
%   执行的代码。
% \end{option}
%
% \begin{option}[added=2022-01-05, rEXP]{<attr>/item-after-code}
%   \begin{syntax}
%     \nopt{<attr>/item-after-code} = <code> \inidef*
%   \end{syntax}
%
%   该选项只适用于类型为 |clist| 的属性,用来设置展示列表每个元素\zhafter 需要执
%   行的代码。
% \end{option}
%
% 列表元素样式代码的执行顺序为:
% \begin{enumerate}[nolistsep]
%   \item \opt{<attr>/item-before-code}
%   \item \opt{<attr>/item-code} or \opt{<attr>/item-code*}
%   \item \opt{<attr>/item-after-code}
% \end{enumerate}
%
% \def\sepini{%
%   \begingroup\normalfont%
%   \hfill%
%   (initially \texttt{\{,\char`~\}} for \texttt{clist}%
%   and \texttt{/} for \texttt{date}, no default)%
%   \endgroup%
% }
% \changes{1.3}{2022-01-09}{Update option}{\opt{<attr>/sep}}
% \begin{option}[added=2022-01-05, updated=2022-01-08, rEXP]{<attr>/sep}
%   \begin{syntax}
%     \nopt{<attr>/sep} = <separator> \sepini\\
%     \nopt{<attr>/sep} = \{ \\
%     ~~\meta{separator between two}, \\
%     ~~\meta{separator between more than two}, \\
%     ~~\meta{separator between final two} \\
%     \} \\
%     \nopt{<attr>/sep} = \{ \\
%     ~~\meta{separator before year}, \\
%     ~~\meta{separator between year and month}, \\
%     ~~\meta{separator between month and day}, \\
%     ~~\meta{separator after day} \\
%     \} \\
%   \end{syntax}
%
%   该选项只适用于类型为 |clist| 或 |date| 的属性,用来设置列表间元素的间隔。参
%   数为一个 \meta{separator} 时,所有元素间的分隔符被设置为 \meta{separator}。
%   \meta{separator before year} 和 \meta{separator after day} 被设置为空。
% \end{option}
%
%   参数为3个元素的逗号分隔的列表时,此选项用来设置列表元素的分隔符,分别用来设
%   置只有两个元素时的分隔符 \meta{separator between two},超过两个元素时的分隔
%   符 \meta{separator between more than two},和最后两个元素之间的分隔符
%   \meta{separator between final two}。对于类型为 |clist| 的属性,设置此选项时
%   如果参数列表数量不是1或者3会触发报错。示例 \ref{cn-clist-sep} 展示了如何设
%   置列表元素间隔。
%
% \iffalse
%<*verb>
% \fi
\begin{example}(clist-sep){设置列表元素间隔}
  \dbNewDatabase{clist-db}{label=clist}
  \dbNewStyle{clist-base}{clist-db}{
    before-code = {\dbIfEmptyF{\begin{enumerate}}},
    after-code  = {\dbIfEmptyF{\end{enumerate}}},
    item-code   = \item \dbuse{label},
  }
  \dbNewStyle[clist-base]{clist-style1}{clist-db}{
    label/sep = {{,~}}
  }
  \dbNewStyle[clist-base]{clist-style2}{clist-db}{
    label/sep = {{,~}, {,~}, ~and~}
  }
  \dbitemkv{clist-db}{label={a, b, c}}
  \dbitemkv{clist-db}{label={1, 2, 3}}
  \dbshow{clist-style1}{clist-db}
  \dbshow{clist-style2}{clist-db}
\end{example}
% \iffalse
%</verb>
% \fi
%
%   参数为4个元素的逗号分隔的列表时,此选项用来设置日期的分隔符,分别用来设
%   置 \meta{year} 之前的分隔符 \meta{separator before year} ,\meta{year} 和
%   \meta{month} 之间的分隔符 \meta{separator between year and month} ,
%   \meta{month} 和 \meta{day} 之间的分隔符,以及 \meta{day} 之后的分隔符。对于
%   类型为 |date| 的属性,设置此选项时如果参数列表数量不是1或者4会触发报错。示例
%   \ref{cn-date-sep} 展示了如何自定义日期间隔符。
%
% \iffalse
%<*verb>
% \fi
\begin{example}(date-sep){设置日期间隔符}
  \dbNewDatabase{date-db}{date=date}
  \dbNewStyle{date-style1}{date-db}{
    item-code = \dbuse{date}\quad,
    date/sep  = -,
  }
  \dbNewStyle{date-style2}{date-db}{
    item-code = \dbuse{date}\quad,
    date/sep  = {\$, +, !, \$},
  }
  \dbitemkv{date-db}{date=2020/01/02}
  \dbitemkv{date-db}{date=2022/07/12}
  \dbshow{date-style1}{date-db}
  \dbshow{date-style2}{date-db}
\end{example}
% \iffalse
%</verb>
% \fi
%
% \changes{1.5}{2022-01-14}{Add option}{\opt{<attr>/format-code}}
% \begin{option}[added=2022-01-14, rEXP]{<attr>/format-code}
%   \begin{syntax}
%     \nopt{<attr>/format-code} = <format code> \inidef
%   \end{syntax}
%
%   该选项用来更精细地控制日期的输出格式。在 \meta{format code}中,|#1| 代表年份,
%   |#2| 代表月份,|#3| 代表天。示例 \ref{cn-date-code} 演示了如何使用该选项。
% \end{option}
%
% \iffalse
%<*verb>
% \fi
\begin{example}(date-code){任意日期格式}
  \dbNewDatabase{date-db}{date=date}
  \dbNewStyle{date-style}{date-db}{
    item-code        = \dbuse{date},
    date/format-code = {日:#3\quad 月:#2\quad 年:#1}
  }
  \dbitemkv{date-db}{date=2022/01/01}
  \dbshow{date-style}{date-db}
\end{example}
% \iffalse
%</verb>
% \fi
%
% \changes{1.3}{2022-01-08}{Add option}{\opt{<attr>/zfill}}
% \begin{option}[added=2022-01-08, EXP]{<attr>/zfill}
%   \begin{syntax}
%     \nopt{<attr>/zfill} = <\TTF> \inidef(true)[true]
%   \end{syntax}
%
%   该选项只适用于类型为 |date| 的属性。控制输出月份和天时是否补零。示例
%   \ref{cn-date-zfill} 展示了补零和不补零的日期。
% \end{option}
%
% \iffalse
%<*verb>
% \fi
\begin{example}(date-zfill){月份和天补零}
  \dbNewDatabase{date-db}{date=date}
  \dbNewStyle {zfill-style}{date-db}{
    item-code = \dbuse{date},
  }
  \dbNewStyle{nofill-style}{date-db}{
    item-code  = \dbuse{date},
    date/zfill = false,
  }
  \dbitemkv{date-db}{date=2022/01/01}
  \dbshow {zfill-style}{date-db}
  \dbshow{nofill-style}{date-db}
\end{example}
% \iffalse
%</verb>
% \fi
%
% \changes{1.4}{2022-01-13}{Add macro}{\cs{dbdatesep}}
% \begin{function}[added=2022-01-13]{\dbdatesep}
%   \begin{syntax}
%     \cs{dbdatesep} \marg{separator}
%   \end{syntax}
%
%   设置内部解析日期时的分隔符,默认为 |/|,即存储数据的格式为 |yyyy/mm/dd|。
%   示例 \ref{cn-inner-date-sep} 演示了使用两种格式存储数据,但实际上分隔符并
%   没有被存储,而是被内部用来解析年、月和日然后存储为三个整数。
% \end{function}
%
% \iffalse
%<*verb>
% \fi
\begin{example}(inner-date-sep){设置日期解析格式}
  \dbNewDatabase{inner-date-db}{date=date}
  \dbNewStyle{inner-date-style}{inner-date-db}{
    item-code = \dbuse{date}\quad,
  }
  \dbitemkv{inner-date-db}{date=2020/01/20}
  \dbdatesep{-}
  \dbitemkv{inner-date-db}{date=2022-01-10}
  \dbshow{inner-date-style}{inner-date-db}
\end{example}
\dbdatesep{/}
% \iffalse
%</verb>
% \fi
% \subsection{过滤器}
%
% 过滤器是一些条件的组合,只有满足过滤器指定条件的数据才会被展现出来。
%
% \begin{function}[added=2022-01-05]{\dbNewReviewPoints}
%   \begin{syntax}
%     \cs{dbNewReviewPoints} \marg{name} \marg{points}
%   \end{syntax}
%
%   定义名为 \meta{name} 的复习间隔列表。\meta{points} 是一系列整数或整数表达
%   式,用于设置日期的过滤器。示例 \ref{cn-review-points} 中,预定义了一个复习
%   间隔列表 |review| 和一个时间锚点 |2022/02/06|,筛选时,将当前条目的日期与
%   描点相比较,计算得到时间间隔 $\meta{interval} = \meta{date anchor} -
%   \meta{date cmp}$,只有当 \meta{interval} 的值在 \meta{points} 中时,条件才
%   成立。
% \end{function}
%
% \iffalse
%<*verb>
% \fi
\DeleteShortVerb{\|}
\begin{example}(review-points){按时间间隔筛选}
  \dbNewDatabase{filter-db}{date=date}
  \dbNewReviewPoints{review}{2, 5}
  \dbNewRawFilter*{review-filter}{filter-db}{date}{review|2022/02/06}
  \dbNewStyle{filter-style}{filter-db}{
    item-code = \dbuse{date}\quad,
    filter    = review-filter,
  }
  \dbitemkv{filter-db}{date=2022/01/30}
  \dbitemkv{filter-db}{date=2022/02/01}
  \dbitemkv{filter-db}{date=2022/02/04}
  \dbshow{filter-style}{filter-db}
\end{example}
\MakeShortVerb{\|}
% \iffalse
%</verb>
% \fi
%
% \changes{1.5}{2022-01-16}{Update env}{add starred version of \env{dbFilters}}
% \begin{environment}[added=2022-01-05, updated=2022-01-16]{dbFilters}
%   \begin{syntax}
%     |\begin|\{\env{dbFilters}\}   \marg{database} \\
%     ~~\meta{code}
%     |\end|\{\env{dbFilters}\} \\
%     |\begin|\{\env{dbFilters}\} * \marg{database} \\
%     ~~\meta{code}
%     |\end|\{\env{dbFilters}\} \\
%   \end{syntax}
%
%   \env{dbFilters}用来定义过滤器,此环境中定义了 \cs{dbNewConditional} 命令用来
%   定义条件和 \cs{dbCombineConditionals} 命令用来组合条件定义过滤器。星号版本在
%   定义条件的同时会定义一个与条件同名且只使用这个条件的过滤器。示例
%   \ref{cn-star-filter} 中定义 |greater| 条件的同时也定义了一个同名的过滤器,你
%   可以直接在 \opt{filter} 中使用这个过滤器。过滤器独立于每个 \meta{database},
%   这意味着你可以在不同数据库中定义名称相同的过滤条件和过滤器。
% \end{environment}
%
% \iffalse
%<*verb>
% \fi
\begin{example}(star-filter){定义与条件同名的过滤器}
  \dbNewDatabase{filter-db}{count=int}
  \begin{dbFilters}*{filter-db}
    \dbNewCond{greater}{count}{\dbval > 3}
  \end{dbFilters}
  \dbNewStyle{filter-style}{filter-db}{
    filter    = greater,
    item-code = \dbuse{count}\quad,
  }
  \dbitemkv{filter-db}{count=2}
  \dbitemkv{filter-db}{count=5}
  \dbshow{filter-style}{filter-db}
\end{example}
% \iffalse
%</verb>
% \fi
%
% \changes{1.5}{2022-01-15}{Add macros}{\cs{dbNewCond}, \cs{dbCombCond}}
% \begin{function}[added=2022-01-05, updated=2022-01-16]{\dbNewConditional,
% \dbNewCond, \dbNewConditional*, \dbNewCond*}
%   \begin{syntax}
%     \cs{dbNewConditional}  \marg{name}        \marg{attr} \marg{cond spec}  \oarg{filter info}
%     \cs{dbNewConditional}* \marg{name}        \marg{attr} \marg{cond spec}  \oarg{filter info} \\[2pt]
%     \cs{dbNewConditional}  \marg{name} \marg{int/fp attr} \marg{expr}       \oarg{filter info}
%     \cs{dbNewConditional}* \marg{name} \marg{int/fp attr} \marg{expr}       \oarg{filter info}
%     \cs{dbNewConditional}  \marg{name} \marg{str/tl attr} \marg{regex expr} \oarg{filter info}
%     \cs{dbNewConditional}* \marg{name} \marg{str/tl attr} \marg{regex expr} \oarg{filter info}
%     \cs{dbNewConditional}  \marg{name}  \marg{clist attr} \marg{val list}   \oarg{filter info}
%     \cs{dbNewConditional}* \marg{name}  \marg{clist attr} \marg{val list}   \oarg{filter info}
%     \cs{dbNewConditional}  \marg{name}   \marg{date attr} \marg{date expr}  \oarg{filter info}
%     \cs{dbNewConditional}* \marg{name}   \marg{date attr} \{\meta{review points}\orbar\meta{date}\} \oarg{filter info}
%   \end{syntax}
%
%   \cs{dbNewConditional} 用来定义名为 \meta{name} 的条件,\meta{attr} 指定条件
%   所绑定的属性,在 \meta{cond spec} 中可以用 \cs{dbval} 指代属性的值。
%   \cs{dbNewCond} 是 \cs{dbNewConditional} 的别名。
% \end{function}
%
%   \changes{1.3}{2022-01-10}{Update doc}{truncated division}
%   对于类型为 |int| 和 |fp| 的属性,\meta{expr} 传递给 \cs{int_compare:nTF} 或
%   \cs{fp_compare:nTF} 处理。
%   \begin{note}
%     |/| 为四舍五入除法,截断除法请用 \cs{dbIntDivTruncate}。
%   \end{note}
%
%   对于类型为 |str| 和 |tl| 的属性,\meta{regex} 为正则表达式,
%   \cs{dbNewConditional} 表示部分匹配,\cs{dbNewConditional*} 表示整体匹配。该
%   选项依赖于 \pkg{l3regex}。示例 \ref{cn-filter-str} 演示了部分匹配和全部匹
%   配的区别,|part| 过滤器匹配所有含数字的 |name|,而 |all| 过滤器匹配全部为
%   数字的 |name|。
%
% \iffalse
%<*verb>
% \fi
\begin{example}(filter-str){匹配字符串}
  \dbNewDatabase{filter-db}{name=str}
  \begin{dbFilters}*{filter-db}
    \dbNewCond{part}{name}{\d+}
    \dbNewCond*{all}{name}{\d+}
  \end{dbFilters}
  \dbNewStyle{part-style}{filter-db}{
    before-code = Match part:~,
    item-code   = \dbuse{name}\quad,
    filter      = part,
  }
  \dbNewStyle{all-style}{filter-db}{
    before-code = Match all:~,
    item-code   = \dbuse{name}\quad,
    filter      = all,
  }
  \dbitemkv{filter-db}{name=123}
  \dbitemkv{filter-db}{name=int12}
  \dbitemkv{filter-db}{name=variable}
  \dbshow{part-style}{filter-db}
  \dbshow {all-style}{filter-db}
\end{example}
% \iffalse
%</verb>
% \fi
%
%   对于类型为 |clist| 的属性,使用 \cs{dbNewConditional} 定义的条件只要
%   \meta{val list} 中的任意一个元素在属性值(列表)中则条件成立;使用
%   \cs{dbNewConditional*} 定义的条件只有 \meta{val list} 中每一个值都在属性值
%   (列表)中条件才成立。示例 \ref{cn-filter-clist} 中过滤器 |or| 匹配含有
%   hard \textbf{或者} red 的标签,而过滤器 |and| 匹配含有 hard \textbf{并且}
%   含有 red 的标签。
%
% \iffalse
%<*verb>
% \fi
\begin{example}(filter-clist){筛选列表}
  \dbNewDatabase{filter-db}{label=clist}
  \begin{dbFilters}*{filter-db}
    \dbNewCond  {or}{label}{hard, red}
    \dbNewCond*{and}{label}{hard, red}
  \end{dbFilters}
  \def\emph#1{\textit{\textbf{#1}}}
  \dbNewStyle{base-style}{filter-db}{
    before-code = {
      \begin{minipage}[t]{.3\textwidth}
        All items
    },
    after-code  = {\end{minipage}},
    item-code   = \par\dbarabic.~\dbuse{label},
  }
  \dbNewStyle[base-style] {or-style}{filter-db}{
    before-code = {
      \begin{minipage}[t]{.3\textwidth}
        Match \emph{any} of hard \emph{or} red
    },
    filter      = or,
  }
  \dbNewStyle[base-style]{and-style}{filter-db}{
    before-code = {
      \begin{minipage}[t]{.3\textwidth}
        Match \emph{all} of hard \emph{and} red
    },
    filter      = and,
  }
  \dbitemkv{filter-db}{label={hard, red}}
  \dbitemkv{filter-db}{label={hard, blue}}
  \dbitemkv{filter-db}{label={easy, blue}}
  \dbitemkv{filter-db}{label={easy, red}}
  \dbitemkv{filter-db}{label={hard, red, flat}}
  \dbshow {base-style}{filter-db}
  \dbshow   {or-style}{filter-db}
  \dbshow  {and-style}{filter-db}
\end{example}
% \iffalse
%</verb>
% \fi
%
%   \changes{1.3}{2022-01-08}{Update logic}{swap definition of starred and
%   unstarred conditionals of date}
%   对于类型为 |date| 的属性,\cs{dbNewConditional} 定义的条件后续处理中会将
%   \meta{expr} 中的所有日期转换成相对1971年1月1日的一个整数值,然后将处理后的
%   表达式传递给 \cs{int_compare:nTF} 做进一步处理。示例 \ref{cn-filter-date}
%   展示了如何使用该选项。
%
% \iffalse
%<*verb>
% \fi
\begin{example}(filter-date){根据日期表达式过滤}
  \dbNewDatabase{filter-db}{date=date}
  \dbNewRawFilter{date-filter}{filter-db}{date}{\dbval >= 2022/02/01}
  \dbNewStyle{filter-style}{filter-db}{
    item-code = \dbuse{date}\quad,
    filter    = date-filter,
  }
  \dbitemkv{filter-db}{date=2022/01/30}
  \dbitemkv{filter-db}{date=2022/02/01}
  \dbitemkv{filter-db}{date=2022/02/04}
  \dbshow{filter-style}{filter-db}
\end{example}
% \iffalse
%</verb>
% \fi
%
%   对于类型为 |date| 的属性,\cs{dbNewConditional*} 使用复习点来定义过滤条件,
%   \meta{review points} 是 \cs{dbNewReviewPoints} 定义的复习点,\meta{date}
%   是用来比较的日期(见示例 \ref{cn-review-points})。
%
% \changes{1.5}{2022-01-16}{Add macro}{\cs{dbNewRawFilter}}
% \begin{function}[added=2022-01-16]{\dbNewRawFilter}
%   \begin{syntax}
%     \cs{dbNewRawFilter}  \marg{name} \marg{database} \marg{attr} \marg{cond spec} \oarg{filter info}
%     \cs{dbNewRawFilter}* \marg{name} \marg{database} \marg{attr} \marg{cond spec} \oarg{filter info}
%     等同于
%     |\begin|\{\env{dbFilters}\}*\phantom{\marg{name}}\marg{database}
%     ~~\cs{dbNewCond}     \marg{name} \phantom{\marg{database}} \marg{attr} \marg{cond spec} \oarg{filter info}
%     ~~\cs{dbNewCond}*    \marg{name} \phantom{\marg{database}} \marg{attr} \marg{cond spec} \oarg{filter info}
%     |\end|\{\env{dbFilters}\}
%   \end{syntax}
%
%   该命令用来快捷地定义单个过滤器。示例 \ref{cn-new-raw-filter} 展示了如何使
%   用该命令,它和示例 \ref{cn-star-filter} 本质上是相同的。
% \end{function}
%
% \iffalse
%<*verb>
% \fi
\begin{example}(new-raw-filter){定义与条件同名的过滤器}
  \dbNewDatabase{filter-db}{count=int}
  \dbNewRawFilter{greater}{filter-db}{count}{\dbval > 3}
  \dbNewStyle{filter-style}{filter-db}{
    filter    = greater,
    item-code = \dbuse{count}\quad,
  }
  \dbitemkv{filter-db}{count=2}
  \dbitemkv{filter-db}{count=5}
  \dbshow{filter-style}{filter-db}
\end{example}
% \iffalse
%</verb>
% \fi
%
% \begin{function}[added=2022-01-05]{\dbCombineConditionals}
%   \begin{syntax}
%     \cs{dbCombineConditionals} \marg{name} \marg{cond combination} \oarg{info}
%   \end{syntax}
%
%   \cs{dbCombineConditionals} 定义名为 \marg{name} 的过滤器,并将
%   \cs{dbNewConditional} 定义的条件组合起来。\meta{cond combination} 中可以使
%   用的关系操作符为 \verb=&&, ||, !=。可以将 \opt{filter} 选项设置为
%   \meta{name} 来应用过滤器。\meta{info} 为过滤器的相关信息,在展示数据库的时
%   候可以用 \cs{dbFilterInfo} 指代。使用示例见示例 \ref{cn-filter}.
% \end{function}
%
% \subsection{存储和使用数据}
%
% \changes{1.4}{2022-01-13}{Update env}{dbitem}
% \begin{environment}[added=2022-01-05, updated=2022-01-13]{dbitem}
%   \begin{syntax}
%     |\begin|\marg{\env{dbitem}} \marg{database} \oarg{attr-val list}
%     ~~\meta{code} \\
%     |\end|\marg{\env{dbitem}}
%   \end{syntax}
%
%   \env{dbitem} 环境用来存储数据。有两种存储数据的方法,较短的数据可以在选项列
%   表中通过键值对设置值,较长的数据可以在 \meta{code} 中使用 \cs{dbsave} 存储。
%   \meta{attr} = \meta{val} 等同于 \cs{dbsave}\marg{attr}\marg{val},
%   \meta{attr}\texttt{*} = \meta{val} 等同于 \cs{dbsave*}\marg{attr}
%   \marg{val},数据在 |e| 或者 |x| 类型的参数中不可展开。\cs{dbsave}会覆盖选
%   项中设置的值。没有设置的值将会被设置为全局默认值。示例 \ref{cn-dbitem} 展
%   示了如何存储数据。
% \end{environment}
%
% \iffalse
%<*verb>
% \fi
\begin{example}(dbitem){存储数据}
  \dbNewDatabase{ques-db}{date=date, ques=tl, ans=tl}
  \dbNewStyle{ques-style}{ques-db}{
    item-code       = {%
      \par\dbuse{date}
      \par\dbarabic.~\dbuse{ques}
      \par\textbf{答案:}~\dbuse{ans}
    },
    item-after-code = \medskip,
    date/code*      = \zhdate{#1},
  }
  \begin{dbitem}{ques-db}[date=2022/01/01]
    \dbsave{ques}{地球到月亮的距离}
    \dbsave{ans} {384,401公里}
  \end{dbitem}
  \begin{dbitem}{ques-db}[date=2022/01/02]
    \dbsave{ques}{鲁迅的本名}
    \dbsave{ans} {周树人}
  \end{dbitem}
  \dbshow{ques-style}{ques-db}
\end{example}
% \iffalse
%</verb>
% \fi
%
% \changes{1.4}{2022-01-13}{Add macro}{\cs{dbitemkv}}
% \begin{function}[added=2022-01-13]{\dbitemkv}
%   \begin{syntax}
%     \cs{dbitemkv} \marg{database} \oarg{attr-val list}
%   \end{syntax}
%
%   只使用 \meta{attr-val list} 来存储数据。
% \end{function}
%
% \changes{1.3}{2022-01-08}{Add macro}{\cs{dbsave*}}
% \begin{function}[added=2022-01-05, updated=2022-01-08]{\dbsave, \dbsave*}
%   \begin{syntax}
%     \cs{dbsave}  \marg{attr} \marg{data} \\
%     \cs{dbsave}* \marg{attr} \marg{data}
%   \end{syntax}
%
%   \cs{dbsave} 用来存储数据,只能在 \env{item} 环境中使用。使用 \cs{dbsave*} 存
%   储的数据会被 \cs{exp_not:n} 包裹。
% \end{function}
%
% \changes{1.2}{2022-01-08}{Update macro}{make \cs{dbuse} fully-expandable}
% \begin{function}[added=2022-01-05, updated=2022-01-08, EXP]{\dbuse}
%   \begin{syntax}
%     \cs{dbuse} \marg{attr}
%   \end{syntax}
%
%   \cs{dbuse} 用来展示数据,只能在 \opt{item-code}, \opt{item-before-code},
%   \opt{item-after-code} 选项中使用。\cs{dbuse} 是可展开的。
% \end{function}
%
% \subsection{条件判别式}
%
% \begin{function}[added=2022-01-05, EXP]{\dbIfEmptyT, \dbIfEmptyF, \dbIfEmptyTF}
%   \begin{syntax}
%     \cs{dbIfEmptyTF} \marg{true code} \marg{false code} \\
%     \cs{dbIfEmptyT}  \marg{true code} \\
%     \cs{dbIfEmptyF} \marg{false code}
%   \end{syntax}
%
%   该判别式用来判断当前数据库是否为空。示例 \ref{cn-empty-db} 演示了如何使用
%   该判别式来预防空的列表环境。
% \end{function}
%
% \iffalse
%<*verb>
% \fi
\begin{example}(empty-db){预防空列表环境}
  \dbNewDatabase{test-db}{text=tl}
  \dbNewRawFilter{alph}{test-db}{text}{\d}
  \dbNewStyle{base-style}{test-db}{
    before-code = \dbIfEmptyTF{Empty db}{\begin{enumerate}},
    after-code  = \dbIfEmptyF{\end{enumerate}}\medskip,
    item-code   = \item \dbuse{text},
  }
  \dbNewStyle[base-style]{empty-style}{test-db}{
    raw-filter=!alph
  }
  \dbitemkv{test-db}{text={$1 + 1 = 2$.}}
  \dbitemkv{test-db}{text={I have 2 pens.}}
  \dbshow {base-style}{test-db}
  \dbshow{empty-style}{test-db}
\end{example}
% \iffalse
%</verb>
% \fi
%
% \changes{1.2}{2022-01-08}{Remove macros}{\cs{dbItemIfEmpty(TF)}, \cs{dbClistItemIfEmpty(TF)}}
%
% \subsection{表达式函数}
%
% \changes{1.3}{2022-01-10}{Add macros}{expression function aliases}
% \begin{function}[added=2022-01-10, EXP]{
%   \dbIntAbs, \dbIntSign, \dbIntDivRound, \dbIntDivTruncate, \dbIntMax,
%   \dbIntMin, \dbIntMod, \dbFpSign,
% }
%   \begin{syntax}
%     \cs{dbIntAbs}         \Arg{intexpr}
%     \cs{dbIntSign}        \Arg{intexpr}
%     \cs{dbIntDivRound}    \Arg{intexpr_1} \Arg{intexpr_2}
%     \cs{dbIntDivTruncate} \Arg{intexpr_1} \Arg{intexpr_2}
%     \cs{dbIntMax}         \Arg{intexpr_1} \Arg{intexpr_2}
%     \cs{dbIntMin}         \Arg{intexpr_1} \Arg{intexpr_2}
%     \cs{dbIntMod}         \Arg{intexpr_1} \Arg{intexpr_2}
%     \cs{dbFpSign}         \Arg{fpexpr}
%   \end{syntax}
% \begin{tblr}{ll}
%   \cs{dbIntAbs}         & 等同于 \cs{int_abs:n} \\
%   \cs{dbIntSign}        & 等同于 \cs{int_sign:n} \\
%   \cs{dbIntDivRound}    & 等同于 \cs{int_div_round:nn} \\
%   \cs{dbIntDivTruncate} & 等同于 \cs{int_div_truncate:nn} \\
%   \cs{dbIntMax}         & 等同于 \cs{int_max:nn} \\
%   \cs{dbIntMin}         & 等同于 \cs{int_min:nn} \\
%   \cs{dbIntMod}         & 等同于 \cs{int_mod:nn} \\
%   \cs{dbFpSign}         & 等同于 \cs{fp_sign:n} \\
% \end{tblr}
%
%   详细的文档见 \pkg{interface3}。示例 \ref{cn-expr-db} 展示了如何筛选3的倍数。
% \end{function}
%
% \iffalse
%<*verb>
% \fi
\begin{example}(expr-db){筛选3的倍数}
  \dbNewDatabase{expr-db}{n=int}
  \dbNewRawFilter{mod}{expr-db}{n}{\dbIntMod{\dbval}{3} = 0}
  \dbNewStyle{expr-style}{expr-db}{
    item-code = \dbuse{n}\quad,
    filter    = mod,
  }
  \dbitemkv{expr-db}{n=2}
  \dbitemkv{expr-db}{n=3}
  \dbitemkv{expr-db}{n=6}
  \dbitemkv{expr-db}{n=7}
  \dbshow{expr-style}{expr-db}
\end{example}
% \iffalse
%</verb>
% \fi
%
% \subsection{特殊命令}
%
% \pkg{dbshow} 定义了一些特殊的命令,会根据语境展开为不同的内容。
%
% \changes{1.1}{2022-01-05}{Add macro}{
%   \cs{dbarabic}, \cs{dbalph}, \cs{dbAlph}, \cs{dbroman},
%   \cs{dbRoman}
% }
% \changes{1.1}{2022-01-06}{Fix bug}{\cs{dbIndex} not defined}
% \begin{function}[added=2022-01-05, EXP]{
%   \dbval, \dbtoday, \dbDatabase, \dbFilterName, \dbFilterInfo,
%   \dbIndex, \dbarabic, \dbalph, \dbAlph, \dbroman, \dbRoman
% }
% \begin{tblr}{ll}
%   \cs{dbval}        & 当前属性的值 \\
%   \cs{dbtoday}      & 当天的日期 \\
%   \cs{dbDatabase}   & 数据库名称 \\
%   \cs{dbFilterName} & 当前样式过滤器的名称 \\
%   \cs{dbFilterInfo} & 当前样式过滤器的相关信息 \\
%   \cs{dbIndex}      & 数据索引,等同于 \cs{dbuse}\marg{id} \\
%   \cs{dbarabic}     & 用数字表示的查询集数据计数 \\
%   \cs{dbalph}       & 用小写字母表示的查询集数据计数 \\
%   \cs{dbAlph}       & 用大写字母表示的查询集数据计数 \\
%   \cs{dbroman}      & 用小写罗马字母表示的查询集数据计数 \\
%   \cs{dbroman}      & 用大写罗马字母表示的查询集数据计数 \\
% \end{tblr}
%
%   \cs{dbtoday} 使用 \cs{dbdatesep} 确定的分隔符来展示日期。
% \end{function}
%
% \changes{1.1}{2022-01-07}{Update doc}{improve example}
% \changes{1.5}{2022-01-17}{Remove doc}{Remove big example}
%
% \title{
%   Package \pkg{dbshow} \fileversion%
%   \protect\footnote{%
%     Repository: \url{https://github.com/ZhiyuanLck/dbshow},
%     Telegram Group: \url{https://t.me/latex_dbshow}}
%   \rlap{\makebox[4cm][r]{
%     \normalsize $\Longrightarrow$ \color{red}
%     \linktarget{zh}{en}{中文版本}
%   }}
% }
% \author{Changkai Li \protect\path{<lichangkai225@qq.com>}}
% \date{\filedate}
% \maketitle
%
% \section{Introduction}
%
% The initial motivation to write this package is that I want to write a
% template, which can collect questions you gave the wrong answer and can
% display those questions you would like to review by some conditionals, such as
% questions with certain label, questions you have answered uncorrectly for
% certain times or questions having not been reviewed for certain days. So this
% package provides a database to do such thing.
%
% The package provides four core functions: data storage and display, data
% filtering, data sorting and data display. All data is saved once and then you
% can display these data with custom filters, orders and styles.
%
% \changes{1.4}{2022-01-10}{Add check}{version of \pkg{l3kernel}}
% \pkg{dbshow} depends on \pkg{l3kernel} with version date after |2022-11-07|.
%
% \changes{1.3}{2022-01-09}{Add doc}{descripton for expansion}
% \begin{itemize}
%   \item Macros with a $\star$ are fully-expandable;
%   \item Macros with a \ding{73} are restricted-expandable;
%   \item Macros without appending a special symbol are nonexpandable;
%   \item Options with a $\star$ \textit{do not} affect the expandability of
%     related macros;
%   \item Options with a \ding{73} affect the expandability of related macros
%     according to its value;
%   \item Options without appending a special symbol make the related macros
%     nonexpandable.
% \end{itemize}
%
% \subsection{Data Types}
%
% The package constructs 6 types based on the internal typed of \pkg{expl3}:
% \begin{Description}[\texttt{clist}]
%   \item[\texttt{date}]
%     date saved in |yyyy/mm/dd| format, supports comparison, sorting
%     (converting to string), default \cs{dbtoday}.
%   \item[\texttt{str}]
%     string,supports regex match and sorting, default empty.
%   \item[\texttt{tl}]
%     token list, supports regex match, default empty.
%   \item[\texttt{int}]
%     integer, supports comparison and sorting, default 0.
%   \item[\texttt{fp}]
%     floating point, supports comparison and sorting, default 0.
%   \item[\texttt{clist}]
%     comma list, default empty.
% \end{Description}
%
% \changes{1.3}{2022-01-08}{Remove dependency}{\pkg{datatime2}}
% All types are internal types of \pkg{expl3} except |date| type, which provides
% by \pkg{dbshow} itself and supports converting to integer and printing with
% style.
%
% \subsection{Comparison to \pkg{datatool}}
%
% \changes{1.2}{2022-01-07}{Add doc}{add comparison to \pkg{datatool}}
%
% \pkg{dbshow} and \pkg{datatool} implement the same core functions. But
% \pkg{dbshow} is based on \pkg{expl3} and it supports string regex and
% multi-level sorting. \pkg{dbshow} tries to divide style from the contents
% (data in database): all styles are predefined and can be reused conveniently
% so that there can be only codes to save data and one-line code to show the
% database inside the \env{document} environment. You can hide the details in
% the preamble and focus on the data you want to display. \pkg{dbshow} provides
% a simple temporary runtime database, which means it can not input and output
% data from/to extern files (they should be responsible by some professional
% programming languages). When you need to delete or revise a record, just go to
% where it is recorded in the source code rather than use a macro to manipulate
% data after they are saved. In a sense, \TeX~file is also a kind of data
% persistence.
%
% \section{Interfaces}
%
% \subsection{Create, Display and Clear Database}
%
% \changes{1.4}{2022-01-10}{Update macro}{change \orbar\ to \cs{dbshow_sep} in
% weird argument of \cs{dbshow_process_default_value:w}}
% \begin{function}[added=2022-01-05, updated=2022-01-10]{\dbNewDatabase, \dbNewDatabase*}
%   \begin{syntax}
%     \cs{dbNewDatabase} \oarg{base database} \marg{database} \{ \\
%     ~~\meta{attr1} = \meta{type spec1}, \\
%     ~~\meta{attr2} = \meta{type spec2}, \\
%     ~~\ldots{} \\
%     \} \\
%     \cs{dbNewDatabase}* \marg{database} \{ \\
%     ~~\meta{attr1} = \meta{type spec1}, \\
%     ~~\meta{attr2} = \meta{type spec2}, \\
%     ~~\ldots{} \\
%     \} \\
%   \end{syntax}
%
% \end{function}
%
% Create a new database named \meta{database}, unstarred form provides the optional
% \meta{base database} from which current database inherit the attributes settings.
% The unstarred form always replace the old definition, while starred form
% appends the new options.
%
% \begin{syntax}
%   \meta{attr} = \meta{type} \\
%   \meta{attr} = \meta{type}\orbar\meta{default}
% \end{syntax}
%
% The first form defines the \meta{attr} as \meta{type}, and the second also
% sets the default value.
%
% \begin{note}
%   Every database has a default attribute |id| to store the index of the item.
% \end{note}
%
% The example below define a database named |ques|.
% \begin{verbatim}
%   \dbNewDatabase{ques}{
%     question = tl, % store question
%     answer = tl,   % store corresponding answer
%     date = date,   % store the date when you were wrong
%     info = tl,     % store extra info
%     labels = clist % store question labels
%   }
% \end{verbatim}
%
% \begin{function}[added=2022-01-05]{\dbshow}
%   \begin{syntax}
%     \cs{dbshow} \marg{style} \marg{database}
%   \end{syntax}
%
%   Show the \meta{database} with \meta{style}.
% \end{function}
%
% \changes{1.2}{2022-01-07}{Add macro}{\cs{dbclear}}
% \begin{function}[added=2022-01-07]{\dbclear}
%   \begin{syntax}
%     \cs{dbclear} \marg{database}
%   \end{syntax}
%
%   Clear the content of \meta{database}.
% \end{function}
%
% \subsection{\cs{dbNewStyle} and Style Options}
%
% \begin{function}[added=2022-01-05, updated=2022-01-15]{\dbNewStyle}
%   \begin{syntax}
%     \cs{dbNewStyle} \oarg{base styles} \marg{style} \marg{database} \marg{opts}
%   \end{syntax}
%
%   Define a new \meta{style} that binds to \meta{database}. The style can
%   inherit from a list of \meta{base styles} such as
%   |\dbNewStyle[base1, base2]{new-style}{ques}{}|.
% \end{function}
%
% \subsubsection{General Options}
%
% \begin{option}[added=2022-01-05]{filter}
%   \begin{syntax}
%     \opt{filter} = <filter> \inidef(-none-)
%   \end{syntax}
%
%   Set the \meta{filter} defined by \cs{dbCombineFilters}. Example
%   \ref{en-filter} shows how to define conditionals and combine them into
%   a filter.
% \end{option}
%
% \DeleteShortVerb{\|}
% \iffalse
%<*verb>
% \fi
\begin{example*}(filter){Filter Items}
  \dbNewDatabase{filter-db}{name=str, count=int}
  \begin{dbFilters}{filter-db}
    \dbNewCond {cond1}{count}{\dbval > 3}
    \dbNewCond*{cond2}{name} {\d+}
    \dbCombCond{filter-and}{cond1 && cond2}
    \dbCombCond{filter-or} {cond1 || cond2}
  \end{dbFilters}
  \dbitemkv{filter-db}{name=123, count=4}
  \dbitemkv{filter-db}{name=ab3, count=2}
  \dbitemkv{filter-db}{name=bag, count=5}
  \dbNewStyle{filter-and-style}{filter-db}{
    filter      = filter-and,
    before-code = \par Filter And\par,
    item-code   = {\dbuse{name}: \dbuse{count}\quad},
  }
  \dbNewStyle{filter-or-style}{filter-db}{
    filter      = filter-or,
    before-code = \par Filter Or\par,
    item-code   = {\dbuse{name}: \dbuse{count}\quad},
  }
  \dbshow{filter-and-style}{filter-db}
  \dbshow{filter-or-style} {filter-db}
\end{example*}
% \iffalse
%</verb>
% \fi
% \MakeShortVerb{\|}
%
% \changes{1.1}{2022-01-06}{Add option}{\opt{raw-filter}}
% \begin{option}[added=2022-01-06]{raw-filter}
%   \begin{syntax}
%     \opt{raw-filter} = <conditional expression> \inidef
%   \end{syntax}
%
%   Set anonymous filter with conditionals defined by \cs{dbNewConditional}.
%   Example \ref{en-raw-filter} shows how to simplify the code of example
%   \ref{en-filter} with \opt{raw-filter} option.
% \end{option}
%
% \DeleteShortVerb{\|}
% \iffalse
%<*verb>
% \fi
\begin{example*}(raw-filter){Using Anonymous Filter}
  \dbNewDatabase{filter-db}{name=str, count=int}
  \begin{dbFilters}{filter-db}
    \dbNewCond {cond1}{count}{\dbval > 3}
    \dbNewCond*{cond2}{name} {\d+}
  \end{dbFilters}
  \dbitemkv{filter-db}{name=123, count=4}
  \dbitemkv{filter-db}{name=ab3, count=2}
  \dbitemkv{filter-db}{name=bag, count=5}
  \dbNewStyle{filter-and-style}{filter-db}{
    raw-filter  = {cond1 && cond2},
    before-code = \par Filter And\par,
    item-code   = {\dbuse{name}: \dbuse{count}\quad},
  }
  \dbNewStyle{filter-or-style}{filter-db}{
    raw-filter  = {cond1 || cond2},
    before-code = \par Filter Or\par,
    item-code   = {\dbuse{name}: \dbuse{count}\quad},
  }
  \dbshow{filter-and-style}{filter-db}
  \dbshow{filter-or-style} {filter-db}
\end{example*}
% \iffalse
%</verb>
% \fi
% \MakeShortVerb{\|}
%
% \changes{1.2}{2022-01-08}{Fix bug}{string sorting bug}
% \begin{option}[added=2022-01-05]{sort}
%   \begin{syntax}
%     \opt{sort} = \{ <attr spec1>, <attr spec2>, \ldots{} \} \inidef
%   \end{syntax}
%
%   Set sorting rules. Attributes of type |str, date, int, fp| is supported to
%   sort. Multi-level sort is allowed. \meta{attr} represents for ascending
%   order, and \meta{attr}\texttt{*} represents for descending order. Example
%   \ref{en-sort} shows how to sort items by |count| in descending order and
%   for the same |count|, sort by |name| in ascending order.
% \end{option}
%
% \iffalse
%<*verb>
% \fi
\begin{example*}(sort){Multi-level Sorting}
  \dbNewDatabase{sort-db}{name=str, count=int}
  \dbNewStyle{sort-style}{sort-db}{
    sort = {count*, name},
    item-code = {\dbuse{name}: \dbuse{count}\quad}
  }
  \dbitemkv{sort-db}{name=bag, count=1}
  \dbitemkv{sort-db}{name=box, count=1}
  \dbitemkv{sort-db}{name=tag, count=2}
  \dbitemkv{sort-db}{name=pen, count=3}
  \dbshow{sort-style}{sort-db}
\end{example*}
% \iffalse
%</verb>
% \fi
%
% \begin{option}[added=2022-01-05, rEXP]{before-code}
%   \begin{syntax}
%     \opt{before-code} = <before code> \inidef*
%   \end{syntax}
%
%   Set the \meta{code} that is executed \enbefore displaying the database.
%   See Example \ref{en-db-wrap}.
% \end{option}
%
% \begin{option}[added=2022-01-05, rEXP]{after-code}
%   \begin{syntax}
%     \opt{after-code} = <after code> \inidef*
%   \end{syntax}
%
%   Set the \meta{code} that is executed \enafter displaying the database.
%   See Example \ref{en-db-wrap}.
% \end{option}
%
% \iffalse
%<*verb>
% \fi
\begin{example*}(db-wrap){Set the Before and After Code of the Database}
  \dbNewDatabase{wrap-db}{text=tl}
  \dbNewStyle{wrap-style}{wrap-db}{
    before-code = \textit{before code}\quad,
    after-code  = \textit{after code},
    item-code   = \dbarabic.~\dbuse{text}\quad
  }
  \dbitemkv{wrap-db}{text=text1}
  \dbitemkv{wrap-db}{text=text2}
  \dbitemkv{wrap-db}{text=text3}
  \dbshow{wrap-style}{wrap-db}
\end{example*}
% \iffalse
%</verb>
% \fi
%
% \begin{option}[added=2022-01-05, rEXP]{item-code}
%   \begin{syntax}
%     \opt{item-code} = <item code> \inidef(\#1)
%   \end{syntax}
%
%   Set the code that display a record. You can use \cs{dbuse} to denote the
%   value of attribute. Example \ref{en-item-code} shows how to display an
%   acronym glossary table.
% \end{option}
%
% \iffalse
%<*verb>
% \fi
\begin{example*}(item-code){Display Database Items}
  \dbNewDatabase{item-db}{acronym=str, desc=tl}
  \dbNewStyle{item-style}{item-db}{
    before-code = {\dbIfEmptyF{\begin{description}}},
    after-code  = {\dbIfEmptyF{\end{description}}},
    item-code   = {\item[\dbuse{acronym}] \dbuse{desc}},
    sort        = acronym,
  }
  \dbitemkv{item-db}{acronym=PM,  desc={Prime Minister}}
  \dbitemkv{item-db}{acronym=CBD, desc={Central Business District}}
  \dbitemkv{item-db}{acronym=DL,  desc={Deep Learning}}
  \dbshow{item-style}{item-db}
\end{example*}
% \iffalse
%</verb>
% \fi
%
% \changes{1.2}{2022-01-08}{Add options}{\opt{record-before-code},
% \opt{record-after-code}}
% \changes{1.5}{2022-01-14}{Update options}{Rename \opt{record-before-code}
% and \opt{record-after-code} to \opt{item-before-code} and
% \opt{item-after-code}}
% \begin{option}[added=2022-01-08, updated=2022-01-14, rEXP]{item-before-code}
%   \begin{syntax}
%     \opt{item-before-code} = <before code> \inidef*
%   \end{syntax}
%
%   Set the \meta{code} that is executed \enbefore displaying a item. See
%   Example \ref{en-item-wrapper}.
% \end{option}
%
% \begin{option}[added=2022-01-08, updated=2022-01-14, rEXP]{item-after-code}
%   \begin{syntax}
%     \opt{item-after-code} = <after code> \inidef*
%   \end{syntax}
%
%   Set the \meta{code} that is executed \enafter displaying the item. See
%   Example \ref{en-item-wrapper}.
% \end{option}
%
% \iffalse
%<*verb>
% \fi
\begin{example*}(item-wrapper){Set the Before and After Code of the Item}
  \dbNewDatabase{item-wrap-db}{text=tl}
  \dbNewStyle{item-wrap-style}{item-wrap-db}{
    item-before-code = \begingroup\ttfamily<,
    item-after-code  = >\endgroup,
    item-code        = \dbuse{text},
  }
  \dbitemkv{item-wrap-db}{text=example}
  \dbshow{item-wrap-style}{item-wrap-db}
\end{example*}
% \iffalse
%</verb>
% \fi
%
% \subsubsection{Attribute Options}
%
% \changes{1.5}{2022-01-14}{Remove option}{\opt{<attr>/wrapper}}
%
% \changes{1.5}{2022-01-14}{Add options}{\opt{<attr>/code},
% \opt{<attr>/code*}}
% \begin{option}[added=2022-01-14, rEXP]{<attr>/code}
%   \begin{syntax}
%     \nopt{<attr>/code} = <code> \inidef(\#1)
%   \end{syntax}
%
%   Set the style code of \meta{attr}. In \meta{code}, |#1| is replaced with the
%   value of \meta{attr}. Example \ref{en-attr-code} prints |name| whose |count|
%   is less than 10 in teal color, otherwise in red color.
% \end{option}
%
% \iffalse
%<*verb>
% \fi
\begin{example*}(attr-code){Set Style Code of Attribute}
  \dbNewDatabase{attr-code-db}{name=str, count=int}
  \begin{dbFilters}{attr-code-db}
    \dbNewCond{large}{count}{\dbval >= 10}
  \end{dbFilters}
  \dbNewStyle{base-style}{attr-code-db}{
    item-code  = \dbuse{name}:~\dbuse{count}\quad,
  }
  \dbNewStyle[base-style]{large-style}{attr-code-db}{
    raw-filter = large,
    name/code  = \textcolor{red}{#1},
  }
  \dbNewStyle[base-style]{small-style}{attr-code-db}{
    raw-filter = !large,
    name/code  = \textcolor{teal}{#1},
  }
  \dbitemkv{attr-code-db}{name=bag, count=1}
  \dbitemkv{attr-code-db}{name=pen, count=12}
  \dbitemkv{attr-code-db}{name=pig, count=5}
  \dbitemkv{attr-code-db}{name=egg, count=50}
  \dbshow{large-style}{attr-code-db}
  \dbshow{small-style}{attr-code-db}
\end{example*}
% \iffalse
%</verb>
% \fi
%
% \begin{option}[added=2022-01-14, rEXP]{<attr>/code*}
%   \begin{syntax}
%     \nopt{<attr>/code*} = <code> \inidef
%   \end{syntax}
%
%   Set the style code of \meta{attr}. In \meta{code}, |#1| is replaced with the
%   \textbf{expanded} value of \meta{attr}. This is useful to commands that
%   require special format of the arguments.
% \end{option}
%
%   Example \ref{en-exp-code} shows how to print Chinese date with \cs{zhdate}
%   of package \pkg{zhnumber}. \cs{zhdate} require argument to be the date of
%   |yyyy/mm/dd|, which is the default date format of \pkg{dbshow}. But we need
%   to expand the date through \opt{date/code*} first because \cs{zhdate} cannot
%   parse the token list other than |yyyy/mm/dd|.
%
% \iffalse
%<*verb>
% \fi
\MakePercentComment
\begin{example*}(exp-code){Show Chinese Date}
  % \usepackage{zhnumber}
  \dbNewDatabase{exp-db}{date=date, event=tl}
  \dbNewStyle{exp-style}{exp-db}{
    item-code  = \par\makebox[4cm][l]{\dbuse{date}}\dbuse{event},
    date/code* = \zhdate{#1},
  }
  \dbitemkv{exp-db}{date=2020/12/31, event=eat}
  \dbitemkv{exp-db}{date=2021/01/01, event=sleep}
  \dbshow{exp-style}{exp-db}
\end{example*}
\MakePercentIgnore
% \iffalse
%</verb>
% \fi
%
% \begin{option}[added=2022-01-05, rEXP]{<attr>/before-code}
%   \begin{syntax}
%     \nopt{<attr>/before-code} = <before code> \inidef*
%   \end{syntax}
%
%   Set the \meta{code} that is executed by \cs{dbuse} \enbefore displaying
%   the value of attribute.
% \end{option}
%
% \begin{option}[added=2022-01-05, rEXP]{<attr>/after-code}
%   \begin{syntax}
%     \nopt{<attr>/after-code} = <after code> \inidef*
%   \end{syntax}
%
%   Set the \meta{code} that is executed by \cs{dbuse} \enafter displaying
%   the value of attribute.
% \end{option}
%
% The style code execution order of attribute is:
% \begin{enumerate}[nolistsep]
%   \item \opt{<attr>/before-code}
%   \item \opt{<attr>/code} or \opt{<attr>/code*}
%   \item \opt{<attr>/after-code}
% \end{enumerate}
%
% \changes{1.5}{2022-01-16}{Add options}{\opt{<attr>/item-code},
% \opt{<attr>/item-code*}}
% \begin{option}[added=2022-01-16, rEXP]{<attr>/item-code}
%   \begin{syntax}
%     \nopt{<attr>/item-code} = <item code> \inidef(\#1)
%   \end{syntax}
%
%   Set the style code of \meta{attr}. In \meta{item code}, |#1| is replaced
%   with the item of the comma list. Example \ref{en-clist-code} shows how to
%   set the style code of the item of the comma list.
% \end{option}
%
% \iffalse
%<*verb>
% \fi
\begin{example*}(clist-code){Set the Style Code of Clist Item}
  \dbNewDatabase{clist-db}{name=str, label=clist}
  \dbNewStyle{clist-style}{clist-db}{
    item-code       = \par\dbuse{name}:~\dbuse{label},
    label/item-code = (\textcolor{red}{\textit{#1}}),
  }
  \dbitemkv{clist-db}{name=pig,  label={animal, meat}}
  \dbitemkv{clist-db}{name=Alex, label={person, male}}
  \dbshow{clist-style}{clist-db}
\end{example*}
% \iffalse
%</verb>
% \fi
%
% \begin{option}[added=2022-01-16, rEXP]{<attr>/item-code*}
%   \begin{syntax}
%     \nopt{<attr>/item-code*} = <item code> \inidef
%   \end{syntax}
%
%   Set the style code of \meta{attr}. In \meta{item code}, |#1| is replaced
%   with the \textbf{expanded} item of the comma list.
% \end{option}
%
% \begin{option}[added=2022-01-05, rEXP]{<attr>/item-before-code}
%   \begin{syntax}
%     \nopt{<attr>/item-before-code} = <before code> \inidef*
%   \end{syntax}
%
%   Only for attributes of type |clist|. Set the \meta{code} that is excuted
%   \enbefore displaying the item of the comma list.
% \end{option}
%
% \begin{option}[added=2022-01-05, rEXP]{<attr>/item-after-code}
%   \begin{syntax}
%     \nopt{<attr>/item-after-code} = <after code> \inidef*
%   \end{syntax}
%
%   Only for attributes of type |clist|. Set the \meta{code} that is excuted
%   \enafter displaying the item of the comma list.
% \end{option}
%
% The style code execution order of comma list item is:
% \begin{enumerate}[nolistsep]
%   \item \opt{<attr>/item-before-code}
%   \item \opt{<attr>/item-code} or \opt{<attr>/item-code*}
%   \item \opt{<attr>/item-after-code}
% \end{enumerate}
%
% \changes{1.3}{2022-01-09}{Update option}{\opt{<attr>/sep}}
% \begin{option}[added=2022-01-05, updated=2022-01-08, rEXP]{<attr>/sep}
%   \begin{syntax}
%     \nopt{<attr>/sep} = <separator> \sepini\\
%     \nopt{<attr>/sep} = \{ \\
%     ~~\meta{separator between two}, \\
%     ~~\meta{separator between more than two}, \\
%     ~~\meta{separator between final two} \\
%     \} \\
%     \nopt{<attr>/sep} = \{ \\
%     ~~\meta{separator before year}, \\
%     ~~\meta{separator between year and month}, \\
%     ~~\meta{separator between month and day}, \\
%     ~~\meta{separator after day} \\
%     \} \\
%   \end{syntax}
%
%   Only for attributes of type |clist| or |date|. Set the separator between
%   items. If the argument is an one-item comma list, all separators are set to
%   \meta{separator} but \meta{separator before year} and \meta{separator after
%   day} is set empty.
% \end{option}
%
%   If the argument is a comma list of 3 items, it is used to set the separator
%   between items of the comma list. Following documentation is quoted from
%   \pkg{interface3}:
%   \begin{quote}
%     If the comma list has more than two items, the \meta{separator between
%     more than two} is placed between each pair of items except the last, for
%     which the \meta{separator between final two} is used. If the comma list
%     has exactly two items, then they are placed in the input stream separated
%     by the \meta{separator between two}. If the comma list has a single item,
%     it is placed in the input stream, and a comma list with no items produces
%     no output.
%   \end{quote}
%   For attributes of type |clist|, incorrect number (numbers exclude 1 and 3)
%   of items of the argument will raise an error. Example \ref{en-clist-sep}
%   shows how to use this option to customize the separators between comma
%   list items.
%
% \iffalse
%<*verb>
% \fi
\begin{example*}(clist-sep){Set Separator Between Items of Comma List}
  \dbNewDatabase{clist-db}{label=clist}
  \dbNewStyle{clist-base}{clist-db}{
    before-code = {\dbIfEmptyF{\begin{enumerate}}},
    after-code  = {\dbIfEmptyF{\end{enumerate}}},
    item-code   = \item \dbuse{label},
  }
  \dbNewStyle[clist-base]{clist-style1}{clist-db}{
    label/sep = {{,~}}
  }
  \dbNewStyle[clist-base]{clist-style2}{clist-db}{
    label/sep = {{,~}, {,~}, ~and~}
  }
  \dbitemkv{clist-db}{label={a, b, c}}
  \dbitemkv{clist-db}{label={1, 2, 3}}
  \dbshow{clist-style1}{clist-db}
  \dbshow{clist-style2}{clist-db}
\end{example*}
% \iffalse
%</verb>
% \fi
%
%   If the argument is a comma list of 4 items, it is used to set the separators
%   of the date. For attributes of type |date|, incorrect number (numbers
%   exclude 1 and 4) will raise an error. Example \ref{en-date-sep} shows how
%   to customize the date separators.
%
% \iffalse
%<*verb>
% \fi
\begin{example*}(date-sep){Set Date Separators}
  \dbNewDatabase{date-db}{date=date}
  \dbNewStyle{date-style1}{date-db}{
    item-code = \dbuse{date}\quad,
    date/sep  = -,
  }
  \dbNewStyle{date-style2}{date-db}{
    item-code = \dbuse{date}\quad,
    date/sep  = {\$, +, !, \$},
  }
  \dbitemkv{date-db}{date=2020/01/02}
  \dbitemkv{date-db}{date=2022/07/12}
  \dbshow{date-style1}{date-db}
  \dbshow{date-style2}{date-db}
\end{example*}
% \iffalse
%</verb>
% \fi
%
% \changes{1.5}{2022-01-14}{Add option}{\opt{<attr>/format-code}}
% \begin{option}[added=2022-01-14, rEXP]{<attr>/format-code}
%   \begin{syntax}
%     \nopt{<attr>/format-code} = <format code> \inidef
%   \end{syntax}
%
%   Use this option to get fine-grained control over date formatting. In
%   \meta{format code}, |#1| represents for the \meta{year}, |#2| represents for
%   the \meta{month} and |#3| represents for the \meta{day}. Example
%   \ref{en-date-code} shows how to format the date with this option.
% \end{option}
%
% \iffalse
%<*verb>
% \fi
\begin{example*}(date-code){Date Formatting}
  \dbNewDatabase{date-db}{date=date}
  \dbNewStyle{date-style}{date-db}{
    item-code        = \dbuse{date},
    date/format-code = {日:#3\quad 月:#2\quad 年:#1}
  }
  \dbitemkv{date-db}{date=2022/01/01}
  \dbshow{date-style}{date-db}
\end{example*}
% \iffalse
%</verb>
% \fi
%
% \changes{1.3}{2022-01-08}{Add option}{\opt{<attr>/zfill}}
% \begin{option}[added=2022-01-08, EXP]{<attr>/zfill}
%   \begin{syntax}
%     \nopt{<attr>/zfill} = <\TTF> \inidef(true)[true]
%   \end{syntax}
%
%   Only for attributes of type |date|. Control whether to fill zero on the left
%   of the month or day. Example \ref{en-date-zfill} shows the differences.
% \end{option}
%
% \iffalse
%<*verb>
% \fi
\begin{example*}(date-zfill){Control the Leading Zero of the Date}
  \dbNewDatabase{date-db}{date=date}
  \dbNewStyle {zfill-style}{date-db}{
    item-code = \dbuse{date},
  }
  \dbNewStyle{nofill-style}{date-db}{
    item-code  = \dbuse{date},
    date/zfill = false,
  }
  \dbitemkv{date-db}{date=2022/01/01}
  \dbshow {zfill-style}{date-db}
  \dbshow{nofill-style}{date-db}
\end{example*}
% \iffalse
%</verb>
% \fi
%
% \changes{1.4}{2022-01-13}{Add macro}{\cs{dbdatesep}}
% \begin{function}[added=2022-01-13]{\dbdatesep}
%   \begin{syntax}
%     \cs{dbdatesep} \marg{separator}
%   \end{syntax}
%
%   Set the separator for internal date parsing. The default value is |/|,
%   i.e. the date must be store in the format of |yyyy/mm/dd|. Example
%   \ref{en-inner-date-sep} shows how to store dates with two formats.
%   However, the separators are not really stored but used to parse the year,
%   month and day, which will be stored internally as three integers.
% \end{function}
%
% \iffalse
%<*verb>
% \fi
\begin{example*}(inner-date-sep){Set Date Parsing Format}
  \dbNewDatabase{inner-date-db}{date=date}
  \dbNewStyle{inner-date-style}{inner-date-db}{
    item-code = \dbuse{date}\quad,
  }
  \dbitemkv{inner-date-db}{date=2020/01/20}
  \dbdatesep{-}
  \dbitemkv{inner-date-db}{date=2022-01-10}
  \dbshow{inner-date-style}{inner-date-db}
\end{example*}
% \iffalse
%</verb>
% \fi
%
% \subsection{Data Filters}
%
% Filter is a combination of conditionals that is used to filter the data you
% want to display.
%
% \begin{function}[added=2022-01-05]{\dbNewReviewPoints}
%   \begin{syntax}
%     \cs{dbNewReviewPoints} \marg{name} \marg{points}
%   \end{syntax}
%
%   Define \meta{points} to filter dates by intervals.  something. \meta{points}
%   is a list of \meta{intexpr}. In example \ref{en-review-points}, a \meta{date
%   anchor} and a list of \meta{points} is defined. During the filtering process,
%   \meta{interval} is calculated by $\meta{interval} = \meta{date anchor} -
%   \meta{date cmp}$, \meta{date cmp} is the \meta{date} of current item. Then
%   each \meta{intexpr} in \meta{points} is compared with \meta{interval} to
%   test if they are equal.
% \end{function}
%
% \iffalse
%<*verb>
% \fi
\DeleteShortVerb{\|}
\dbdatesep{/}
\begin{example*}(review-points){Filter with Review Points}
  \dbNewDatabase{filter-db}{date=date}
  \dbNewReviewPoints{review}{2, 5}
  \dbNewRawFilter*{review-filter}{filter-db}{date}{review|2022/02/06}
  \dbNewStyle{filter-style}{filter-db}{
    item-code = \dbuse{date}\quad,
    filter    = review-filter,
  }
  \dbitemkv{filter-db}{date=2022/01/30}
  \dbitemkv{filter-db}{date=2022/02/01}
  \dbitemkv{filter-db}{date=2022/02/04}
  \dbshow{filter-style}{filter-db}
\end{example*}
\MakeShortVerb{\|}
% \iffalse
%</verb>
% \fi
%
% \changes{1.5}{2022-01-16}{Update env}{add starred version of \env{dbFilters}}
% \begin{environment}[added=2022-01-05, updated=2022-01-16]{dbFilters}
%   \begin{syntax}
%     |\begin|\{\env{dbFilters}\}   \marg{database} \\
%     ~~\meta{code}
%     |\end|\{\env{dbFilters}\} \\
%     |\begin|\{\env{dbFilters}\} * \marg{database} \\
%     ~~\meta{code}
%     |\end|\{\env{dbFilters}\} \\
%   \end{syntax}
%
%   Filters are defined inside \env{dbFilters} environment, inside which,
%   \cs{dbNewConditional} is defined to declare conditionals and
%   \cs{dbCombineConditionals} is defined to combine conditionals. Starred
%   version will define a filter of the same name of conditional as soon as you
%   define it. In example \ref{en-star-filter}, line 3 define the conditional
%   and the filter named |greater| so that you can use it in option \opt{option}.
%   Filters are independent in different databases, which means the same name of
%   filters is allowed in different databases.
% \end{environment}
%
% \iffalse
%<*verb>
% \fi
\begin{example*}(star-filter){Define Conditional and Filter at Same Time}
  \dbNewDatabase{filter-db}{count=int}
  \begin{dbFilters}*{filter-db}
    \dbNewCond{greater}{count}{\dbval > 3}
  \end{dbFilters}
  \dbNewStyle{filter-style}{filter-db}{
    filter    = greater,
    item-code = \dbuse{count}\quad,
  }
  \dbitemkv{filter-db}{count=2}
  \dbitemkv{filter-db}{count=5}
  \dbshow{filter-style}{filter-db}
\end{example*}
% \iffalse
%</verb>
% \fi
%
% \changes{1.5}{2022-01-15}{Add macros}{\cs{dbNewCond}, \cs{dbCombCond}}
% \begin{function}[added=2022-01-05, updated=2022-01-16]{\dbNewConditional,
% \dbNewCond, \dbNewConditional*, \dbNewCond*}
%   \begin{syntax}
%     \cs{dbNewConditional}  \marg{name}        \marg{attr} \marg{cond spec}  \oarg{filter info}
%     \cs{dbNewConditional}* \marg{name}        \marg{attr} \marg{cond spec}  \oarg{filter info} \\[2pt]
%     \cs{dbNewConditional}  \marg{name} \marg{int/fp attr} \marg{expr}       \oarg{filter info}
%     \cs{dbNewConditional}* \marg{name} \marg{int/fp attr} \marg{expr}       \oarg{filter info}
%     \cs{dbNewConditional}  \marg{name} \marg{str/tl attr} \marg{regex expr} \oarg{filter info}
%     \cs{dbNewConditional}* \marg{name} \marg{str/tl attr} \marg{regex expr} \oarg{filter info}
%     \cs{dbNewConditional}  \marg{name}  \marg{clist attr} \marg{val list}   \oarg{filter info}
%     \cs{dbNewConditional}* \marg{name}  \marg{clist attr} \marg{val list}   \oarg{filter info}
%     \cs{dbNewConditional}  \marg{name}   \marg{date attr} \marg{date expr}  \oarg{filter info}
%     \cs{dbNewConditional}* \marg{name}   \marg{date attr} \{\meta{review points}\orbar\meta{date}\} \oarg{filter info}
%   \end{syntax}
%
%   Define the conditional named \meta{name} that binds to \meta{attr}. \cs{dbval}
%   is replaced with the real value of the attribute inside the \meta{cond spec}.
% \end{function}
%
%   \changes{1.3}{2022-01-10}{Update doc}{truncated division}
%   For attributes of type |int| and |fp|, \meta{expr} is passed to
%   \cs{int_compare:nTF} or \cs{fp_compare:nTF}.
%   \begin{note}
%     Division using |/| rounds to the closest integer. Use \cs{dbIntDivTruncate} to rounds
%     the result toward 0.
%   \end{note}
%
%   For attribute of type |str| and |tl|, unstarred form matches any part while
%   starred form matches the whole part with the \meta{regex expr}, which is
%   shown in example \ref{en-filter-str}: filter |part| match |name|s that
%   contain numbers and filter |all| match |name|s that is composed of digits.
%
% \iffalse
%<*verb>
% \fi
\begin{example*}(filter-str){Match Strings}
  \dbNewDatabase{filter-db}{name=str}
  \begin{dbFilters}*{filter-db}
    \dbNewCond{part}{name}{\d+}
    \dbNewCond*{all}{name}{\d+}
  \end{dbFilters}
  \dbNewStyle{part-style}{filter-db}{
    before-code = Match part:~,
    item-code   = \dbuse{name}\quad,
    filter      = part,
  }
  \dbNewStyle{all-style}{filter-db}{
    before-code = Match all:~,
    item-code   = \dbuse{name}\quad,
    filter      = all,
  }
  \dbitemkv{filter-db}{name=123}
  \dbitemkv{filter-db}{name=int12}
  \dbitemkv{filter-db}{name=variable}
  \dbshow{part-style}{filter-db}
  \dbshow {all-style}{filter-db}
\end{example*}
% \iffalse
%</verb>
% \fi
%
%   For attributes of type |clist|, the conditional defined by unstarred form is
%   true if any item of \meta{val list} is in the comma list. While the
%   conditional defined by starred form is true only if every item of \meta{val
%   list} is in the comma list. In example \ref{en-filter-clist}, filter |or|
%   match labels that contain hard \textbf{or} red, and filter |and| match
%   labels that contain hard \textbf{and} red.
%
% \iffalse
%<*verb>
% \fi
\begin{example*}(filter-clist){Filter Lists}
  \dbNewDatabase{filter-db}{label=clist}
  \begin{dbFilters}*{filter-db}
    \dbNewCond  {or}{label}{hard, red}
    \dbNewCond*{and}{label}{hard, red}
  \end{dbFilters}
  \def\emph#1{\textit{\textbf{#1}}}
  \dbNewStyle{base-style}{filter-db}{
    before-code = {
      \begin{minipage}[t]{.3\textwidth}
        All items
    },
    after-code  = {\end{minipage}},
    item-code   = \par\dbarabic.~\dbuse{label},
  }
  \dbNewStyle[base-style] {or-style}{filter-db}{
    before-code = {
      \begin{minipage}[t]{.3\textwidth}
        Match \emph{any} of hard \emph{or} red
    },
    filter      = or,
  }
  \dbNewStyle[base-style]{and-style}{filter-db}{
    before-code = {
      \begin{minipage}[t]{.3\textwidth}
        Match \emph{all} of hard \emph{and} red
    },
    filter      = and,
  }
  \dbitemkv{filter-db}{label={hard, red}}
  \dbitemkv{filter-db}{label={hard, blue}}
  \dbitemkv{filter-db}{label={easy, blue}}
  \dbitemkv{filter-db}{label={easy, red}}
  \dbitemkv{filter-db}{label={hard, red, flat}}
  \dbshow {base-style}{filter-db}
  \dbshow   {or-style}{filter-db}
  \dbshow  {and-style}{filter-db}
\end{example*}
% \iffalse
%</verb>
% \fi
%
%   \changes{1.3}{2022-01-08}{Update logic}{swap definition of starred and
%   unstarred conditionals of date}
%   For attributes of type |date|, unstarred form replace each date with a
%   integer representing for the days between \meta{date} and 1971/01/01, and
%   the result is passed to \cs{int_compare:nTF}. Example \ref{en-filter-date}
%   shows how to use the \meta{date expr} to filter the data.
%
% \iffalse
%<*verb>
% \fi
\begin{example*}(filter-date){Filter with Date Expression}
  \dbNewDatabase{filter-db}{date=date}
  \dbNewRawFilter{date-filter}{filter-db}{date}{\dbval >= 2022/02/01}
  \dbNewStyle{filter-style}{filter-db}{
    item-code = \dbuse{date}\quad,
    filter    = date-filter,
  }
  \dbitemkv{filter-db}{date=2022/01/30}
  \dbitemkv{filter-db}{date=2022/02/01}
  \dbitemkv{filter-db}{date=2022/02/04}
  \dbshow{filter-style}{filter-db}
\end{example*}
% \iffalse
%</verb>
% \fi
%
%   Starred form defines the conditional with review points defined by
%   \cs{dbNewReviewPoints} and \meta{date} is the date to be compared (see
%   example \ref{en-review-points}).
%
% \changes{1.5}{2022-01-16}{Add macro}{\cs{dbNewRawFilter}}
% \begin{function}[added=2022-01-16]{\dbNewRawFilter}
%   \begin{syntax}
%     \cs{dbNewRawFilter}  \marg{name} \marg{database} \marg{attr} \marg{cond spec} \oarg{filter info}
%     \cs{dbNewRawFilter}* \marg{name} \marg{database} \marg{attr} \marg{cond spec} \oarg{filter info}
%     \textnormal{\textit{Equal to}}
%     |\begin|\{\env{dbFilters}\}*\phantom{\marg{name}}\marg{database}
%     ~~\cs{dbNewCond}     \marg{name} \phantom{\marg{database}} \marg{attr} \marg{cond spec} \oarg{filter info}
%     ~~\cs{dbNewCond}*    \marg{name} \phantom{\marg{database}} \marg{attr} \marg{cond spec} \oarg{filter info}
%     |\end|\{\env{dbFilters}\}
%   \end{syntax}
%
%   Use this command to quickly define a \meta{filter} that has the same name
%   with the \meta{conditional}. Example \ref{en-new-raw-filter} is actually the
%   same with example \ref{en-star-filter}, but it is more concise.
% \end{function}
%
% \iffalse
%<*verb>
% \fi
\begin{example*}(new-raw-filter){Define Conditional and Filter at Same Time}
  \dbNewDatabase{filter-db}{count=int}
  \dbNewRawFilter{greater}{filter-db}{count}{\dbval > 3}
  \dbNewStyle{filter-style}{filter-db}{
    filter    = greater,
    item-code = \dbuse{count}\quad,
  }
  \dbitemkv{filter-db}{count=2}
  \dbitemkv{filter-db}{count=5}
  \dbshow{filter-style}{filter-db}
\end{example*}
% \iffalse
%</verb>
% \fi
%
% \begin{function}[added=2022-01-05]{\dbCombineConditionals}
%   \begin{syntax}
%     \cs{dbCombineConditionals} \marg{name} \marg{cond combination} \oarg{info}
%   \end{syntax}
%
%   Define the filter \meta{name}, which combine the conditionals and store
%   the extra \meta{info} into \cs{dbFilterInfo}.  Supported operators are
%   \verb=&&, ||, !=. You can set the option \opt{filter} to \meta{name} to
%   apply the filter when you display the database. See example
%   \ref{en-filter}.
% \end{function}
%
% \subsection{Store and Use Data}
%
% \changes{1.4}{2022-01-13}{Update env}{dbitem}
% \begin{environment}[added=2022-01-05, updated=2022-01-13]{dbitem}
%   \begin{syntax}
%     |\begin|\marg{\env{dbitem}} \marg{database} \oarg{attr-val list}
%     ~~\meta{code} \\
%     |\end|\marg{\env{dbitem}}
%   \end{syntax}
%
%   The data are stored with \env{dbitem} environment in two ways. Short data
%   can be stored in \meta{attr-val list} and long data can be stored by
%   \cs{dbsave}, which will suppress the value set by the option list.
%   \meta{attr} = \meta{val} is equal to \cs{dbsave}\marg{attr}\marg{val}, and
%   \meta{attr} = \meta{val} is equal to \cs{dbsave*}\marg{attr}\marg{val},
%   in which case, data will not be expanded in an |e| or |x|-type argument.
%   Example \ref{en-dbitem} shows how to store data with \env{dbitem}.
% \end{environment}
%
% \iffalse
%<*verb>
% \fi
\begin{example*}(dbitem){Store Date}
  \dbNewDatabase{ques-db}{date=date, ques=tl, ans=tl}
  \dbNewStyle{ques-style}{ques-db}{
    item-code       = {%
      \par\dbuse{date}
      \par\dbarabic.~\dbuse{ques}
      \par\textbf{Answer:}~\dbuse{ans}
    },
    item-after-code = \medskip,
  }
  \begin{dbitem}{ques-db}[date=2022/01/01]
    \dbsave{ques}{Distance from earth to moon?}
    \dbsave{ans} {384,401 kilometers.}
  \end{dbitem}
  \begin{dbitem}{ques-db}[date=2022/01/02]
    \dbsave{ques}{The number of English letters?}
    \dbsave{ans} {26.}
  \end{dbitem}
  \dbshow{ques-style}{ques-db}
\end{example*}
% \iffalse
%</verb>
% \fi
%
% \changes{1.4}{2022-01-13}{Add macro}{\cs{dbitemkv}}
% \begin{function}[added=2022-01-13]{\dbitemkv}
%   \begin{syntax}
%     \cs{dbitemkv} \marg{database} \oarg{attr-val list}
%   \end{syntax}
%
%   Store data with \meta{attr-val list}.
% \end{function}
%
% \subsection{\cs{dbsave} and \cs{dbuse}}
%
% \changes{1.3}{2022-01-08}{Add macro}{\cs{dbsave*}}
% \begin{function}[added=2022-01-05, updated=2022-01-08]{\dbsave, \dbsave*}
%   \begin{syntax}
%     \cs{dbsave}  \marg{attr} \marg{data} \\
%     \cs{dbsave}* \marg{attr} \marg{data}
%   \end{syntax}
%
%   \cs{dbsave} save the \meta{data} to \meta{attr} of current record.
%   \cs{dbsave} can be used only inside the \env{dbitem} environment.
%   \meta{data} stored by \cs{dbsave*} is wrapped with \cs{exp_not:n} while
%   \meta{data} stored by \cs{dbsave} keeps the same.
% \end{function}
%
% \changes{1.2}{2022-01-08}{Update macro}{make \cs{dbuse} fully-expandable}
% \begin{function}[added=2022-01-05, updated=2022-01-08, EXP]{\dbuse}
%   \begin{syntax}
%     \cs{dbuse} \marg{attr}
%   \end{syntax}
%
%   Display the value of \meta{attr} of current record. \cs{dbuse} is
%   \textbf{expandable} and can be only used inside the option \opt{item-code},
%   \opt{item-before-code}, \opt{item-after-code}.
% \end{function}
%
% \subsection{Conditionals}
%
% \begin{function}[added=2022-01-05, EXP]{\dbIfEmptyT, \dbIfEmptyF, \dbIfEmptyTF}
%   \begin{syntax}
%     \cs{dbIfEmptyTF} \marg{true code} \marg{false code} \\
%     \cs{dbIfEmptyT}  \marg{true code} \\
%     \cs{dbIfEmptyF} \marg{false code}
%   \end{syntax}
%
%   Test if the database is empty. Example \ref{en-empty-db} shows how to
%   avoid an empty list environment.
% \end{function}
%
% \iffalse
%<*verb>
% \fi
\begin{example*}(empty-db){Avoid Empty List Environment}
  \dbNewDatabase{test-db}{text=tl}
  \dbNewRawFilter{alph}{test-db}{text}{\d}
  \dbNewStyle{base-style}{test-db}{
    before-code = \dbIfEmptyTF{Empty db}{\begin{enumerate}},
    after-code  = \dbIfEmptyF{\end{enumerate}}\medskip,
    item-code   = \item \dbuse{text},
  }
  \dbNewStyle[base-style]{empty-style}{test-db}{
    raw-filter=!alph
  }
  \dbitemkv{test-db}{text={$1 + 1 = 2$.}}
  \dbitemkv{test-db}{text={I have 2 pens.}}
  \dbshow {base-style}{test-db}
  \dbshow{empty-style}{test-db}
\end{example*}
% \iffalse
%</verb>
% \fi
%
% \changes{1.2}{2022-01-08}{Remove macros}{\cs{dbItemIfEmpty(TF)}, \cs{dbClistItemIfEmpty(TF)}}
%
% \subsection{Expression Functions}
%
% \changes{1.3}{2022-01-10}{Add macros}{expression function aliases}
% \begin{function}[added=2022-01-10, EXP]{
%   \dbIntAbs, \dbIntSign, \dbIntDivRound, \dbIntDivTruncate, \dbIntMax,
%   \dbIntMin, \dbIntMod, \dbFpSign,
% }
%   \begin{syntax}
%     \cs{dbIntAbs}         \Arg{intexpr}
%     \cs{dbIntSign}        \Arg{intexpr}
%     \cs{dbIntDivRound}    \Arg{intexpr_1} \Arg{intexpr_2}
%     \cs{dbIntDivTruncate} \Arg{intexpr_1} \Arg{intexpr_2}
%     \cs{dbIntMax}         \Arg{intexpr_1} \Arg{intexpr_2}
%     \cs{dbIntMin}         \Arg{intexpr_1} \Arg{intexpr_2}
%     \cs{dbIntMod}         \Arg{intexpr_1} \Arg{intexpr_2}
%     \cs{dbFpSign}         \Arg{fpexpr}
%   \end{syntax}
% \begin{tblr}{ll}
%   \cs{dbIntAbs}         & is equal to \cs{int_abs:n} \\
%   \cs{dbIntSign}        & is equal to \cs{int_sign:n} \\
%   \cs{dbIntDivRound}    & is equal to \cs{int_div_round:nn} \\
%   \cs{dbIntDivTruncate} & is equal to \cs{int_div_truncate:nn} \\
%   \cs{dbIntMax}         & is equal to \cs{int_max:nn} \\
%   \cs{dbIntMin}         & is equal to \cs{int_min:nn} \\
%   \cs{dbIntMod}         & is equal to \cs{int_mod:nn} \\
%   \cs{dbFpSign}         & is equal to \cs{fp_sign:n} \\
% \end{tblr}
%
%   See detailed documentation of \pkg{interface3}. Example \ref{en-expr-db}
%   shows how to filter multiples of 3.
% \end{function}
%
% \iffalse
%<*verb>
% \fi
\begin{example*}(expr-db){Filter Multiples of 3}
  \dbNewDatabase{expr-db}{n=int}
  \dbNewRawFilter{mod}{expr-db}{n}{\dbIntMod{\dbval}{3} = 0}
  \dbNewStyle{expr-style}{expr-db}{
    item-code = \dbuse{n}\quad,
    filter    = mod,
  }
  \dbitemkv{expr-db}{n=2}
  \dbitemkv{expr-db}{n=3}
  \dbitemkv{expr-db}{n=6}
  \dbitemkv{expr-db}{n=7}
  \dbshow{expr-style}{expr-db}
\end{example*}
% \iffalse
%</verb>
% \fi
%
% \subsection{Special Macros}
%
% Some special macros are defined to expand to different contents according to context.
%
% \changes{1.1}{2022-01-05}{Add macro}{
%   \cs{dbarabic}, \cs{dbalph}, \cs{dbAlph}, \cs{dbroman},
%   \cs{dbRoman}
% }
% \changes{1.1}{2022-01-06}{Fix bug}{\cs{dbIndex} not defined}
% \begin{function}[added=2022-01-05, EXP]{
%   \dbval, \dbDatabase, \dbFilterName, \dbFilterInfo,
%   \dbIndex, \dbarabic, \dbalph, \dbAlph, \dbroman, \dbRoman
% }
% \begin{tblr}{ll}
%   \cs{dbval}        & Attribute value, only accessible in \cs{dbNewConditional}. \\
%   \cs{dbtoday}      & Date of today. \\
%   \cs{dbDatabase}   & Database name. \\
%   \cs{dbFilterName} & Filter name. \\
%   \cs{dbFilterInfo} & Filter information. \\
%   \cs{dbIndex}      & Record index, identical to \cs{dbuse}\marg{id} \\
%   \cs{dbarabic}     & Show the counter of query set as digits. \\
%   \cs{dbalph}       & Show the counter of query set as lowercase letters. \\
%   \cs{dbAlph}       & Show the counter of query set as uppercase letters. \\
%   \cs{dbroman}      & Show the counter of query set as lowercase roman numerals. \\
%   \cs{dbroman}      & Show the counter of query set as uppercase roman numerals. \\
% \end{tblr}
%
%   \cs{dbtoday} show the current date with the separator defined by
%   \cs{dbdatesep}.
% \end{function}
%
% \changes{1.1}{2022-01-07}{Update doc}{improve example}
% \changes{1.5}{2022-01-17}{Remove doc}{Remove big example}
%
% \end{documentation}
%
% \StopEventually{}
%
% \newgeometry{
%   left=5.5cm,
%   right=2cm,
%   top=2cm,
%   bottom=2cm
% }
%
% \begin{implementation}
%
% \section{Implementation}
%
% \changes{1.4}{2022-01-13}{Update code}{merge code and doc into
% \pkg{dbshow.dtx}}
%
%    \begin{macrocode}
%<*package>
%<@@=dbshow>
%    \end{macrocode}
%
% Check version of \pkg{l3kernel}.
%    \begin{macrocode}
% prop_concat, prop_gset_from_keyval
\__kernel_dependency_version_check:nn { 2021-11-07 } { l3prop }
% str_compare
\__kernel_dependency_version_check:nn { 2021-05-17 } { l3str }
% clist_map_tokens, clist_use
\__kernel_dependency_version_check:nn { 2021-05-10 } { l3clist }
%    \end{macrocode}
%
% \subsection{Variants and Variables}
%
%    \begin{macrocode}
\cs_generate_variant:Nn \msg_warning:nnnn      { nnnx }
\cs_generate_variant:Nn \keys_set:nn           { nv }
\cs_generate_variant:Nn \tl_put_right:Nn       { Nv }
\cs_generate_variant:Nn \clist_use:nn          { xx }
\cs_generate_variant:Nn \clist_use:nnnn        { xxxx }
\cs_generate_variant:Nn \clist_map_inline:nn   { Vn }
\cs_generate_variant:Nn \prop_get:NnN          { cVN }
\cs_generate_variant:Nn \regex_extract_all:nnN { nVN }
\cs_generate_variant:Nn \regex_split:nnN       { nVN }
\prg_generate_conditional_variant:Nnn \str_compare:nNn { VNV } { TF }
\prg_generate_conditional_variant:Nnn \int_compare:nNn { VNV } { TF }
\prg_generate_conditional_variant:Nnn \fp_compare:nNn  { VNV } { TF }
\prg_generate_conditional_variant:Nnn \regex_match:nn  { VV }  { TF }
\prg_generate_conditional_variant:Nnn \clist_if_in:Nn  { Nx }  { TF }
%    \end{macrocode}
%
% \begin{variable}{\c_@@_default_value_prop}
% Default default value of different types.
%    \begin{macrocode}
\prop_const_from_keyval:Nn \c_@@_default_value_prop {
  date  = \dbtoday,
  str   = ,
  tl    = ,
  clist = ,
  int   = 0,
  fp    = 0
}
%    \end{macrocode}
% \end{variable}
%
% \begin{variable}{\@@_type_clist}
% Supported types by \pkg{dbshow}.
%    \begin{macrocode}
\clist_const:Nn \@@_type_clist { date, str, tl, clist, int, fp }
%    \end{macrocode}
% \end{variable}
%
% \begin{variable}{\g_@@_raw_filter_int}
% Counter of anonymous filter.
%    \begin{macrocode}
\int_gzero_new:N \g_@@_raw_filter_int
%    \end{macrocode}
% \end{variable}
%
% \subsection{Messages}
%
% \begin{arguments}
%   \item \meta{database}
% \end{arguments}
%    \begin{macrocode}
\msg_new:nnn { dbshow } { non-existent-database } {
  Database~'#1'~does~not~exist~\msg_line_context:.
}
%    \end{macrocode}
%
% \changes{1.5}{2022-01-15}{Update check code}{transform arguments to string
% before check}
% \begin{macro}{\@@_check_database:n}
% Check if the database is valid.
% \begin{arguments}
%   \item \meta{database}
% \end{arguments}
%    \begin{macrocode}
\cs_new:Nn \@@_check_database:n {
  \prop_if_exist:cF { g_@@_attr_type_prop_\tl_to_str:n {#1} }
    { \msg_fatal:nnn { dbshow } { non-existent-database } {#1} }
}
%    \end{macrocode}
% \end{macro}
%
% \begin{arguments}
%   \item \meta{database}
%   \item \meta{attr}
% \end{arguments}
%    \begin{macrocode}
\msg_new:nnn { dbshow } { non-existent-attr } {
  Attribute~'#2'~of~database~'#1'~does~not~exist~\msg_line_context:.
}
%    \end{macrocode}
%
% \begin{macro}{\@@_check_attr:nn, \@@_check_attr:nV}
% Check if the attribute is valid.
% \begin{arguments}
%   \item \meta{database}
%   \item \meta{attr}
% \end{arguments}
%    \begin{macrocode}
\cs_new:Nn \@@_check_attr:nn {
  \prop_if_in:cnF { g_@@_attr_type_prop_\tl_to_str:n {#1} } {#2}
    { \msg_fatal:nnnn { dbshow } { non-existent-attr } {#1} {#2} }
}
\cs_generate_variant:Nn \@@_check_attr:nn { nV }
%    \end{macrocode}
% \end{macro}
%
% \begin{arguments}
%   \item \meta{style}
%   \item \meta{database}
% \end{arguments}
%    \begin{macrocode}
\msg_new:nnn { dbshow } { non-existent-style } {
  Style~'#1'~of~database~'#2'~does~not~exist~\msg_line_context:.
}
%    \end{macrocode}
%
% \begin{macro}{\@@_check_style:nn}
% Check if the style is valid.
% \begin{arguments}
%   \item \meta{style}
%   \item \meta{database}
% \end{arguments}
%    \begin{macrocode}
\cs_new:Nn \@@_check_style:nn {
  \tl_if_exist:cF { g_@@_style_opts_tl_\tl_to_str:n {#1}_\tl_to_str:n {#2} }
    { \msg_fatal:nnnn { dbshow } { non-existent-style } {#1} {#2} }
}
%    \end{macrocode}
% \end{macro}
%
% \begin{arguments}
%   \item \meta{database}
%   \item \meta{cond}
% \end{arguments}
%    \begin{macrocode}
\msg_new:nnn { dbshow } { non-existent-cond } {
  Conditional~'#2'~of~database~'#1'~does~not~exist~\msg_line_context:.
}
%    \end{macrocode}
%
% \begin{macro}{\@@_check_cond:nnn}
% Check if the conditional is valid.
% \begin{arguments}
%   \item \meta{database}
%   \item \meta{cond}
% \end{arguments}
%    \begin{macrocode}
\cs_new:Nn \@@_check_cond:nnn {
  \tl_if_exist:cF { g_@@_filter_attr_\tl_to_str:n {#1}_\tl_to_str:n {#2} }
    { \msg_fatal:nnnn { dbshow } { non-existent-cond } {#1} {#2} }
}
%    \end{macrocode}
% \end{macro}
%
% \begin{arguments}
%   \item \meta{database}
%   \item \meta{filter}
% \end{arguments}
%    \begin{macrocode}
\msg_new:nnn { dbshow } { non-existent-filter } {
  Filter~'#2'~of~database~'#1'~does~not~exist~and~is~ignored~\msg_line_context:.
}
%    \end{macrocode}
%
% \begin{macro}{\@@_check_filter:nn, \@@_check_filter:nv}
% Check if the filter is valid. Ignore the default filter |-none-|.
% \begin{arguments}
%   \item \meta{database}
%   \item \meta{filter}
% \end{arguments}
%    \begin{macrocode}
\cs_new:Nn \@@_check_filter:nn {
  \seq_if_exist:cF
  { g_@@_filter_run_seq_\tl_to_str:n {#1}_\tl_to_str:n {#2} } {
    \str_if_eq:eeF {#2} { -none- } {
      \msg_warning:nnnx { dbshow } { non-existent-filter } {#1} {#2}
    }
  }
}
\cs_generate_variant:Nn \@@_check_filter:nn { nv }
%    \end{macrocode}
% \end{macro}
%
% \begin{arguments}
%   \item \meta{type}
% \end{arguments}
%    \begin{macrocode}
\msg_new:nnn { dbshow } { non-existent-type } {
  Type~'#1'~does~not~exist,~the~type~of~attribute~should~be~one~of~
  \{date,~str,~tl,~clist,~int,~fp\}~\msg_line_context:.
}
%    \end{macrocode}
%
% \begin{macro}{\@@_check_type:n}
% Check if the type is valid.
% \begin{arguments}
%   \item \meta{type}
% \end{arguments}
%    \begin{macrocode}
\cs_new:Nn \@@_check_type:n {
  \clist_if_in:NnF \@@_type_clist {#1}
    { \msg_fatal:nnn { dbshow } { non-existent-type } {#1} }
}
%    \end{macrocode}
% \end{macro}
%
% \begin{arguments}
%   \item \meta{valid count}
%   \item \meta{real count}
%   \item \meta{value}
% \end{arguments}
%    \begin{macrocode}
\msg_new:nnn { dbshow } { wrong-seperator } {
  option~'sep'~should~contain~#1~items~but~only~#2~items~was~given,~
  sep~=~\{#3\}~\msg_line_context:.
}
%    \end{macrocode}
%
% \begin{macro}{\@@_sep_fatal:nnn, \@@_sep_fatal:xxx}
% Check the value of \opt{<attr>/sep} is valid.
% \begin{arguments}
%   \item \meta{valid count}
%   \item \meta{real count}
%   \item \meta{value}
% \end{arguments}
%    \begin{macrocode}
\cs_new:Nn \@@_sep_fatal:nnn {
  \msg_fatal:nnnnn { dbshow } { wrong-seperator } {#1} {#2} {#3}
}
\cs_generate_variant:Nn \@@_sep_fatal:nnn { xxx }
%    \end{macrocode}
% \end{macro}
%
% \begin{arguments}
%   \item \meta{type}
% \end{arguments}
%    \begin{macrocode}
\msg_new:nnn { dbshow } { unsupported-sort-type } {
  unsupported~sort~type:~'#1'~\msg_line_context:.~The~type~should~be~one~of~
  \{str,~date,~int,~fp\}.
}
%    \end{macrocode}
%
% \subsection{Create Database}
%
% \changes{1.5}{2022-01-15}{Fix bug}{can not use \cs{dbdatesep} midway}
% \begin{macro}{\@@_process_default_value:w}
% Create map from \meta{attr} to \meta{type} and map from \meta{attr} to
% \meta{default value}. Note that only one expansion is needed to get the
% correct default value from \cs{l_@@_tmp_default}.
% \begin{arguments}
%   \item \meta{database}
%   \item \meta{attr}
%   \item \meta{type}
%   \item \meta{default value}
% \end{arguments}
%    \begin{macrocode}
\cs_new:Npn \@@_process_default_value:w
#1\@@_sep#2\@@_sep#3|#4\@@_stop {
  \@@_check_type:n {#3}
  \prop_gput:cxx { g_@@_attr_type_prop_#1 } {#2} {#3}
  \prop_gput:cno { g_@@_default_map_#1 } {#2} {#4}
}
%    \end{macrocode}
% \end{macro}
%
% \begin{macro}{\@@_process_attr_type_prop:n}
% Parse default value from the value of the type map. If \meta{type spec} is
% \meta{type\orbar default value} then set the default value to \meta{default
% value}, otherwise if \meta{type spec} is \meta{type} then set the default
% value to the default value of the type.
% \begin{arguments}
%   \item \meta{database}
% \end{arguments}
%    \begin{macrocode}
\cs_new_protected:Nn \@@_process_attr_type_prop:n {
  \prop_gclear_new:c { g_@@_default_map_#1 }
%    \end{macrocode}
% \begin{arguments}[2]
%   \item \meta{attr}
%   \item \meta{type spec}
% \end{arguments}
%    \begin{macrocode}
  \prop_map_inline:cn { g_@@_attr_type_prop_#1 } {
    \str_if_in:nnTF {##2} { | } {
      \@@_process_default_value:w
        #1\@@_sep##1\@@_sep##2\@@_stop
    } {
      \prop_get:NnN \c_@@_default_value_prop {##2} \l_@@_tmp_default
      \@@_process_default_value:w
        #1\@@_sep##1\@@_sep##2|\l_@@_tmp_default\@@_stop
    }
  }
}
%    \end{macrocode}
% \end{macro}
%
% \begin{macro}{\@@_database_new:nn}
% Create a new \meta{database} with \meta{database spec}, which is a list of
% \meta{attr} = \meta{type spec}. This function initialize the index counter
% and save \meta{attr} = \meta{type spec} pairs to the type map. If
% \meta{database} has existed, the old definition is discarded.
% \begin{arguments}
%   \item \meta{database}
%   \item \meta{database spec}
% \end{arguments}
%    \begin{macrocode}
\cs_new_protected:Nn \@@_database_new:nn {
  \int_gzero_new:c { g_@@_counter_#1 }
  \prop_gset_from_keyval:cn { g_@@_attr_type_prop_#1 } {#2}
}
%    \end{macrocode}
% \end{macro}
%
% \begin{macro}{\@@_database_new_append:nn}
% Create a new \meta{database} with \meta{database spec}, which is a list of
% \meta{attr} = \meta{type spec}. This function initialize the index counter
% and save \meta{attr} = \meta{type spec} pairs to the type map. If
% \meta{database} has existed, the old definition is merged into the new
% definition that has a higher priority.
% \begin{arguments}
%   \item \meta{database}
%   \item \meta{database spec}
% \end{arguments}
%    \begin{macrocode}
\cs_new_protected:Nn \@@_database_new_append:nn {
  \int_gzero_new:c { g_@@_counter_#1 }
  \prop_if_exist:cF { g_@@_attr_type_prop_#1 }
    { \prop_new:c { g_@@_attr_type_prop_#1 } }
  \prop_gset_from_keyval:Nn \l_tmpa_prop {#2}
  \prop_concat:ccc { g_@@_attr_type_prop_#1 }
    { g_@@_attr_type_prop_#1 } { l_tmpa_prop }
}
%    \end{macrocode}
% \end{macro}
%
% \begin{macro}{\@@_database_new_inherit:nnn}
% Create a new \meta{database} with \meta{database spec}, which is a list of
% \meta{attr} = \meta{type spec}. The new \meta{database} is based on
% \meta{base database}. If \meta{database} is equal to \meta{base database},
% \cs{@@_database_new_append:nn} is called, otherwise the index counter is
% initialized and the definition is merged with the definition of \meta{base
% dadatabase}.
% \begin{arguments}
%   \item \meta{database}
%   \item \meta{base database}
%   \item \meta{database spec}
% \end{arguments}
%    \begin{macrocode}
\cs_new_protected:Nn \@@_database_new_inherit:nnn {
  \@@_check_database:n {#2}
  \str_if_eq:nnTF {#1} {#2} {
    \@@_database_new_append:nn {#1} {#3}
  } {
    \int_gzero_new:c { g_@@_counter_#1 }
    \prop_gset_from_keyval:cn { g_@@_attr_type_prop_#1 } {#3}
    \prop_concat:ccc { g_@@_attr_type_prop_#1 }
      { g_@@_attr_type_prop_#2 } { g_@@_attr_type_prop_#1 }
  }
}
%    \end{macrocode}
% \end{macro}
%
% \begin{macro}{\dbNewDatabase}
% User interface to create a new \meta{database}. Internal attribute |id| is
% added to \meta{database}. After attributes are settle down, default values
% are parsed by \cs{@@_process_attr_type_prop:n}, and keys serving to save data of
% \meta{database} are defined. At last, we define the |default| style as its
% name implies.
% \begin{arguments}
%   \item \meta{star} that indicates append or not
%   \item \meta{base database}
%   \item \meta{database}
%   \item \meta{database spec}
% \end{arguments}
%    \begin{macrocode}
\NewDocumentCommand { \dbNewDatabase } { s o m m } {
  \IfNoValueTF {#2} {
    \IfBooleanTF {#1}
      { \@@_database_new_append:nn {#3} {#4} }
      { \@@_database_new:nn {#3} {#4} }
  } { \@@_database_new_inherit:nnn {#3} {#2} {#4} }
  \@@_database_new_append:nn {#3} { id=int }
  \@@_process_attr_type_prop:n {#3}
  \@@_set_database_keys:n {#3}
  \dbNewStyle{default}{#3}{}
}
%    \end{macrocode}
% \end{macro}
%
% \begin{macro}{\@@_set_database_keys:n}
% Set keys of \meta{database} to make it able to save data with key-value
% pairs.
% \begin{arguments}
%   \item \meta{database}
% \end{arguments}
%    \begin{macrocode}
\cs_new_protected:Nn \@@_set_database_keys:n {
%    \end{macrocode}
% \begin{arguments}[2]
%   \item \meta{type}
% \end{arguments}
%    \begin{macrocode}
  \prop_map_inline:cn { g_@@_attr_type_prop_#1 } {
    \keys_define:nn { dbshow/database/#1 } {
%    \end{macrocode}
% \begin{arguments}[3]
%   \item \meta{data}
% \end{arguments}
%    \begin{macrocode}
      ##1 .code:n = \@@_save_data:nnn {#1} {##1} {####1},
      ##1* .code:n = {
        \@@_save_data:nnn {#1} {##1} { \exp_not:n {####1} },
      },
    }
  }
}
%    \end{macrocode}
% \end{macro}
%
% \begin{macro}{\@@_get_type:nn, \@@_get_type:nV}
% Convenient function to get the \meta{type} of \meta{attr} of \meta{database}.
% \begin{arguments}
%   \item \meta{database}
%   \item \meta{attr}
% \end{arguments}
%    \begin{macrocode}
\cs_new:Nn \@@_get_type:nn {
  \prop_item:cn { g_@@_attr_type_prop_#1 } {#2}
}
\cs_generate_variant:Nn \@@_get_type:nn { nV }
%    \end{macrocode}
% \end{macro}
%
% \begin{macro}{\@@_get_counter:n}
% Convenient function to get the value of the index counter.
% \begin{arguments}
%   \item \meta{database}
% \end{arguments}
%    \begin{macrocode}
\cs_new:Nn \@@_get_counter:n {
  \int_use:c { g_@@_counter_#1 }
}
%    \end{macrocode}
% \end{macro}
%
% \begin{macro}{\@@_step_counter:n}
% Convenient function to step the index counter.
% \begin{arguments}
%   \item \meta{database}
% \end{arguments}
%    \begin{macrocode}
\cs_new:Nn \@@_step_counter:n {
  \int_gincr:c { g_@@_counter_#1 }
}
%    \end{macrocode}
% \end{macro}
%
% \begin{macro}{\dbclear}
% User interface to clear the \meta{database}. The index counter is set to
% zero and data is not really wiped.
% \begin{arguments}
%   \item \meta{database}
% \end{arguments}
%    \begin{macrocode}
\NewDocumentCommand { \dbclear } { m } {
  \int_gzero:c { g_@@_counter_#1 }
}
%    \end{macrocode}
% \end{macro}
%
% \subsection{Store Data}
%
% \begin{macro}{\@@_save_data:nnn, \@@_save_data:nnx}
% Internal function to store data into
% \csnot{g_@@_data_\meta{database}_\meta{attr}_\meta{index}}.
% \begin{arguments}
%   \item \meta{database}
%   \item \meta{attr}
%   \item \meta{data}
% \end{arguments}
%    \begin{macrocode}
\cs_new:Nn \@@_save_data:nnn {
  \@@_check_attr:nn {#1} {#2}
  \str_case_e:nn { \@@_get_type:nn {#1} {#2} } {
    { str }   { \str_clear_new:c }
    { tl }    { \tl_gclear_new:c }
    { clist } { \clist_gclear_new:c }
    { int }   { \int_gzero_new:c }
    { fp }    { \fp_gzero_new:c }
    { date }  { \__dbdate_gclear_new:x }
  } { g_@@_data_#1_#2_\@@_get_counter:n {#1} }
  \str_case_e:nn { \@@_get_type:nn {#1} {#2} } {
    { str }   { \str_gset:cn }
    { tl }    { \tl_gset:cn }
    { clist } { \clist_gset:cn }
    { int }   { \int_gset:cn }
    { fp }    { \fp_gset:cn }
    { date }  { \__dbdate_gset:xx }
  } { g_@@_data_#1_#2_\@@_get_counter:n {#1} } {#3}
}
\cs_generate_variant:Nn \@@_save_data:nnn { nnx }
%    \end{macrocode}
% \end{macro}
%
% \changes{1.4}{2022-01-14}{Fix bug}{data cannot contain \cs{par}}
% \begin{macro}{dbitem, \@@_set_default:nn, \dbsave, \dbsave*}
% Environment \env{dbitem} is the user interface to save data with \meta{attr}
% = \meta{data} pairs or using \cs{dbsave}. First we step the index counter
% and set the default value for each attribute. Then we set the value by
% \meta{attr-value list} and the index is also stored to default attribute
% |id|.
% \begin{arguments}
%   \item \meta{database}
%   \item \meta{attr-value list}
% \end{arguments}
%    \begin{macrocode}
\NewDocumentEnvironment { dbitem } { m +O{} } {
  \@@_check_database:n {#1}
  \@@_step_counter:n {#1}
%    \end{macrocode}
% This function is used to set the default values of each attribute.
% \begin{arguments}
%   \item \meta{attr}
% \end{arguments}
%    \begin{macrocode}
  \cs_set:Nn \@@_set_default:nn {
    \@@_save_data:nnx {#1} {##1} {
      \prop_item:cn { g_@@_default_map_#1 } {##1}
    }
  }
  \prop_map_function:cN { g_@@_attr_type_prop_#1 } \@@_set_default:nn
  \@@_save_data:nnx {#1} { id } { \@@_get_counter:n {#1} }
  \keys_set:nn { dbshow/database/#1 } {#2}
%    \end{macrocode}
% \begin{arguments}
%   \item \meta{star} that indicates whether to use \cs{exp_not:n}
%   \item \meta{attr}
%   \item \meta{data}
% \end{arguments}
%    \begin{macrocode}
  \NewDocumentCommand { \dbsave } { s m +m } {
    \IfBooleanTF {##1} {
      \@@_save_data:nnn {#1} {##2} { \exp_not:n {##3} }
    } {
      \@@_save_data:nnn {#1} {##2} {##3}
    }
  }
} {}
%    \end{macrocode}
% \end{macro}
%
% \begin{macro}{\dbitemkv}
% Store data with \meta{attr-value list}
% \begin{arguments}
%   \item \meta{database}
%   \item \meta{attr-value list}
% \end{arguments}
%    \begin{macrocode}
\NewDocumentCommand { \dbitemkv } { m +m } {
  \begin{dbitem}{#1}[#2]
  \end{dbitem}
}
%    \end{macrocode}
% \end{macro}
%
% \subsection{Filter}
%
% Following functions have the same argument specifications:
% \begin{arguments}
%   \item \meta{star boolean}
%   \item \meta{expr} specified by \cs{dbNewConditional}
%   \item \meta{current value}, i.e. \cs{dbval}
%   \item \meta{true code} to set filter boolean to true
%   \item \meta{false code} to set filter boolean to false
% \end{arguments}
% These functions aim to filter \meta{attr} of certain type with \meta{expr}
% in every epoch of iteration. The basic idea is to save the filter result
% into a boolean variable corresponding to each conditional.
%
% \begin{macro}{\@@_filter_int:NNNnn, \@@_filter_int:cccnn}
% Filter integers.
% \begin{arguments}
%   \item \meta{star boolean}
%   \item \meta{expr} specified by \cs{dbNewConditional}
%   \item \meta{current value}, i.e. \cs{dbval}
%   \item \meta{true code} to set filter boolean to true
%   \item \meta{false code} to set filter boolean to false
% \end{arguments}
%    \begin{macrocode}
\cs_new:Nn \@@_filter_int:NNNnn {
  \int_compare:nTF {#2} {#4} {#5}
}
\cs_generate_variant:Nn \@@_filter_int:NNNnn { cccnn }
%    \end{macrocode}
% \end{macro}
%
% \begin{macro}{\@@_filter_fp:NNNnn, \@@_filter_fp:cccnn}
% Filter floating point values.
% \begin{arguments}
%   \item \meta{star boolean}
%   \item \meta{expr} specified by \cs{dbNewConditional}
%   \item \meta{current value}, i.e. \cs{dbval}
%   \item \meta{true code} to set filter boolean to true
%   \item \meta{false code} to set filter boolean to false
% \end{arguments}
%    \begin{macrocode}
\cs_new:Nn \@@_filter_fp:NNNnn {
  \int_compare:nTF {#2} {#4} {#5}
}
\cs_generate_variant:Nn \@@_filter_fp:NNNnn { cccnn }
%    \end{macrocode}
% \end{macro}
%
% \begin{macro}{\@@_filter_clist:NNNnn, \@@_filter_clist:cccnn}
% Filter comma lists. If \meta{star bool} is true, all values specified by
% \meta{conditional} should be contained in current list. Otherwise, condition
% is met if any value specified by \meta{conditional} appears in current list.
% \begin{arguments}
%   \item \meta{star boolean}
%   \item \meta{expr} specified by \cs{dbNewConditional}
%   \item \meta{current value}, i.e. \cs{dbval}
%   \item \meta{true code} to set filter boolean to true
%   \item \meta{false code} to set filter boolean to false
% \end{arguments}
%    \begin{macrocode}
\cs_new_protected:Nn \@@_filter_clist:NNNnn {
%    \end{macrocode}
% Initial filter boolean of starred conditional is true. And then we check
% every value in \meta{conditional}. If there is some value that does not
% appear in current list, the result is set to false and the loop is broken.
%    \begin{macrocode}
  \bool_if:NTF #1 {
    #4 \clist_map_inline:Vn #2
    { \clist_if_in:NnF #3 {##1} { #5 \clist_map_break: } }
  } {
%    \end{macrocode}
% Initial filter boolean of unstarred conditional is false. And then we check
% every value in \meta{conditional}. If there is some value that does appear
% in current list, the result is set to true and the loop is broken.
%    \begin{macrocode}
    #5 \clist_map_inline:Vn #2
    { \clist_if_in:NnT #3 {##1} { #4 \clist_map_break: } }
  }
}
\cs_generate_variant:Nn \@@_filter_clist:NNNnn { cccnn }
%    \end{macrocode}
% \end{macro}
%
% \begin{macro}{\@@_filter_str:NNNnn, \@@_filter_str:cccnn}
% Filter strings with regex expression. Starred filter matches the whole
% string while unstarred matches part of the string.
% \begin{arguments}
%   \item \meta{star boolean}
%   \item \meta{expr} specified by \cs{dbNewConditional}
%   \item \meta{current value}, i.e. \cs{dbval}
%   \item \meta{true code} to set filter boolean to true
%   \item \meta{false code} to set filter boolean to false
% \end{arguments}
%    \begin{macrocode}
\cs_new_protected:Nn \@@_filter_str:NNNnn {
  \bool_if:NT #1 {
    \tl_put_left:Nn #2 { \A }
    \tl_put_right:Nn #2 { \Z }
  }
  \regex_match:VVTF #2 #3 {#4} {#5}
}
\cs_generate_variant:Nn \@@_filter_str:NNNnn { cccnn }
%    \end{macrocode}
% \end{macro}
%
% \begin{macro}{\@@_filter_tl:NNNnn, \@@_filter_tl:cccnn}
% Filter token list with regex expression. It is the same with string.
%    \begin{macrocode}
\cs_gset_eq:NN \@@_filter_tl:NNNnn \@@_filter_str:NNNnn
\cs_generate_variant:Nn \@@_filter_tl:NNNnn { cccnn }
%    \end{macrocode}
% \end{macro}
%
% \begin{macro}{\@@_parse_date_cond:NNw}
% Help function to parse \meta{review points} and \meta{date} and store them
% in \meta{clist var} and \meta{tl var}.
% \begin{arguments}
%   \item \meta{clist var} to store \meta{review points}
%   \item \meta{tl var} to store \meta{date}
% \end{arguments}
%    \begin{macrocode}
\cs_new_protected:Npn \@@_parse_date_cond:NNw #1#2#3|#4\@@_stop {
  \clist_set_eq:Nc #1 { g__review_points_#3 }
  \tl_set:Nn #2 {#4}
}
%    \end{macrocode}
% \end{macro}
%
% \begin{macro}{\@@_filter_date:NNNnn, \@@_filter_date:cccnn}
% Filter dates by \meta{review points} or \meta{expr}.
% \begin{arguments}
%   \item \meta{star boolean}
%   \item \meta{expr} specified by \cs{dbNewConditional}
%   \item \meta{current value}, i.e. \cs{dbval}
%   \item \meta{true code} to set filter boolean to true
%   \item \meta{false code} to set filter boolean to false
% \end{arguments}
%    \begin{macrocode}
\cs_new_protected:Nn \@@_filter_date:NNNnn {
%    \end{macrocode}
% For starred \meta{conditional}, calculate the days between current day, i.e.
% \cs{dbval} and the date to be compared, i.e. \cs{l_@@_filter_tmp_tl} to see
% if the result appears in the \meta{review points}.
%    \begin{macrocode}
  \bool_if:NTF #1 {
    \int_zero_new:N \l_@@_filter_diff_int
    \exp_last_unbraced:NNNV \@@_parse_date_cond:NNw
      \l_@@_filter_tmp_clist \l_@@_filter_tmp_tl {#2} \@@_stop
    \__dbdate_clear_new:n { tmp_day1 }
    \__dbdate_clear_new:n { tmp_day2 }
    \__dbdate_set:xx { tmp_day1 } { \l_@@_filter_tmp_tl }
    \__dbdate_set:xx { tmp_day2 } {#3}
    \__dbdate_sub:nnN { tmp_day1 } { tmp_day2 } \l_@@_filter_diff_int
    #5
    \clist_map_inline:Nn \l_@@_filter_tmp_clist {
      \int_compare:nNnT { \l_@@_filter_diff_int } = {##1}
        { #4 \clist_map_break: }
    }
  } {
%    \end{macrocode}
% For unstarred \meta{conditional} which parses \meta{expr}. We first replace
% each date to an absolute integer. We did not use \cs{regex_replace_all:nnN}
% because \cs{_dbdate_to_int:nN} is nonexpandable. So the code seems a little
% complicated and unsightly, but it worked. Note that \meta{expr} should be
% updated locally.
%    \begin{macrocode}
    \int_zero_new:N \l_@@_filter_tmpa_int
    \int_zero_new:N \l_@@_filter_tmpb_int
    \tl_set:Nx \l_@@_expr_tl {#2}
    \regex_extract_all:nVN { \d{4}/\d+/\d+ }
      \l_@@_expr_tl \l_@@_filter_date_seq
    \regex_split:nVN { \d{4}/\d+/\d+ }
      \l_@@_expr_tl \l_@@_filter_other_seq
    \tl_clear:N \l_@@_expr_tl
    \int_set:Nn \l_@@_filter_tmpa_int
      { \seq_count:N \l_@@_filter_date_seq }
    \int_step_inline:nn { \l_@@_filter_tmpa_int } {
      \tl_put_right:Nx \l_@@_expr_tl
        { \seq_item:Nn \l_@@_filter_other_seq {##1} }
      \__dbdate_clear_new:n { date-tmp }
      \__dbdate_set:xx { date-tmp }
        { \seq_item:Nn \l_@@_filter_date_seq {##1} }
      \__dbdate_to_int:nN { date-tmp } \l_@@_filter_tmpb_int
      \tl_put_right:Nx \l_@@_expr_tl
        { \int_use:N \l_@@_filter_tmpb_int }
    }
    \tl_put_right:Nx \l_@@_expr_tl {
      \seq_item:Nn \l_@@_filter_other_seq
        { \l_@@_filter_tmpa_int + 1 }
    }
    \int_compare:nTF { \l_@@_expr_tl } {#4} {#5}
  }
}
\cs_generate_variant:Nn \@@_filter_date:NNNnn { cccnn }
%    \end{macrocode}
% \end{macro}
%
% \begin{macro}{\@@_filter:nnn}
% Filter records with \meta{conditional}
% \begin{arguments}
%   \item \meta{database}
%   \item \meta{conditional}
%   \item \meta{index}
% \end{arguments}
%    \begin{macrocode}
\cs_new_protected:Nn \@@_filter:nnn {
  \tl_set_eq:Nc \l_@@_attr_tl { g_@@_filter_attr_tl_#1_#2 }
  \cs_set_eq:Nc \dbval { g_@@_data_#1_\l_@@_attr_tl _#3 }
  \tl_set:Nx \l_@@_type_tl { \@@_get_type:nV {#1} \l_@@_attr_tl }
  \use:c
    { @@_filter_\l_@@_type_tl :cccnn }
    { g_@@_cond_star_bool_#1_#2 }
    { g_@@_filter_expr_tl_#1_#2 }
    { dbval }
    { \bool_gset_true:c  { g_@@_filter_bool_#1_#2 } }
    { \bool_gset_false:c { g_@@_filter_bool_#1_#2 } }
}
%    \end{macrocode}
% \end{macro}
%
% \begin{macro}{\@@_new_conditional:nnnnn}
% For a \meta{conditional} of \meta{attr}, map \meta{conditional} to
% \meta{attr} and map \meta{conditional} to {expr}. The \meta{boolean var} is
% created to store the filter result. An hook function is defined and the
% \meta{conditional} is recorded in the sequence.
% \begin{arguments}
%   \item \meta{database}
%   \item \meta{conditional}
%   \item \meta{attr}
%   \item \meta{expr}
%   \item \meta{star}
% \end{arguments}
%    \begin{macrocode}
\cs_new_protected:Nn \@@_new_conditional:nnnnn {
  \@@_check_database:n {#1}
  \@@_check_attr:nn {#1} {#3}
  \tl_gset:cn { g_@@_filter_attr_tl_#1_#2 } {#3}
  \tl_gset:cn { g_@@_filter_expr_tl_#1_#2 } {#4}
  \bool_if_exist:cF { g_@@_filter_bool_#1_#2 }
    { \bool_new:c { g_@@_filter_bool_#1_#2 } }
  \bool_if_exist:cF { g_@@_cond_star_bool_#1_#2 }
    { \bool_new:c { g_@@_cond_star_bool_#1_#2 } }
  \IfBooleanTF {#5}
    { \bool_gset_true:c { g_@@_cond_star_bool_#1_#2 } }
    { \bool_gset_false:c { g_@@_cond_star_bool_#1_#2 } }
%    \end{macrocode}
% Filter hook function.
% \begin{arguments}[2]
%   \item \meta{index}
% \end{arguments}
%    \begin{macrocode}
  \cs_gset:cn { g_@@_filter_hook_#1_#2:n } {
    \@@_filter:nnn {#1} {#2} {##1}
  }
  \bool_if_exist:cF { g_@@_cond_exist_bool_#1_#2 }
    { \bool_set_false:c { g_@@_cond_exist_bool_#1_#2 } }
  \bool_if:cF { g_@@_cond_exist_bool_#1_#2 }
    { \seq_gput_right:cn { g_@@_cond_seq_#1 } {#2} }
}
%    \end{macrocode}
% \end{macro}
%
% \begin{macro}{\@@_combine_conditional:nnn, \@@_combine_conditional:nVn}
% First extract all the \meta{conditional} in \meta{conditional expr} and for
% every \meta{conditional} in the record sequence, if it is in
% \meta{conditional expr}, then add the corresponding hook function to running
% sequence. Then replace all the \meta{conditional} in \meta{conditional expr}
% with corresponding boolean variable and save the result in the filter
% boolean variable.
% \begin{arguments}
%   \item \meta{database}
%   \item \meta{filter}
%   \item \meta{conditional expr}
% \end{arguments}
%    \begin{macrocode}
\cs_new_protected:Nn \@@_combine_conditional:nnn {
  \tl_gset_eq:cN { g_@@_filter_bool_tl_#1_#2 } \c_true_bool
  \seq_gclear_new:c { g_@@_filter_run_seq_#1_#2 }
  \regex_extract_all:nnN { [^!=&<>()\ ]+ } {#3} \l_@@_cond_seq
%    \end{macrocode}
% \begin{arguments}[2]
%   \item \meta{conditional}
% \end{arguments}
%    \begin{macrocode}
  \seq_map_inline:Nn \l_@@_cond_seq {
    \seq_if_in:cnT { g_@@_cond_seq_#1 } {##1} {
      \seq_if_in:cnF { g_@@_filter_run_seq_#1_#2 }
        { g_@@_filter_hook_#1_##1:n }
      {
        \seq_gput_right:cn { g_@@_filter_run_seq_#1_#2 }
          { g_@@_filter_hook_#1_##1:n }
      }
    }
  }
  \tl_set:Nn \l_@@_cond_expr_tl {#3}
  \regex_replace_all:nnN
    { (\w|-|\d|\_)+ }
    { \c{ g_@@_filter_bool_#1_\0 } }
    \l_@@_cond_expr_tl
  \tl_gset_eq:cN
    { g_@@_filter_bool_tl_#1_#2 } \l_@@_cond_expr_tl
}
\cs_generate_variant:Nn \@@_combine_conditional:nnn { nVn }
%    \end{macrocode}
% \end{macro}
%
% \begin{macro}{dbFilters, \dbNewConditional, \dbNewCond,
% \dbCombineConditionals, \dbCombCond}
% Environment to define conditionals and filters.
% \begin{arguments}
%   \item \meta{database}
% \end{arguments}
%    \begin{macrocode}
\NewDocumentEnvironment { dbFilters } { s m } {
  \seq_if_exist:cF { g_@@_cond_seq_#2 }
    { \seq_new:c { g_@@_cond_seq_#2 } }
%    \end{macrocode}
% \begin{arguments}[2]
%   \item \meta{star}
%   \item \meta{filter/conditional}
%   \item \meta{attr}
%   \item \meta{expr}
%   \item \meta{filter info}
% \end{arguments}
%    \begin{macrocode}
  \DeclareDocumentCommand { \dbNewConditional } { s m m m O{} } {
    \@@_new_conditional:nnnnn {#2} {##2} {##3} {##4} {##1}
    \IfValueT {#1} {
      \dbCombCond{##2}{##2}[##5]
    }
  }
%    \end{macrocode}
% \begin{arguments}[2]
%   \item \meta{filter}
%   \item \meta{conditional expr}
%   \item \meta{filter info}
% \end{arguments}
%    \begin{macrocode}
  \DeclareDocumentCommand { \dbCombineConditionals } { m m O{} } {
    \tl_gset:cn { g_@@_filter_info_tl_#2_##1 } {##3}
    \@@_combine_conditional:nnn {#2} {##1} {##2}
  }
  \cs_set_eq:NN \dbNewCond  \dbNewConditional
  \cs_set_eq:NN \dbCombCond \dbCombineConditionals
} {}
%    \end{macrocode}
% \end{macro}
%
% \begin{macro}{\dbNewRawFilter, \dbNewRawFilter*}
% Define filter with single conditional.
% \begin{arguments}
%   \item \meta{star}
%   \item \meta{filter and conditional name}
%   \item \meta{database}
%   \item \meta{attr}
%   \item \meta{expr}
%   \item \meta{filter info}
% \end{arguments}
%    \begin{macrocode}
\DeclareDocumentCommand { \dbNewRawFilter } { s m m m m O{} } {
  \seq_if_exist:cF { g_@@_cond_seq_#3 }
    { \seq_new:c { g_@@_cond_seq_#3 } }
  \@@_new_conditional:nnnnn {#3} {#2} {#4} {#5} {#1}
  \tl_gset:cn { g_@@_filter_info_tl_#3_#2 } {#6}
  \@@_combine_conditional:nnn {#3} {#2} {#2}
}
%    \end{macrocode}
% \end{macro}
%
% \begin{macro}{\dbNewReviewPoints}
% User interface to define \meta{review points}.
% \begin{arguments}
%   \item \meta{name}
%   \item \meta{points}
% \end{arguments}
%    \begin{macrocode}
\NewDocumentCommand { \dbNewReviewPoints } { m m } {
  \clist_set:cn { g__review_points_#1 } {#2}
}
%    \end{macrocode}
% \end{macro}
%
% \begin{macro}{\dbIntAbs, \dbIntSign, \dbIntDivRound, \dbIntDivTruncate,
% \dbIntMax, \dbIntMin, \dbIntMod, \dbFpSign}
% Function aliases to support more operations of integer and floating points.
%    \begin{macrocode}
\cs_gset_eq:NN \dbIntAbs         \int_abs:n
\cs_gset_eq:NN \dbIntSign        \int_sign:n
\cs_gset_eq:NN \dbIntDivRound    \int_div_round:nn
\cs_gset_eq:NN \dbIntDivTruncate \int_div_truncate:nn
\cs_gset_eq:NN \dbIntMax         \int_max:nn
\cs_gset_eq:NN \dbIntMin         \int_min:nn
\cs_gset_eq:NN \dbIntMod         \int_mod:nn
\cs_gset_eq:NN \dbFpSign         \fp_sign:n
%    \end{macrocode}
% \end{macro}
%
% \subsection{Style and Options}
%
% \begin{macro}{\@@_new_attr_style:nnn}
% Define style keys for each attribute.
% \begin{arguments}
%   \item \meta{style}
%   \item \meta{database}
%   \item \meta{attr}
% \end{arguments}
%    \begin{macrocode}
\cs_new_protected:Nn \@@_new_attr_style:nnn {
  \@@_check_attr:nn {#2} {#3}
  \keys_define:nn { dbshow/style/#1/#3 } {
    before-code      .tl_gset:c    = {
      g_@@_style_attr_before_tl_#1_#2_#3
    },
    before-code      .initial:n    = ,
    after-code       .tl_gset:c    = {
      g_@@_style_attr_after_tl_#1_#2_#3
    },
    after-code       .initial:n    = ,
    code             .code:n       = {
      \bool_gset_false:c { g_@@_style_attr_exp_bool_#1_#2_#3 }
      \cs_gset:cn { @@_style_attr_code_#1_#2_#3:n } {##1}
    },
    code             .initial:n    = {##1},
    code*            .code:n       = {
      \bool_gset_true:c  { g_@@_style_attr_exp_bool_#1_#2_#3 }
      \cs_gset:cn { @@_style_attr_code_#1_#2_#3:n } {##1}
    },
%    \end{macrocode}
% For comma list and date.
%    \begin{macrocode}
    sep              .clist_gset:c = {
      g_@@_style_attr_sep_#1_#2_#3
    },
%    \end{macrocode}
% Only for comma list.
%    \begin{macrocode}
    item-before-code .tl_gset:c    = {
      g_@@_style_attr_item_before_tl_#1_#2_#3
    },
    item-before-code .initial:n    = ,
    item-after-code  .tl_gset:c    = {
      g_@@_style_attr_item_after_tl_#1_#2_#3
    },
    item-after-code  .initial:n    = ,
    item-code        .code:n       = {
      \bool_gset_false:c { g_@@_style_clist_item_exp_bool_#1_#2_#3 }
      \cs_gset:cn { @@_style_clist_item_code_#1_#2_#3:n } {##1}
    },
    item-code        .initial:n    = {##1},
    item-code*       .code:n       = {
      \bool_gset_true:c  { g_@@_style_clist_item_exp_bool_#1_#2_#3 }
      \cs_gset:cn { @@_style_clist_item_code_#1_#2_#3:n } {##1}
    },
%    \end{macrocode}
% Only for date.
%    \begin{macrocode}
    zfill            .bool_gset:c  = {
      g_@@_style_date_zfill_bool_#1_#2_#3
    },
    zfill            .initial:n    = true,
    zfill            .default:n    = true,
    format-code      .code:n       = {
      \cs_gset:cn { @@_style_date_format_code_#1_#2_#3:nnn } {##1}
      \cs_generate_variant:cn
        { @@_style_date_format_code_#1_#2_#3:nnn } { xxx }
    },
  }
  \str_case_e:nn { \@@_get_type:nn {#2} {#3} } {
    { clist }
    { \keys_set:nn { dbshow/style/#1/#3 } { sep = { { ,~ } } } }
    { date }
    { \keys_set:nn { dbshow/style/#1/#3 } { sep = { { /  } } } }
  }
}
%    \end{macrocode}
% \end{macro}
%
% \begin{macro}{\@@_new_database_style:nn}
% Define style keys.
% \begin{arguments}
%   \item \meta{style}
%   \item \meta{database}
% \end{arguments}
%    \begin{macrocode}
\cs_new_protected:Nn \@@_new_database_style:nn {
  \@@_check_database:n {#2}
  \keys_define:nn { dbshow/style/#1 } {
    raw-filter         .code:n       = {
      \int_gincr:N  \g_@@_raw_filter_int
      \str_set:Nx \l_@@_raw_filter_str
        { -raw\int_use:N \g_@@_raw_filter_int - }
      \tl_gset:cV { g_@@_filter_#1_#2 } \l_@@_raw_filter_str
      \@@_combine_conditional:nVn {#2}  \l_@@_raw_filter_str {##1}
    },
    filter           .tl_gset:c    = { g_@@_filter_#1_#2 },
    filter           .initial:n    = -none-,
    sort             .clist_gset:c = { g_@@_sort_clist_#1_#2 },
    before-code      .tl_gset:c    = { g_@@_style_before_tl_#1_#2 },
    before-code      .initial:n    = ,
    after-code       .tl_gset:c    = { g_@@_style_after_tl_#1_#2 },
    after-code       .initial:n    = ,
    item-before-code .tl_gset:c    = { g_@@_style_item_before_tl_#1_#2 },
    item-before-code .initial:n    = ,
    item-after-code  .tl_gset:c    = { g_@@_style_item_after_tl_#1_#2 },
    item-after-code  .initial:n    = ,
    item-code        .code:n       = {
      \bool_gset_false:c { g_@@_style_item_exp_bool_#1_#2 }
      \tl_gset:cn { g_@@_style_item_tl_#1_#2 } {##1}
    },
    item-code        .initial:n    = ,
    item-code*       .code:n       = {
      \bool_gset_true:c { g_@@_style_item_exp_bool_#1_#2 }
      \tl_gset:cn { g_@@_style_item_tl_#1_#2 } {##1}
    },
  }
  \prop_map_inline:cn { g_@@_attr_type_prop_#2 }
    { \@@_new_attr_style:nnn {#1} {#2} {##1} }
}
%    \end{macrocode}
% \end{macro}
%
% \changes{1.5}{2022-01-15}{Add check}{check if database is valid in
% \cs{dbNewStyle}}
% \begin{macro}{\dbNewStyle}
% Set style options based on \meta{base style}.
% \begin{arguments}
%   \item \meta{base style clist}
%   \item \meta{style}
%   \item \meta{database}
%   \item \meta{options}
% \end{arguments}
%    \begin{macrocode}
\NewDocumentCommand { \dbNewStyle } { o m m +m } {
  \@@_check_database:n {#3}
  \tl_gset:cn { g_@@_style_opts_tl_#2_#3 } { #4, }
  \IfValueT {#1} {
    \tl_clear_new:N \l_@@_style_tmp_tl
%    \end{macrocode}
% \begin{arguments}[2]
%   \item \meta{base style}
% \end{arguments}
%    \begin{macrocode}
    \clist_map_inline:nn {#1} {
      \@@_check_style:nn {##1} {#3}
      \tl_if_exist:cT { g_@@_style_opts_tl_##1_#3 } {
        \tl_concat:ccc { l_@@_style_tmp_tl }
          { l_@@_style_tmp_tl } { g_@@_style_opts_tl_##1_#3 }
      }
    }
    \tl_gconcat:ccc { g_@@_style_opts_tl_#2_#3 }
      { l_@@_style_tmp_tl } { g_@@_style_opts_tl_#2_#3 }
  }
  \@@_new_database_style:nn {#2} {#3}
  \keys_set:nv { dbshow/style/#2 } { g_@@_style_opts_tl_#2_#3 }
}
%    \end{macrocode}
% \end{macro}
%
% \subsection{Sort}
%
% \begin{macro}{\@@_sort_parse_star:NNNw}
% Parse descending sorting rule.
% \begin{arguments}
%   \item \meta{tl var} representing for relation to keep order the same
%   \item \meta{tl var} representing for relation to swap the order
%   \item \meta{tl var} to store the \meta{attr}
% \end{arguments}
%    \begin{macrocode}
\cs_new_protected:Npn \@@_sort_parse_star:NNNw #1#2#3#4* {
  \tl_set:Nn #1 { > }
  \tl_set:Nn #2 { < }
  \tl_set:Nn #3 {#4}
}
%    \end{macrocode}
% \end{macro}
%
% \begin{macro}{\@@_sort:nNn}
% Sort the records.
% \begin{arguments}
%   \item \meta{database}
%   \item \meta{index clist}
%   \item \meta{style}
% \end{arguments}
%    \begin{macrocode}
\cs_new_protected:Nn \@@_sort:nNn {
  \int_zero_new:N \l_@@_sort_len_int
  \int_zero_new:N \l_@@_sort_tmp_int
  \int_set:Nn \l_@@_sort_len_int
    { \clist_count:c { g_@@_sort_clist_#3_#1 } }
%    \end{macrocode}
% Sort recursively.
%    \begin{macrocode}
  \clist_sort:Nn #2 {
    \int_zero:N \l_@@_sort_tmp_int
%    \end{macrocode}
% Sort single attribute.
%    \begin{macrocode}
    \cs_set:Nn \@@_sort_single: {
      \int_incr:N \l_@@_sort_tmp_int
      \str_set:Nx \l_@@_sort_attr_str {
        \clist_item:cn
          { g_@@_sort_clist_#3_#1 }
          { \l_@@_sort_tmp_int }
      }
%    \end{macrocode}
% Parse \meta{attr} and corresponding \meta{type} and set compare operators.
%    \begin{macrocode}
      \str_if_in:NnTF \l_@@_sort_attr_str { * } {
        \exp_after:wN \@@_sort_parse_star:NNNw
          \exp_after:wN \l_@@_sort_same_op_tl
          \exp_after:wN \l_@@_sort_swap_op_tl
          \exp_after:wN \l_@@_sort_attr_str
          \l_@@_sort_attr_str
      } {
        \tl_set:Nn \l_@@_sort_same_op_tl { < }
        \tl_set:Nn \l_@@_sort_swap_op_tl { > }
      }
%    \end{macrocode}
% Check if type is valid and transform date to string.
%    \begin{macrocode}
      \@@_check_attr:nV {#1} \l_@@_sort_attr_str
      \tl_set:Nx \l_@@_sort_type_tl
        { \@@_get_type:nV {#1} \l_@@_sort_attr_str }
      \clist_if_in:nVF
      { str, int, date, fp } { \l_@@_sort_type_tl } {
        \msg_fatal:nnx { dbshow } { unsupported-sort-type }
          { \l_@@_sort_type_tl }
      }
      \str_if_eq:eeT { \l_@@_sort_type_tl } { date }
        { \tl_set:Nn \l_@@_sort_type_tl { str } }
%    \end{macrocode}
% Set operands and compare function.
%    \begin{macrocode}
      \cs_set_eq:Nc \l_@@_sort_tmpa_tl
        { g_@@_data_#1_\l_@@_sort_attr_str _##1 }
      \cs_set_eq:Nc \l_@@_sort_tmpb_tl
        { g_@@_data_#1_\l_@@_sort_attr_str _##2 }
      \cs_set_eq:Nc \@@_compare
        { \l_@@_sort_type_tl _compare:VNVTF }
%    \end{macrocode}
% Compare two operands and if they are equal, compare the next attribute.
%    \begin{macrocode}
      \@@_compare \l_@@_sort_tmpa_tl
        \l_@@_sort_same_op_tl \l_@@_sort_tmpb_tl
        { \sort_return_same: }
      {
        \@@_compare \l_@@_sort_tmpa_tl
          \l_@@_sort_swap_op_tl \l_@@_sort_tmpb_tl
          { \sort_return_swapped: }
        {
          \int_compare:nTF
            { \l_@@_sort_len_int = \l_@@_sort_tmp_int }
            { \sort_return_same: }
            { \@@_sort_single: }
        }
      }
    }
    \@@_sort_single:
  }
}
%    \end{macrocode}
% \end{macro}
%
% \subsection{Display Data}
%
% \begin{macro}{\@@_clist_wrapper:NNNNn}
% Wrap the clist item with \meta{before code} and \meta{after code}.
% \begin{arguments}
%   \item \meta{before code tl}
%   \item \meta{after code tl}
%   \item \meta{item code cs}
%   \item \meta{exp boolean var}
%   \item \meta{item}
% \end{arguments}
%    \begin{macrocode}
\cs_new:Nn \@@_clist_wrapper:NNNNn {
  \bool_if:NTF #4
    { \exp_not:n { { #1\exp_args:Nx#3{#5}#2 }, } }
    { \exp_not:n { { #1#3{#5}#2 }, } }
}
%    \end{macrocode}
% \end{macro}
%
% \begin{macro}{\@@_clist_use:NNNNNN, \@@_clist_use:cccccc}
% Display a comma list.
% \begin{arguments}
%   \item \meta{data clist}
%   \item \meta{separator clist}
%   \item \meta{before code tl}
%   \item \meta{after code tl}
%   \item \meta{item code cs}
%   \item \meta{exp boolean var}
% \end{arguments}
%    \begin{macrocode}
\cs_new:Nn \@@_clist_use:NNNNNN {
  \int_case:nnF { \clist_count:N #2 } {
    { 1 } {
      \clist_use:xx
        { \clist_map_tokens:Nn #1 { \@@_clist_wrapper:NNNNn #3#4#5#6 } }
        { \clist_item:Nn #2 { 1 } }
    }
    { 3 } {
      \clist_use:xxxx
        { \clist_map_tokens:Nn #1 { \@@_clist_wrapper:NNNNn #3#4#5#6 } }
        { \clist_item:Nn #2 { 1 } }
        { \clist_item:Nn #2 { 2 } }
        { \clist_item:Nn #2 { 3 } }
    }
  } {
    \@@_sep_fatal:xxx
      { 1~or~3 }
      { \clist_count:N #2 }
      { \clist_use:Nn #2 { ,~ } }
  }
}
\cs_generate_variant:Nn \@@_clist_use:NNNNNN { cccccc }
%    \end{macrocode}
% \end{macro}
%
% \begin{macro}{\@@_date_use:nNN, \@@_date_use:ncc}
% Display date.
% \begin{arguments}
%   \item \meta{data}
%   \item \meta{separator clist}
%   \item \meta{zfill boolean}
% \end{arguments}
%    \begin{macrocode}
\cs_new:Nn \@@_date_use:nNN {
  \int_case:nnF { \clist_count:N #2 } {
    { 1 } {
      \bool_if:NTF {#3}
        { \__dbdate_use_zfill:nf }
        { \__dbdate_use:nf }
        {#1}
        { \clist_item:Nn #2 { 1 } }
    }
    { 4 } {
      \bool_if:NTF {#3}
        { \__dbdate_use_zfill:nffff }
        { \__dbdate_use:nffff }
        {#1}
        { \clist_item:Nn #2 { 1 } }
        { \clist_item:Nn #2 { 2 } }
        { \clist_item:Nn #2 { 3 } }
        { \clist_item:Nn #2 { 4 } }
    }
  } {
    \@@_sep_fatal:xxx
      { 1~or~4 }
      { \clist_count:N #2 }
      { \clist_use:Nn #2 { ,~ } }
  }
}
\cs_generate_variant:Nn \@@_date_use:nNN { ncc }
%    \end{macrocode}
% \end{macro}
%
% \begin{macro}{\@@_use_data:nnnn, \@@_use_data_raw:nnnn}
% Display Data. \cs{@@_use_data:nnnn} wrap the \meta{attr} data and
% \cs{@@_use_data_raw:nnnn} display the underlying data.
% \begin{arguments}
%   \item \meta{database}
%   \item \meta{attr}
%   \item \meta{index}
%   \item \meta{style}
% \end{arguments}
%    \begin{macrocode}
\cs_new:Nn \@@_use_data:nnnn {
  \bool_if:cT { g_@@_style_attr_exp_bool_#4_#1_#2 }
    { \exp_args:Nnx }
  \use:c { @@_style_attr_code_#4_#1_#2:n }
  { \@@_use_data_raw:nnnn {#1} {#2} {#3} {#4} }
}
\cs_new:Nn \@@_use_data_raw:nnnn {
  \str_case_e:nn
  { \prop_item:cn { g_@@_attr_type_prop_#1 } {#2} } {
    { str }   { \str_use:c { g_@@_data_#1_#2_#3 } }
    { tl }    { \tl_use:c  { g_@@_data_#1_#2_#3 } }
    { int }   { \int_use:c { g_@@_data_#1_#2_#3 } }
    { fp }    { \fp_use:c  { g_@@_data_#1_#2_#3 } }
    { clist } {
      \@@_clist_use:cccccc { g_@@_data_#1_#2_#3 }
        { g_@@_style_attr_sep_#4_#1_#2 }
        { g_@@_style_attr_item_before_tl_#4_#1_#2 }
        { g_@@_style_attr_item_after_ tl_#4_#1_#2 }
        {   @@_style_clist_item_code_    #4_#1_#2:n }
        { g_@@_style_clist_item_exp_bool_#4_#1_#2 }
    }
    { date }  {
      \cs_if_exist_use:cTF { @@_style_date_format_code_#4_#1_#2:xxx } {
        { \__dbdate_get_year:n  { g_@@_data_#1_#2_#3 } }
        { \__dbdate_get_month:n { g_@@_data_#1_#2_#3 } }
        { \__dbdate_get_day:n   { g_@@_data_#1_#2_#3 } }
      } {
        \@@_date_use:ncc { g_@@_data_#1_#2_#3 }
          { g_@@_style_attr_sep_#4_#1_#2 }
          { g_@@_style_date_zfill_bool_#4_#1_#2 }
      }
    }
  }
}
%    \end{macrocode}
% \end{macro}
%
% \begin{macro}{\dbDatabase, \dbFilterName, \dbFilterInfo}
% Define context macros.
% \begin{arguments}
%   \item \meta{database}
%   \item \meta{filter}
% \end{arguments}
%    \begin{macrocode}
\cs_new_protected:Nn \@@_show_set_macro:nn {
  \tl_set:Nn    \dbDatabase   {#1}
  \tl_set:Nn    \dbFilterName {#2}
  \tl_set_eq:Nc \dbFilterInfo { g_@@_filter_info_tl_#1_#2 }
}
%    \end{macrocode}
% \end{macro}
%
% \begin{macro}{\@@_show_filter:nnN}
% Filter records by executing the hook function in the running sequence and
% then testing the result boolean.
% \begin{arguments}
%   \item \meta{database}
%   \item \meta{filter}
%   \item \meta{index clist}
% \end{arguments}
%    \begin{macrocode}
\cs_new_protected:Nn \@@_show_filter:nnN {
%    \end{macrocode}
% \begin{arguments}[2]
%   \item \meta{index}
% \end{arguments}
%    \begin{macrocode}
  \int_step_inline:nn { \@@_get_counter:n {#1} } {
    \seq_if_exist:cTF { g_@@_filter_run_seq_#1_#2 } {
%    \end{macrocode}
% \begin{arguments}[3]
%   \item \meta{hook}
% \end{arguments}
%    \begin{macrocode}
      \seq_map_inline:cn { g_@@_filter_run_seq_#1_#2 } {
        \use:c {####1} {##1}
      }
      \exp_args:Nv
      \bool_if:nT { g_@@_filter_bool_tl_#1_#2 }
        { \clist_put_right:Nn #3 {##1} }
    } { \clist_put_right:Nn #3 {##1} }
  }
}
%    \end{macrocode}
% \end{macro}
%
% \begin{macro}{\@@_show_set_counter:N, \dbalph, \dbAlph, \dbarabic, \dbroman,
% \dbRoman}
% Define macros to display counter.
% \begin{arguments}
%   \item \meta{int var}
% \end{arguments}
%    \begin{macrocode}
\cs_new_protected:Nn \@@_show_set_counter:N {
  \tl_set:Nx \dbalph   { \int_to_alph:n   {#1} }
  \tl_set:Nx \dbAlph   { \int_to_Alph:n   {#1} }
  \tl_set:Nx \dbarabic { \int_to_arabic:n {#1} }
  \tl_set:Nx \dbRoman  { \int_to_Roman:n  {#1} }
  \tl_set:Nx \dbroman  { \int_to_roman:n  {#1} }
}
%    \end{macrocode}
% \end{macro}
%
% \begin{macro}{\@@_show_set_if_last:NN, \dbIfLastT, \dbIfLastF, \dbIfLastTF}
% Define conditional to check if the current item is the last item.
% \begin{arguments}
%   \item \meta{current index}
%   \item \meta{count}
% \end{arguments}
%    \begin{macrocode}
\cs_new_protected:Nn \@@_show_set_if_last:NN {
  \prg_set_conditional:Nnn \@@_show_if_last: { T, F, TF } {
    \int_compare:nNnTF {#1} = {#2}
      { \prg_return_true: }
      { \prg_return_false: }
  }
  \cs_set_eq:NN \dbIfLastT  \@@_show_if_last:T
  \cs_set_eq:NN \dbIfLastF  \@@_show_if_last:F
  \cs_set_eq:NN \dbIfLastTF \@@_show_if_last:TF
}
%    \end{macrocode}
% \end{macro}
%
% \begin{macro}{\@@_show_item:nn, \dbIndex, \dbuse}
% Display single record.
% \begin{arguments}
%   \item \meta{style}
%   \item \meta{database}
%   \item \meta{index clist}
% \end{arguments}
%    \begin{macrocode}
\cs_new_protected:Nn \@@_show_item:nnN {
  \int_zero_new:N \l_@@_show_int
  \int_zero_new:N \l_@@_show_count_int
  \int_set:Nn \l_@@_show_count_int { \clist_count:N #3 }
  \tl_clear_new:N \l_@@_item_tl
%    \end{macrocode}
% \begin{arguments}[2]
%   \item \meta{index}
% \end{arguments}
%    \begin{macrocode}
  \clist_map_inline:Nn #3 {
    \int_incr:N \l_@@_show_int
    \@@_show_set_if_last:NN \l_@@_show_int \l_@@_show_count_int
    \@@_show_set_counter:N  \l_@@_show_int
    \tl_set:Nn \dbIndex {##1}
%    \end{macrocode}
% \begin{arguments}[3]
%   \item \meta{attr}
% \end{arguments}
%    \begin{macrocode}
    \cs_set:Npn \dbuse ####1 {
      \@@_check_attr:nn {#2} {####1}
      \tl_use:c { g_@@_style_attr_before_tl_#1_#2_####1 }
      \@@_use_data:nnnn {#2} {####1} {##1} {#1}
      \tl_use:c { g_@@_style_attr_after_ tl_#1_#2_####1 }
    }
    \bool_if:cTF { g_@@_style_item_exp_bool_#1_#2 } {
      \tl_put_right:Nx \l_@@_item_tl
        { \tl_use:c { g_@@_style_item_before_tl_#1_#2 } }
      \tl_put_right:Nx \l_@@_item_tl
        { \tl_use:c { g_@@_style_item_       tl_#1_#2 } }
      \tl_put_right:Nx \l_@@_item_tl
        { \tl_use:c { g_@@_style_item_after_ tl_#1_#2 } }
    } {
      \tl_use:c { g_@@_style_item_before_tl_#1_#2 }
      \tl_use:c { g_@@_style_item_       tl_#1_#2 }
      \tl_use:c { g_@@_style_item_after_ tl_#1_#2 }
    }
  }
  \l_@@_item_tl
}
%    \end{macrocode}
% \end{macro}
%
% \begin{macro}{\@@_show_set_cond:N, \dbIfEmptyT, \dbIfEmptyF, \dbIfEmptyTF}
% Define conditional to test if the number of records to show is zero.
% \begin{arguments}
%   \item \meta{index clist}
% \end{arguments}
%    \begin{macrocode}
\cs_new_protected:Nn \@@_show_set_cond:N {
  \prg_set_conditional:Nnn \@@_if_empty: { T, F, TF } {
    \clist_if_empty:NTF #1
      { \prg_return_true: }
      { \prg_return_false: }
  }
  \cs_set_eq:NN \dbIfEmptyT  \@@_if_empty:T
  \cs_set_eq:NN \dbIfEmptyF  \@@_if_empty:F
  \cs_set_eq:NN \dbIfEmptyTF \@@_if_empty:TF
}
%    \end{macrocode}
% \end{macro}
%
% \changes{1.5}{2022-01-15}{Fix bug}{\cs{dbIfEmptyF} undefined}
% \begin{macro}{\@@_show:nnn, \@@_show:nnv}
% First filter records and sort them if needed and display at last.
% \begin{arguments}
%   \item \meta{style}
%   \item \meta{database}
%   \item \meta{filter}
% \end{arguments}
%    \begin{macrocode}
\cs_new_protected:Nn \@@_show:nnn {
  \@@_show_set_macro:nn {#2} {#3}
  \clist_clear_new:N \l_@@_show_index_clist
  \@@_show_filter:nnN {#2} {#3} \l_@@_show_index_clist
  \clist_if_empty:cF { g_@@_sort_clist_#1_#2 }
    { \@@_sort:nNn {#2} \l_@@_show_index_clist {#1} }
  \@@_show_set_cond:N \l_@@_show_index_clist
  \tl_use:c { g_@@_style_before_tl_#1_#2 }
  \@@_show_item:nnN {#1} {#2} \l_@@_show_index_clist
  \tl_use:c { g_@@_style_after_tl_#1_#2 }
}
\cs_generate_variant:Nn \@@_show:nnn { nnv }
%    \end{macrocode}
% \end{macro}
%
% \begin{macro}{\dbshow}
% User insterface to display the \meta{database} with \meta{style}.
% \begin{arguments}
%   \item \meta{style}
%   \item \meta{database}
% \end{arguments}
%    \begin{macrocode}
\NewDocumentCommand { \dbshow } { m m } {
  \@@_check_database:n {#2}
  \@@_check_style:nn   {#1} {#2}
  \@@_check_filter:nv  {#2} { g_@@_filter_#1_#2 }
  \@@_show:nnv {#1}    {#2} { g_@@_filter_#1_#2 }
}
%    \end{macrocode}
% \end{macro}
%
% \subsection{Date Type}
%
%    \begin{macrocode}
%<@@=dbdate>
%    \end{macrocode}
%
% \begin{arguments}
%   \item \meta{date}
%   \item \meta{date sep}
% \end{arguments}
%    \begin{macrocode}
\msg_new:nnn { dbshow } { wrong-date-sep } {
  can~not~parse~the~date~'#1'~with~the~global~date~separator~'#2'~
  \msg_line_context:.~Please~set~the~correct~date~separator~with~
  \dbdatesep.
}
%    \end{macrocode}
%
% \begin{macro}{\@@_check_date_sep:nn, \@@_check_date_sep:Vn}
% Check if the global date separator is valid.
% \begin{arguments}
%   \item \meta{date}
%   \item \meta{date sep}
% \end{arguments}
%    \begin{macrocode}
\cs_new:Nn \@@_check_date_sep:nn {
  \int_zero_new:N \l_@@_sep_int
  \tl_map_inline:nn {#1} {
    \tl_if_eq:nnT {#2} {##1} { \int_incr:N \l_@@_sep_int }
    \int_compare:nNnT { \l_@@_sep_int } > { 2 } { \tl_map_break: }
  }
  \int_compare:nNnF { \l_@@_sep_int } = { 2 }
    { \msg_fatal:nnnn { dbshow } { wrong-date-sep } {#1} {#2} }
}
\cs_generate_variant:Nn \@@_check_date_sep:nn { nV }
%    \end{macrocode}
% \end{macro}
%
% \begin{macro}[pTF]{\@@_if_leap:n}
% Check if the year is leap.
% \begin{arguments}
%   \item \meta{year}
% \end{arguments}
%    \begin{macrocode}
\prg_new_conditional:Nnn \@@_if_leap:n { T, F, TF, p } {
  \bool_if:nTF {
    \int_compare_p:nNn { \int_mod:nn {#1} { 400 } } = { 0 } ||
    (!\int_compare_p:nNn { \int_mod:nn {#1} { 100 } } = { 0 } &&
      \int_compare_p:nNn { \int_mod:nn {#1} { 4 } } = { 0 })
  } { \prg_return_true: } { \prg_return_false: }
}
%    \end{macrocode}
% \end{macro}
%
% \begin{variable}{\c_@@_month_clist}
% Number of days of every month.
%    \begin{macrocode}
\clist_const:Nn \c_@@_month_clist
  { 31, 28, 31, 30, 31, 30, 31, 31, 30, 31, 30, 31 }
%    \end{macrocode}
% \end{variable}
%
% \begin{macro}{\@@_to_int:nnnN}
% Transform date to integer relative to |1971-01-01|.
% \begin{arguments}
%   \item \meta{year}
%   \item \meta{month}
%   \item \meta{day}
%   \item \meta{int var} to store the result
% \end{arguments}
%    \begin{macrocode}
\cs_new_protected:Nn \@@_to_int:nnnN {
  \int_zero_new:N \l_@@_ans_int
  \int_zero_new:N \l_@@_tmpa_int
  \int_zero_new:N \l_@@_tmpb_int
  \int_set:Nn \l_@@_ans_int { #3 - 1 }
  \int_step_inline:nn { #2 - 1 } {
    \int_add:Nn \l_@@_ans_int {
      \clist_item:Nn \c_@@_month_clist {##1}
    }
    \bool_if:nT {
      \int_compare_p:nNn {##1} = { 2 } &&
      \@@_if_leap_p:n {#1}
    } { \int_incr:N \l_@@_ans_int }
  }
  \int_add:Nn \l_@@_ans_int { 365 * (#1 - 1971) }
  \int_add:Nn \l_@@_ans_int {
    \int_div_truncate:nn { #1 - 1 } { 4 } -
    \int_div_truncate:nn { 1971 } { 4 }
  }
  \int_sub:Nn \l_@@_ans_int {
    \int_div_truncate:nn { #1 - 1 } { 100 } -
    \int_div_truncate:nn { 1971 } { 100 }
  }
  \int_add:Nn \l_@@_ans_int {
    \int_div_truncate:nn { #1 - 1 } { 400 } -
    \int_div_truncate:nn { 1971 } { 400 }
  }
  \int_set_eq:NN #4 \l_@@_ans_int
}
%    \end{macrocode}
% \end{macro}
%
% \begin{macro}{\@@_to_int:NNNN, \@@_to_int:cccN}
% Transform date to integer relative to |1971-01-01|.
% \begin{arguments}
%   \item \meta{year int var}
%   \item \meta{month int var}
%   \item \meta{day int var}
%   \item \meta{int var} to store the result
% \end{arguments}
%    \begin{macrocode}
\cs_new_protected:Nn \@@_to_int:NNNN {
  \@@_to_int:nnnN {#1} {#2} {#3} #4
}
\cs_generate_variant:Nn \@@_to_int:NNNN { cccN }
%    \end{macrocode}
% \end{macro}
%
% \begin{macro}{\@@_to_int:nN}
% Transform date to integer relative to |1971-01-01|.
% \begin{arguments}
%   \item \meta{date var}
%   \item \meta{int var} to store the result
% \end{arguments}
%    \begin{macrocode}
\cs_new:Nn \@@_to_int:nN {
  \@@_to_int:cccN
    { @@_year_#1 }
    { @@_month_#1 }
    { @@_day_#1 }
    #2
}
%    \end{macrocode}
% \end{macro}
%
% \begin{macro}{\@@_set_val:n, \@@_gset_val:n}
% Set the value of \meta{data var} to |yyyy/mm/dd|.
% \begin{arguments}
%   \item \meta{date var}
% \end{arguments}
%    \begin{macrocode}
\cs_new_protected:Nn \@@_set_val:n {
  \tl_set:cx {#1} { \@@_use_zfill:nn {#1} { \g_@@_sep_tl } }
}
\cs_new_protected:Nn \@@_gset_val:n {
  \tl_gset:cx {#1} { \@@_use_zfill:nn {#1} { \g_@@_sep_tl } }
}
%    \end{macrocode}
% \end{macro}
%
% \begin{macro}{\@@_init:n, \@@_ginit:n}
% Initialize \meta{date var}.
% \begin{arguments}
%   \item \meta{date var}
% \end{arguments}
%    \begin{macrocode}
\cs_new_protected:Nn \@@_init:n {
  \@@_set:nnnn {#1} { 1971 } { 1 } { 1 }
  \@@_set_val:n {#1}
}
\cs_new_protected:Nn \@@_ginit:n {
  \@@_gset:nnnn {#1} { 1971 } { 1 } { 1 }
  \@@_gset_val:n {#1}
}
%    \end{macrocode}
% \end{macro}
%
% \begin{macro}{\@@_new:n, \@@_new:x}
% Create a new date variable.
% \begin{arguments}
%   \item \meta{date var}
% \end{arguments}
%    \begin{macrocode}
\cs_new_protected:Nn \@@_new:n {
  \int_new:c { @@_year_#1 }
  \int_new:c { @@_month_#1 }
  \int_new:c { @@_day_#1 }
  \@@_ginit:n {#1}
}
\cs_generate_variant:Nn \@@_new:n { x }
%    \end{macrocode}
% \end{macro}
%
% \begin{macro}{\@@_clear_new:n, \@@_gclear_new:n, \@@_clear_new:x,
% \@@_gclear_new:x}
% Clear or create a new date variable.
% \begin{arguments}
%   \item \meta{date var}
% \end{arguments}
%    \begin{macrocode}
\cs_new_protected:Nn \@@_clear_new:n {
  \int_zero_new:c { @@_year_#1 }
  \int_zero_new:c { @@_month_#1 }
  \int_zero_new:c { @@_day_#1 }
  \@@_init:n {#1}
}
\cs_generate_variant:Nn \@@_clear_new:n { x }
\cs_new_protected:Nn \@@_gclear_new:n {
  \int_gzero_new:c { @@_year_#1 }
  \int_gzero_new:c { @@_month_#1 }
  \int_gzero_new:c { @@_day_#1 }
  \@@_ginit:n {#1}
}
\cs_generate_variant:Nn \@@_gclear_new:n { x }
%    \end{macrocode}
% \end{macro}
%
% \begin{macro}{\@@_get_year:n, \@@_get_year:x, \@@_get_month:n,
% \@@_get_month:x, \@@_get_day:n, \@@_get_day:x}
% Extract year, month or day from \meta{date var}.
% \begin{arguments}
%   \item \meta{date var}
% \end{arguments}
%    \begin{macrocode}
\cs_new:Nn \@@_get_year:n {
  \int_use:c { @@_year_#1 }
}
\cs_new:Nn \@@_get_month:n {
  \int_use:c { @@_month_#1 }
}
\cs_new:Nn \@@_get_day:n {
  \int_use:c { @@_day_#1 }
}
\cs_generate_variant:Nn \@@_get_year:n  { x }
\cs_generate_variant:Nn \@@_get_month:n { x }
\cs_generate_variant:Nn \@@_get_day:n   { x }
%    \end{macrocode}
% \end{macro}
%
% \begin{macro}{\@@_set:nnnn, \@@_gset:nnnn}
% Set the value of \meta{date var}.
% \begin{arguments}
%   \item \meta{date var}
%   \item \meta{year}
%   \item \meta{month}
%   \item \meta{day}
% \end{arguments}
%    \begin{macrocode}
\cs_new_protected:Nn \@@_set:nnnn {
  \int_set:cn { @@_year_#1 }  {#2}
  \int_set:cn { @@_month_#1 } {#3}
  \int_set:cn { @@_day_#1 }   {#4}
  \@@_set_val:n {#1}
}
\cs_new_protected:Nn \@@_gset:nnnn {
  \int_gset:cn { @@_year_#1 }  {#2}
  \int_gset:cn { @@_month_#1 } {#3}
  \int_gset:cn { @@_day_#1 }   {#4}
  \@@_gset_val:n {#1}
}
%    \end{macrocode}
% \end{macro}
%
% \begin{macro}{\@@_set_sep:n, \@@_set:w, \@@_gset:w, \dbdatesep}
% Set internal date separator. Default is |/|.
% \begin{arguments}
%   \item \meta{separator}
% \end{arguments}
%    \begin{macrocode}
\cs_new_protected:Nn \@@_set_sep:n {
  \tl_gset:Nn \g_@@_sep_tl { #1 }
%    \end{macrocode}
% Set the value of \meta{date var}.
% \begin{arguments}[2]
%   \item \meta{date var}
%   \item \meta{year}
%   \item \meta{month}
%   \item \meta{day}
% \end{arguments}
%    \begin{macrocode}
  \cs_gset_protected:Npn \@@_set:w ##1\@@_sep##2#1##3#1##4\@@_stop {
    \@@_clear_new:n {##1}
    \@@_set:nnnn {##1} {##2} {##3} {##4}
  }
  \cs_gset_protected:Npn \@@_gset:w ##1\@@_sep##2#1##3#1##4\@@_stop {
    \@@_gclear_new:n {##1}
    \@@_gset:nnnn {##1} {##2} {##3} {##4}
  }
}
\cs_gset_eq:NN \dbdatesep \@@_set_sep:n
\dbdatesep{/}
%    \end{macrocode}
% \end{macro}
%
% \begin{macro}{\@@_set:nn, \@@_set:xx, \@@_gset:nn, \@@_gset:xx}
% Set the value of \meta{date var}.
% \begin{arguments}
%   \item \meta{date var}
%   \item \meta{date}
% \end{arguments}
%    \begin{macrocode}
\cs_new_protected:Nn \@@_set:nn {
  \@@_check_date_sep:nV {#2} \g_@@_sep_tl
  \@@_set:w #1\@@_sep#2\@@_stop
}
\cs_generate_variant:Nn \@@_set:nn { xx }
\cs_new_protected:Nn \@@_gset:nn {
  \@@_check_date_sep:nV {#2} \g_@@_sep_tl
  \@@_gset:w #1\@@_sep#2\@@_stop
}
\cs_generate_variant:Nn \@@_gset:nn { xx }
%    \end{macrocode}
% \end{macro}
%
% \begin{macro}{\@@_sub:nnN}
% Calculate the number of days between \meta{date var2} and \meta{date var1}.
% \begin{arguments}
%   \item \meta{date var1}
%   \item \meta{date var2}
%   \item \meta{int var} to store the result
% \end{arguments}
%    \begin{macrocode}
\cs_new_protected:Nn \@@_sub:nnN {
  \int_zero_new:N \l_@@_sub_tmpa_int
  \int_zero_new:N \l_@@_sub_tmpb_int
  \@@_to_int:nN {#1} \l_@@_sub_tmpa_int
  \@@_to_int:nN {#2} \l_@@_sub_tmpb_int
  \int_set:Nn #3 { \l_@@_sub_tmpa_int - \l_@@_sub_tmpb_int }
}
%    \end{macrocode}
% \end{macro}
%
% \begin{macro}{\@@_show_two:N, \@@_show_two:c}
% Prepend 0 to single digit.
%    \begin{macrocode}
\cs_new:Nn \@@_show_two:N {
  \int_compare:nNnTF {#1} > { 9 }
    { \int_use:N #1 } { 0\int_use:N #1 }
}
\cs_generate_variant:Nn \@@_show_two:N { c }
%    \end{macrocode}
% \end{macro}
%
% \begin{macro}{\@@_use:nnnnn, \@@_use:nffff, \@@_use_zfill:nnnnn,
% \@@_use_zfill:nffff}
% Display date.
% \begin{arguments}
%   \item \meta{date var}
%   \item \meta{separator 1}
%   \item \meta{separator 2}
%   \item \meta{separator 3}
%   \item \meta{separator 4}
% \end{arguments}
%    \begin{macrocode}
\cs_new:Nn \@@_use:nnnnn {
  #2\int_use:c { @@_year_#1 }
  #3\int_use:c { @@_month_#1 }
  #4\int_use:c { @@_day_#1 }#5
}
\cs_generate_variant:Nn \@@_use:nnnnn { nffff }
\cs_new:Nn \@@_use_zfill:nnnnn {
  #2\int_use:c { @@_year_#1 }
  #3\@@_show_two:c { @@_month_#1 }
  #4\@@_show_two:c { @@_day_#1 }#5
}
\cs_generate_variant:Nn \@@_use_zfill:nnnnn { nffff }
%    \end{macrocode}
% \end{macro}
%
% \begin{macro}{\@@_use:nn, \@@_use:nf, \@@_use_zfill:nn, \@@_use_zfill:nf}
% Display date with the same separator.
%    \begin{macrocode}
\cs_new:Nn \@@_use:nn {
  \@@_use:nnnnn {#1} {} {#2} {#2} {}
}
\cs_generate_variant:Nn \@@_use:nn { nf }
\cs_new:Nn \@@_use_zfill:nn {
  \@@_use_zfill:nnnnn {#1} {} {#2} {#2} {}
}
\cs_generate_variant:Nn \@@_use_zfill:nn { nf }
%    \end{macrocode}
% \end{macro}
%
% \begin{macro}{\@@_show:n}
% Show date in terminal.
% \begin{arguments}
%   \item \meta{date var}
% \end{arguments}
%    \begin{macrocode}
\cs_new_protected:Nn \@@_show:n {
  \exp_args:Nx \tl_show:n { >#1~=~\@@_use:nn {#1} { - } }
}
%    \end{macrocode}
% \end{macro}
%
% \begin{variable}{\dbtoday}
% Display date of today in |yyyy/mm/dd|.
%    \begin{macrocode}
\tl_set:Nn \dbtoday {
  \int_use:N \c_sys_year_int  \g_@@_sep_tl
  \int_use:N \c_sys_month_int \g_@@_sep_tl
  \int_use:N \c_sys_day_int
}
%    \end{macrocode}
% \end{variable}
%
%    \begin{macrocode}
\endinput
%</package>
%    \end{macrocode}
%
% \end{implementation}
%
% \Finale
%
\endinput
